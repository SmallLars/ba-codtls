\documentclass{beamer}
%\documentclass[hyperref={pdfpagelabels=false}]{beamer}

\mode<presentation>
{
\usetheme{Tzi}
%\setbeamercovered{transparent}
}
\setbeamertemplate{navigation symbols}{}

\usepackage[ngerman]{babel}
\usepackage[utf8]{inputenc}
\usepackage{times}
\usepackage[T1]{fontenc}
\usepackage{amsmath}
\usepackage{array}
\usepackage{multirow}
\usepackage{color}
\graphicspath{{pic/}{../pic/}}

\title[Anpassungen von DTLS]%Short Title
{% Long title
  Anpassungen von DTLS zur sicheren Kommunikation in eingeschränkten Umgebungen
}

\author[Lars Schmertmann]%Short Author
{
{Lars Schmertmann}\\
\vspace{.2cm}
{\scriptsize lars@tzi.org}
}
\institute{TZI, Universit\"{a}t Bremen, Deutschland}

\date[09.2013]%short
{{Kolloquium zur Bachelorarbeit\\
12.09.2013}}%long

\begin{document}

\setbeamercolor{postit}{fg=black,bg=yellow}

\begin{frame}
  \titlepage
\end{frame}

\begin{frame}{Gliederung}
  \begin{itemize}
    \item Motivation
    \item Hardware \& Umgebung
    \item DTLS
      \begin{itemize}
        \item Handshake
        \item Ausgewählte DTLS-Header
      \end{itemize}
    \item Mögliche Lösungen
    \begin{itemize}
      \item Handshake über CoAP
      \item Stateless Header Compression
    \end{itemize}
    \item Geeignete Ciphersuit
    \item Praktische Umsetzung
    \item Fazit
    \item Zeitplan
  \end{itemize}
\end{frame}

\begin{frame}{Motivation}
  \begin{itemize}
    \item Bachelorprojekt GOBI
    \begin{itemize}
      \item Sicherheit ist auch in eingeschränkten \newline Umgebungen notwendig
      \item Mikro-DTLS (eher Mikro-Sicherheit)
    \end{itemize}
    \item TLS und DTLS sind bewährte Standards
    \item DTLS komplexer als TLS
  \end{itemize}
\end{frame}

\begin{frame}{Hardware \& Umgebung}
  \begin{itemize}
    \item Econotag: mc13224v Development-Board
    \begin{itemize}
      \item Freescale MC13224v ARM7TDMI-S Microcontroller
      \item IEEE 802.15.4 Funkstandard
      \item AES Hardware-Engine
      \item 128 KiB Flash-Speicher
      \item 96 KiB RAM
    \end{itemize}
    \item IEEE 802.15.4 MTU: 127 Byte
    \begin{itemize}
      \item - 48 Byte 6LoWPAN-Header
      \item - 8 Byte UDP-Header
      \item = 71 Byte Nutzdaten
    \end{itemize}
    \item SmartAppContiki = Contiki + Erbium + CoAP 13
    \begin{itemize}
      \item 81 KiB (bei angepasster Konfiguration)
    \end{itemize}
  \end{itemize}
\end{frame}

\begin{frame}{DTLS}{Handshake}
  \begin{scriptsize}
  \tt ~~~~~~~~~~~~~~~~Client~~~~~~~~~~~~~Server\\
  \tt ~~~~~~~~~~~~~~~~---------~~~~~~~~~~~~~---------\\
  \tt ~~~~~~~~~~~ClientHello~~--0-->\\
  \tt ~~~~~~~~~~~~~~~~~~~~~~~~~~~~~<--0--~~HelloVerifyRequest~($ \supset $~cookie)\\
  \tt ClientHello~(+~cookie)~~--1-->\\
  \tt ~~~~~~~~~~~~~~~~~~~~~~~~~~~~~<--1--~~ServerHello\\
  \tt ~~~~~~~~~~~~~~~~~~~~~~~~~~~~~<--2--~*Certificate\\
  \tt ~~~~~~~~~~~~~~~~~~~~~~~~~~~~~<--3--~*ServerKeyExchange\\
  \tt ~~~~~~~~~~~~~~~~~~~~~~~~~~~~~<--4--~*CertificateRequest\\
  \tt ~~~~~~~~~~~~~~~~~~~~~~~~~~~~~<--5--~~ServerHelloDone\\
  \tt ~~~~~~~~~~~Certificate*~--2-->\\
  \tt ~~~~~ClientKeyExchange~~--3-->\\
  \tt ~~~~~CertificateVerify*~--4-->\\
  \tt ~~~~[ChangeCipherSpec]~~---->\\
  \tt ~~~~~~~~~~~~~~Finished~~--5-->\\
  \tt ~~~~~~~~~~~~~~~~~~~~~~~~~~~~~<----~~[ChangeCipherSpec]\\
  \tt ~~~~~~~~~~~~~~~~~~~~~~~~~~~~~<--6--~~Finished\\
  \tt ~~~~~~Application~Data~~<----------->~~Application~Data
  \end{scriptsize}
\end{frame}

\begin{frame}{DTLS}{Ausgewählte DTLS-Header}
  \begin{columns}
    \column{.5\textwidth}
      struct \{\\
      \qquad ContentType type;\\
      \qquad ProtocolVersion version;\\
      \qquad \textcolor{blue}{uint16 epoch;}\\
      \qquad \textcolor{blue}{uint48 sequence\_number;}\\
      \qquad uint16 length;\\
      \qquad uint8 ~~payload[length];\\
      \} DTLS\_Record;\\
      ~\\
      = 13 Byte
    \column{.5\textwidth}
      struct \{\\
      \qquad HandshakeType msg\_type;\\
      \qquad uint24 length;\\
      \qquad \textcolor{blue}{uint16 message\_seq;}\\
      \qquad \textcolor{blue}{uint24 fragment\_offset;}\\
      \qquad \textcolor{blue}{uint24 fragment\_length;}\\
      \qquad uint8 ~~payload[f.\_length];\\
      \} Handshake;\\
      ~\\
      = 12 Byte
  \end{columns}
\end{frame}

\begin{frame}{Mögliche Lösungen}
  Vorschläge im Entwurf von K. Hartke und O. Bergmann:\\
  http://tools.ietf.org/html/draft-hartke-core-codtls-02\\
  \begin{itemize}
    \item Handshake über CoAP
    \item Stateless Header Compression
  \end{itemize}
\end{frame}

\begin{frame}{Mögliche Lösungen}{Handshake über CoAP - Teil 1}
  \tt ~~~~~~POST /dtls ---->\\
  \tt ~ClientHello\\
  \tt ~~~~~~~~~~~~~~~~~<---- 4.01 Unauthorized\\
  \tt ~~~~~~~~~~~~~~~~~~~~~~~~~~HelloVerifyRequest\\
  \tt ~~~~~~POST /dtls ---->\\
  \tt ~ClientHello\\
  \tt (mit cookie)\\
  \tt ~~~~~~~~~~~~~~~~~<---- 2.01 Created\\
  \tt ~~~~~~~~~~~~~~~~~~~~~~~~~~ServerHello (S=X)\\
  \tt ~~~~~~~~~~~~~~~~~~~~~~~~~*Certificate\\
  \tt ~~~~~~~~~~~~~~~~~~~~~~~~~*ServerKeyExchange\\
  \tt ~~~~~~~~~~~~~~~~~~~~~~~~~*CertificateRequest\\
  \tt ~~~~~~~~~~~~~~~~~~~~~~~~~~ServerHelloDone
\end{frame}
\begin{frame}{Mögliche Lösungen}{Handshake über CoAP - Teil 2}
  \tt ~~~~~~~POST /dtls/X ---->\\
  \tt ~~~~~~Certificate*\\
  \tt ClientKeyExchange\\
  \tt CertificateVerify*\\
  \tt ~ChangeCipherSpec\\
  \tt ~~~~~~~~~Finished\\
  ~\\
  \tt ~~~~~~~~~~~~~~~~~~~~~~<---- 2.04 Changed\\
  \tt ~~~~~~~~~~~~~~~~~~~~~~~~~~~~~~~ChangeCipherSpec\\
  \tt ~~~~~~~~~~~~~~~~~~~~~~~~~~~~~~~Finished\\
  ~\\
  \tt ~~~~~Application Data <---> Application Data
\end{frame}

\begin{frame}{Mögliche Lösungen}{Stateless Header Compression}
  \begin{columns}
    \column{.5\textwidth}
      struct \{\\
      \qquad uint8 ~~:1;\\
      \qquad RecordType type:2;\\
      \qquad Version version:2;\\
      \qquad Epoch epoch:3;\\
      \qquad uint8 ~~:3;\\
      \qquad SequenceNumber snr:3;\\
      \qquad RecordLength length:2;\\
      \qquad uint8 ~~payload[0];\\
      \} DTLSRecord\_t;\\
      ~\\
      = 2 - 15 Byte
    \column{.5\textwidth}
      struct \{\\
      \qquad ContentType type:6;\\
      \qquad ContentLength len:2;\\
      \qquad uint8 ~~payload[0];\\
      \} Content\_t;\\
      ~\\
      ~\\
      ~\\
      ~\\
      ~\\
      ~\\
      = 1 - 4 Byte
  \end{columns}
\end{frame}

\begin{frame}{Geeignete Ciphersuit}
  \begin{itemize}
    \item TLS\_PSK\_ECDH\_WITH\_AES\_128\_CCM\_8
  \end{itemize}
  Eigenschaften:
  \begin{itemize}
    \item Durch CCM wird keine Hash-Funktion \newline für den MAC benötigt
    \item Durch den Einsatz von CMAC mit PSK kann \newline auf HMAC mit SHA-256 verzichtet werden
  \end{itemize}
  Auswirkungen auf die Programmgröße:
  \begin{itemize}
    \item Einsparung durch AES in Hardware \newline und Nutzung von CCM + CMAC: \textasciitilde 3 KiB
  \end{itemize}
\end{frame}

\begin{frame}{Praktische Umsetzung}
  \begin{itemize}
    \item Ablage von Read-Only-Daten im Flash-Speicher
    \item Ein Handshake zur Zeit
    \begin{itemize}
     \item Stack mit Push und Clear im Flash-Speicher für "`Finished"'
    \end{itemize}
    \item Wechsel des Pre-shared Key nach Handshake
    \begin{itemize}
      \item Abrufbar über CoAP-URI
    \end{itemize}
    \item Ablage der Session-Daten im Flash-Speicher
    \item Implementierung einzelner Funktionen in Assembler
    \item Software-Update über CoAP
    \begin{itemize}
      \item Session-Daten bleiben erhalten
      \item Sequenznummer geht verloren
    \end{itemize}
  \end{itemize}
\end{frame}

\begin{frame}{Fazit}
  \begin{itemize}
    \item Implementierung ist derzeit 10 KiB groß
    \item 7,2 KiB setzen sich zusammen:
    \begin{itemize}
      \item 2,41 KiB ECC-Funktionen
      \item 0,95 KiB AES-Funktionen (CCM + CMAC)
      \item 0,80 KiB Flash-Speicher-Funktionen
      \item 0,79 KiB Session-Verwaltung
      \item 0,15 KiB Pseudo-Random-Funktion
      \item 1,78 KiB Handshake-Ressource
      \item 0,32 KiB Parse \& Send
    \end{itemize}
    \item Stack-Größe von 2 KiB reicht aus
    \item ECC-Multiplikation: \o 6,0 s
    \item Handshake: \o 14 s
  \end{itemize}
\end{frame}

\begin{frame}{Zeitplan}
  \begin{itemize}
    \item Es verbleiben 4 Wochen bis zum Vorlesungsbeginn
    \begin{itemize}
      \item 2 Wochen für Kapitel "`Praktische Umsetzung"'
      \item 1 Woche für Kapitel "`Vergleich"' und "`Fazit"'
      \item 1 Woche Feintuning
    \end{itemize}
    \item 14.10.2013 Abgabe der Arbeit
  \end{itemize}
\end{frame}

\begin{frame}{}
  \begin{center}
    \begin{LARGE}
      Vielen Dank für\\
      eure Aufmerksamkeit\\
      ~\\
      \includegraphics[scale=0.2]{pic/smilie.png}
    \end{LARGE}
  \end{center}
\end{frame}

\begin{frame}{Fragen}
  \begin{center}
    \begin{LARGE}
      \begin{tabular}{ccccc}
         & \textbf{?} & \textbf{?} & \textbf{?} & \\
        \textbf{?} & \textbf{?} &  & \textbf{?} & \textbf{?}\\
         &  &  & \textbf{?} & \textbf{?}\\
         &  & \textbf{?} & \textbf{?} & \\
         & \textbf{?} & \textbf{?} &  & \\
         &  &  &  & \\
         & \textbf{?} & \textbf{?} &  & \\
         & \textbf{?} & \textbf{?} &  & 
      \end{tabular}
    \end{LARGE}
  \end{center}
\end{frame}

\end{document}