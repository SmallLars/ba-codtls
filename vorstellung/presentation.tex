\documentclass{beamer}
%\documentclass[hyperref={pdfpagelabels=false}]{beamer}

\mode<presentation>
{
\usetheme{Tzi}
%\setbeamercovered{transparent}
}
\setbeamertemplate{navigation symbols}{}

\usepackage[ngerman]{babel}
\usepackage[utf8]{inputenc}
\usepackage{times}
\usepackage[T1]{fontenc}
\usepackage{amsmath}
\usepackage{array}
\usepackage{multirow}
\usepackage{color}
\graphicspath{{pic/}{../pic/}}

\title[Anpassungen von DTLS]%Short Title
{% Long title
  Vorstellung der Bachelorarbeit: Anpassungen von DTLS zur sicheren Kommunikation in eingeschränkten Umgebungen
}

\author[Lars Schmertmann]%Short Author
{
{Lars Schmertmann}\\
\vspace{.2cm}
{\scriptsize lars@tzi.org}
}
\institute{TZI, Universit\"{a}t Bremen, Deutschland}

\date[08.2013]%short
{{Kolloquium der AG Rechnernetze\\
09.08.2013}}%long

\begin{document}

\setbeamercolor{postit}{fg=black,bg=yellow}

\begin{frame}
  \titlepage
\end{frame}

\begin{frame}{Gliederung}
  \begin{itemize}
    \item Motivation
    \begin{itemize}
     \item Ausgewählte DTLS-Header
    \end{itemize}
    \item Mögliche Lösungen
    \begin{itemize}
      \item Handshake über CoAP
      \item Stateless Header Compression
    \end{itemize}
    \item Hardware
    \item Geeignetes Ciphersuit
  \end{itemize}

\end{frame}

\begin{frame}{Motivation}
  \begin{itemize}
    \item Sicherheit ist auch in eingeschränkten Umgebungen notwendig
    \item TLS und DTLS sind bewährte Standards
    \item DTLS komplexer als TLS
    \item Problematisch in eingeschränkten Umgebungen
    \item 13 Byte Header bei 127 Byte Payload in 6LoWPAN
    \item Hoher Aufwand während des Handshakes
  \end{itemize}
\end{frame}

\begin{frame}{Motivation}{Ausgewählte DTLS-Header}
  \begin{columns}
    \column{.5\textwidth}
      struct \{\\
      \qquad ContentType type;\\
      \qquad ProtocolVersion version;\\
      \qquad \textcolor{blue}{uint16 epoch;}\\
      \qquad \textcolor{blue}{uint48 sequence\_number;}\\
      \qquad uint16 length;\\
      \qquad uint8 ~~fragment[length];\\
      \} DTLS\_Record;
    \column{.5\textwidth}
      struct \{\\
      \qquad HandshakeType msg\_type;\\
      \qquad uint24 length;\\
      \qquad \textcolor{blue}{uint16 message\_seq;}\\
      \qquad \textcolor{blue}{uint24 fragment\_offset;}\\
      \qquad \textcolor{blue}{uint24 fragment\_length;}\\
      \qquad uint8 ~~fragment[length];\\
      \} Handshake;
  \end{columns}
\end{frame}

\begin{frame}{Mögliche Lösungen}
  Vorschlänge im Entwurf von k. Hartge und O. Bergmann:\\
  http://tools.ietf.org/html/draft-hartke-core-codtls-02\\
  \begin{itemize}
    \item Handshake über CoAP
    \item Stateless Header Compression
  \end{itemize}
\end{frame}

\begin{frame}{Mögliche Lösungen}{Handshake über CoAP - Teil 1}
  \tt ~~~~~~POST /dtls ---->\\
  \tt ~ClientHello\\
  \tt ~~~~~~~~~~~~~~~~~<---- 1.02 Verify\\
  \tt ~~~~~~~~~~~~~~~~~~~~~~~~~~HelloVerifyRequest\\
  \tt ~~~~~~POST /dtls ---->\\
  \tt ~ClientHello\\
  \tt (mit cookie)\\
  \tt ~~~~~~~~~~~~~~~~~<---- 2.01 Created\\
  \tt ~~~~~~~~~~~~~~~~~~~~~~~~~~ServerHello (S=X)\\
  \tt ~~~~~~~~~~~~~~~~~~~~~~~~~*Certificate\\
  \tt ~~~~~~~~~~~~~~~~~~~~~~~~~~ServerKeyExchange\\
  \tt ~~~~~~~~~~~~~~~~~~~~~~~~~*CertificateRequest\\
  \tt ~~~~~~~~~~~~~~~~~~~~~~~~~~ServerHelloDone
\end{frame}
\begin{frame}{Mögliche Lösungen}{Handshake über CoAP - Teil 2}
  \tt ~~~~~~~POST /dtls?s=X ---->\\
  \tt ~~~~~~Certificate*\\
  \tt ClientKeyExchange\\
  \tt CertificateVerify*\\
  \tt ~ChangeCipherSpec\\
  \tt ~~~~~~~~~Finished\\
  ~\\
  \tt ~~~~~~~~~~~~~~~~~~~~~~<---- 2.04 Changed\\
  \tt ~~~~~~~~~~~~~~~~~~~~~~~~~~~~~~~ChangeCipherSpec\\
  \tt ~~~~~~~~~~~~~~~~~~~~~~~~~~~~~~~Finished\\
  ~\\
  \tt ~~~~~Application Data <---> Application Data
\end{frame}

\begin{frame}{Mögliche Lösungen}{Stateless Header Compression}
  \begin{columns}
    \column{.5\textwidth}
      struct \{\\
      \qquad uint8 ~~:1;\\
      \qquad RecordType type:2;\\
      \qquad Version version:2;\\
      \qquad Epoch epoch:3;\\
      \qquad uint8 ~~:3;\\
      \qquad SequenceNumber snr:3;\\
      \qquad RecordLength length:2;\\
      \qquad uint8 ~~payload[0];\\
      \} DTLSRecord\_t;
    \column{.5\textwidth}
      struct \{\\
      \qquad ContentType type:6;\\
      \qquad ContentLength len:2;\\
      \qquad uint8 ~~payload[0];\\
      \} Content\_t;\\
      ~\\
      ~\\
      ~\\
      ~\\
      ~
  \end{columns}
\end{frame}

\begin{frame}{Hardware}
Redbee Econotag
AES-Hardware
\end{frame}

\begin{frame}{Geeignetes Ciphersuit}
  \begin{itemize}
    \item TLS\_PSK\_ECDH\_WITH\_AES\_128\_CCM\_8
  \end{itemize}
  Bei Speichermangel:
  \begin{itemize}
    \item TLS\_PSK\_WITH\_AES\_128\_CCM\_8
  \end{itemize}
  Für beide gilt:
  \begin{itemize}
    \item Verzicht auf HMAC mit SHA-256
    \item Nutzung von CBC-MAC
  \end{itemize}
  Größe des Binärcodes:
  \begin{itemize}
    \item Für HMAC und ECC jeweils knapp 3KiB
  \end{itemize}
\end{frame}

\end{document}