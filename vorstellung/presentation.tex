\documentclass{beamer}
%\documentclass[hyperref={pdfpagelabels=false}]{beamer}

\mode<presentation>
{
\usetheme{Tzi}
%\setbeamercovered{transparent}
}
\setbeamertemplate{navigation symbols}{}

\usepackage[english]{babel}
\usepackage[latin1]{inputenc}
\usepackage{times}
\usepackage[T1]{fontenc}
\usepackage{amsmath}
\usepackage{array}
\usepackage{multirow}
\graphicspath{{pic/}{../pic/}}

\title[Smart Objects Security]%Short Title
{% Long title
  Security Requirements for Managing Smart Objects in Home Automation
}

\author[Gerdes, Bergmann]%Short Author
{
{Stefanie Gerdes, {\bf Olaf Bergmann}}\\
\vspace{.2cm}
{\scriptsize \{gerdes$|$bergmann\}@tzi.org}
}
\institute{TZI, Universit\"{a}t Bremen, Germany}

\date[September 2012]%short
{{Smart Objects Resource Management (SmaRT)\\
Workshop at the 4th International Conference on\\
Mobile Networks and Management (MONAMI 2012)\\
September 26, 2012}}%long

\begin{document}

\setbeamercolor{postit}{fg=black,bg=yellow}

\begin{frame}
  \titlepage
\end{frame}

\end{document}


% Seite 1
% 
% Lars Schmertmann
% Anpassungen von DTLS zur sicheren Kommunikation in eingeschränkten Umgebungen
% Betreut von Olaf Bergmann
% 
% Seite 2
% 
% Motivation:
%   TLS/DTLS bewährte Standards
%   DTLS komplexer als TLS
%   Problematisch in eingeschränkten Umgebungen
%   5 / 13 Byte Header bei 127 Byte 6lowpan payload
% 
% \begin{figure}[ht]
%   \centering
%   \begin{lstlisting}[language=c]
%   struct {
%     ContentType type;
%     ProtocolVersion version;
%     uint16 epoch;                           // Nur bei DTLS
%     uint48 sequence_number;                 // Nur bei DTLS
%     uint16 length;
%     uint8  fragment[DTLS_Record.length];
%   } DTLS_Record;
%   \end{lstlisting}
%   \caption{Header des Record Layer Protokolls von TLS / DTLS}
%   \label{fig:recordlayer}
% \end{figure}
% 
% Seite 3
% 
% Lösungen
%   Verwendung von Stateless Headercompression [1]
%   Handshake über CoAP [1]
% \begin{figure}[ht]
%   \centering
%   \begin{lstlisting}[language=c]
%   struct {
%     HandshakeType msg_type;
%     uint24 length;
%     uint16 message_seq;                     // Nur bei DTLS
%     uint24 fragment_offset;                 // Nur bei DTLS
%     uint24 fragment_length;                 // Nur bei DTLS
%     uint8  fragment[Handshake.length];
%   } Handshake;
%   \end{lstlisting}
%   \caption{Header des Handshake Protokolls von TLS / DTLS}
%   \label{fig:handshakelayer}
% \end{figure}
%     separat Antwort
%     blockwise
% 
% Seite 4
% 
% Ciphersuit
%   TLS\_PSK\_ECDH\_WITH\_AES\_128\_CCM\_8
%   Bei Speichermangel: TLS\_PSK\_WITH\_AES\_128\_CCM\_8
%   Dabei wichtig HMAC -> CBC-MAC knapp 3kB für HMAC sparen [2]
% 
% Seite 5
% 
% Hardware
%   Redbee Econotag
%   AES-Modul nutzen
% 
% 
% 
% 
% 
% [1] http://tools.ietf.org/html/draft-hartke-core-codtls-02
% [2] http://tools.ietf.org/html/draft-tschofenig-lwig-tls-minimal-03
