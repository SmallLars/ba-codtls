\documentclass{beamer}
%\documentclass[hyperref={pdfpagelabels=false}]{beamer}

\mode<presentation>
{
\usetheme{Tzi}
%\setbeamercovered{transparent}
}
\setbeamertemplate{navigation symbols}{}

\usepackage[ngerman]{babel}
\usepackage[utf8]{inputenc}
\usepackage{times}
\usepackage[T1]{fontenc}
\usepackage{amsmath}
\usepackage{array}
\usepackage{multirow}
\usepackage{color}
\graphicspath{{pic/}{../pic/}}

\title[Anpassungen von DTLS]%Short Title
{% Long title
  Vorstellung der Bachelorarbeit: Anpassungen von DTLS zur sicheren Kommunikation in eingeschränkten Umgebungen
}

\author[Lars Schmertmann]%Short Author
{
{Lars Schmertmann}\\
\vspace{.2cm}
{\scriptsize lars@tzi.org}
}
\institute{TZI, Universit\"{a}t Bremen, Deutschland}

\date[08.2013]%short
{{Kolloquium der AG Rechnernetze\\
30.09.2013}}%long

\begin{document}

\setbeamercolor{postit}{fg=black,bg=yellow}

\begin{frame}
  \titlepage
\end{frame}

\begin{frame}{Gliederung}
  \begin{itemize}
    \item Motivation
    \item Hardware \& Umgebung
    \item DTLS
      \begin{itemize}
        \item Ausgewählte DTLS-Header
        \item Handshake
      \end{itemize}
    \item Mögliche Lösungen
    \begin{itemize}
      \item Stateless Header Compression
      \item Handshake über CoAP
    \end{itemize}
    \item Geeignetes Ciphersuit
  \end{itemize}
\end{frame}

\begin{frame}{Motivation}
  \begin{itemize}
    \item Projekt GOBI: Mikro-DTLS (eher Mikro-Sicherheit)
    \item Sicherheit ist auch in eingeschränkten \newline Umgebungen notwendig
    \item TLS und DTLS sind bewährte Standards
    \item DTLS komplexer als TLS
    \item Problematisch in eingeschränkten Umgebungen
  \end{itemize}
\end{frame}

\begin{frame}{Hardware \& Umgebung}
  \begin{itemize}
    \item Econotag: mc13224v Development-Board
    \begin{itemize}
      \item Freescale MC13224v ARM7TDMI-S Microcontroller
      \item IEEE 802.15.4 Funkstandard
      \item AES Hardware-Engine
      \item 128 KiB Flash-Speicher
      \item 96 KiB RAM
    \end{itemize}
    \item IEEE 802.15.4 MTU: 127 Byte
    \begin{itemize}
      \item - 48 Byte 6LoWPAN-Header
      \item - 8 Byte UDP-Header
      \item = 71 Byte Nutzdaten
    \end{itemize}
  \end{itemize}
\end{frame}

\begin{frame}{DTLS}{Ausgewählte DTLS-Header}
  \begin{columns}
    \column{.5\textwidth}
      struct \{\\
      \qquad ContentType type;\\
      \qquad ProtocolVersion version;\\
      \qquad \textcolor{blue}{uint16 epoch;}\\
      \qquad \textcolor{blue}{uint48 sequence\_number;}\\
      \qquad uint16 length;\\
      \qquad uint8 ~~fragment[length];\\
      \} DTLS\_Record;\\
      ~\\
      = 13 Byte
    \column{.5\textwidth}
      struct \{\\
      \qquad HandshakeType msg\_type;\\
      \qquad uint24 length;\\
      \qquad \textcolor{blue}{uint16 message\_seq;}\\
      \qquad \textcolor{blue}{uint24 fragment\_offset;}\\
      \qquad \textcolor{blue}{uint24 fragment\_length;}\\
      \qquad uint8 ~~fragment[length];\\
      \} Handshake;\\
      ~\\
      = 12 Byte
  \end{columns}
\end{frame}

\begin{frame}{DTLS}{Handshake}
  \begin{scriptsize}
  \tt ~~~~~~~~~~~~~~~~Client~~~~~~~~~~~~~Server\\
  \tt ~~~~~~~~~~~~~~~~---------~~~~~~~~~~~~~---------\\
  \tt ~~~~~~~~~~~ClientHello~~--0-->\\
  \tt ~~~~~~~~~~~~~~~~~~~~~~~~~~~~~<--0--~~HelloVerifyRequest~($ \supset $~cookie)\\
  \tt ClientHello~(+~cookie)~~--1-->\\
  \tt ~~~~~~~~~~~~~~~~~~~~~~~~~~~~~<--1--~~ServerHello\\
  \tt ~~~~~~~~~~~~~~~~~~~~~~~~~~~~~<--2--~*Certificate\\
  \tt ~~~~~~~~~~~~~~~~~~~~~~~~~~~~~<--3--~*ServerKeyExchange\\
  \tt ~~~~~~~~~~~~~~~~~~~~~~~~~~~~~<--4--~*CertificateRequest\\
  \tt ~~~~~~~~~~~~~~~~~~~~~~~~~~~~~<--5--~~ServerHelloDone\\
  \tt ~~~~~~~~~~~Certificate*~--2-->\\
  \tt ~~~~~ClientKeyExchange~~--3-->\\
  \tt ~~~~~CertificateVerify*~--4-->\\
  \tt ~~~~[ChangeCipherSpec]~~--5-->\\
  \tt ~~~~~~~~~~~~~~Finished~~--6-->\\
  \tt ~~~~~~~~~~~~~~~~~~~~~~~~~~~~~<--6--~~[ChangeCipherSpec]\\
  \tt ~~~~~~~~~~~~~~~~~~~~~~~~~~~~~<--7--~~Finished\\
  \tt ~~~~~~Application~Data~~<----------->~~Application~Data
  \end{scriptsize}
\end{frame}

\begin{frame}{Mögliche Lösungen}
  Vorschlänge im Entwurf von k. Hartge und O. Bergmann:\\
  http://tools.ietf.org/html/draft-hartke-core-codtls-02\\
  \begin{itemize}
    \item Handshake über CoAP
    \item Stateless Header Compression
  \end{itemize}
\end{frame}

\begin{frame}{Mögliche Lösungen}{Handshake über CoAP - Teil 1}
  \tt ~~~~~~POST /dtls ---->\\
  \tt ~ClientHello\\
  \tt ~~~~~~~~~~~~~~~~~<---- 1.02 Verify\\
  \tt ~~~~~~~~~~~~~~~~~~~~~~~~~~HelloVerifyRequest\\
  \tt ~~~~~~POST /dtls ---->\\
  \tt ~ClientHello\\
  \tt (mit cookie)\\
  \tt ~~~~~~~~~~~~~~~~~<---- 2.01 Created\\
  \tt ~~~~~~~~~~~~~~~~~~~~~~~~~~ServerHello (S=X)\\
  \tt ~~~~~~~~~~~~~~~~~~~~~~~~~*Certificate\\
  \tt ~~~~~~~~~~~~~~~~~~~~~~~~~*ServerKeyExchange\\
  \tt ~~~~~~~~~~~~~~~~~~~~~~~~~*CertificateRequest\\
  \tt ~~~~~~~~~~~~~~~~~~~~~~~~~~ServerHelloDone
\end{frame}
\begin{frame}{Mögliche Lösungen}{Handshake über CoAP - Teil 2}
  \tt ~~~~~~~POST /dtls?s=X ---->\\
  \tt ~~~~~~Certificate*\\
  \tt ClientKeyExchange\\
  \tt CertificateVerify*\\
  \tt ~ChangeCipherSpec\\
  \tt ~~~~~~~~~Finished\\
  ~\\
  \tt ~~~~~~~~~~~~~~~~~~~~~~<---- 2.04 Changed\\
  \tt ~~~~~~~~~~~~~~~~~~~~~~~~~~~~~~~ChangeCipherSpec\\
  \tt ~~~~~~~~~~~~~~~~~~~~~~~~~~~~~~~Finished\\
  ~\\
  \tt ~~~~~Application Data <---> Application Data
\end{frame}

\begin{frame}{Mögliche Lösungen}{Stateless Header Compression}
  \begin{columns}
    \column{.5\textwidth}
      struct \{\\
      \qquad uint8 ~~:1;\\
      \qquad RecordType type:2;\\
      \qquad Version version:2;\\
      \qquad Epoch epoch:3;\\
      \qquad uint8 ~~:3;\\
      \qquad SequenceNumber snr:3;\\
      \qquad RecordLength length:2;\\
      \qquad uint8 ~~payload[0];\\
      \} DTLSRecord\_t;\\
      ~\\
      = 2 - 15 Byte
    \column{.5\textwidth}
      struct \{\\
      \qquad ContentType type:6;\\
      \qquad ContentLength len:2;\\
      \qquad uint8 ~~payload[0];\\
      \} Content\_t;\\
      ~\\
      ~\\
      ~\\
      ~\\
      ~\\
      ~\\
      = 1 - 3 Byte
  \end{columns}
\end{frame}

\begin{frame}{Geeignetes Ciphersuit}
  \begin{itemize}
    \item TLS\_PSK\_ECDH\_WITH\_AES\_128\_CCM\_8
  \end{itemize}
  Eigenschaften:
  \begin{itemize}
    \item Durch CCM wird keine Hash-Funktion \newline für den MAC benötigt
    \item Durch den Einsatz von CBC-MAC mit PSK kann \newline auf HMAC mit SHA-256 verzichtet werden
  \end{itemize}
  Auswirkungen auf die Programmgröße:
  \begin{itemize}
    \item Einsparung durch AES in Hardware: \textasciitilde 1.5KiB
    \item Einsparung durch CBC-MAC mit PSK: \textasciitilde 1.5KiB
  \end{itemize}
\end{frame}

\end{document}