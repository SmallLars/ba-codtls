\chapter{Definition des Ciphersuits}

Vorhanden rfc6655 \cite{rfc6655}:\\
CipherSuite TLS\_PSK\_WITH\_AES\_128\_CCM\_8     = {0xC0,0xA8}\\
CipherSuite TLS\_PSK\_WITH\_AES\_256\_CCM\_8     = {0xC0,0xA9)\\
CipherSuite TLS\_PSK\_DHE\_WITH\_AES\_128\_CCM\_8 = {0xC0,0xAA}\\
CipherSuite TLS\_PSK\_DHE\_WITH\_AES\_256\_CCM\_8 = {0xC0,0xAB}


Neu:\\
CODTLS\_PSK\_ECDH\_WITH\_AES\_128\_CCM\_8 = {0xFF,0x01}\\
ccm gemäß rfc3610 \cite{rfc3610}\\
mit Nonce_Len = 12, Len_len = 3, Mac_len = 8

Nonce:\\
Nonce mit Länge 12 Byte = IV + Epoch + message\_seq\\
Zwar auf beiden Seiten gleich aber irrelevant da unterschiedliche keys.\\
außerdem unterschiedliche IV's

Generell:\\
CBC-MAC anstatt HMAC mit SHA256
\begin{lstlisting}[language=c]
PRF(secret, label, seed) = P_hash(secret + label + seed)

P_hash(seed) = CBC-MAC(A(1) + seed) +
               CBC-MAC(A(2) + seed) +
               CBC-MAC(A(3) + seed) + ...
A(0) = seed
A(i) = CBC-MAC(A(i-1))

master_secret  = PRF(pre_master_secret, "master secret", ClientHello.random + ServerHello.random)[0..47]; (105 | 153) = (36 | 84) + 13 + 28 + 28
finished       = PRF(master_secret, finished_label, Hash(handshake_messages))[0..verify_data_length-1];            79 = 48 + 15 + 16
                 finished_label = "client finished" | "server finished"
key_block      = PRF(master_secret, "key expansion", server_random + client_random);                              117 = 48 + 13 + 28 + 28

key_block      = client_write_MAC_key[0]
                 server_write_MAC_key[0]
                 client_write_key[16]
                 server_write_key[16]
                 client_write_IV[4]
                 server_write_IV[4]
\end{lstlisting}

Finished:\\
Hash mit CBC-MAC der während dem Handshake ausgetauschten Daten wie sie über die Leitung gingen (nicht entpackt, headerkompression bleibt)\\
wird mit den Sicherheitsparametern der neuen Epoche verschlüsselt.\\
20 Byte lang   ->   Verschlüsselte 12 Byte finished (12) + (CBC\_MAC) (8)