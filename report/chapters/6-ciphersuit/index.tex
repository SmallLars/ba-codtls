\chapter{Definition des Ciphersuits}

Vorhanden rfc6655 \cite{rfc6655}:\\
CipherSuite TLS\_PSK\_WITH\_AES\_128\_CCM\_8     = {0xC0,0xA8}\\
CipherSuite TLS\_PSK\_WITH\_AES\_256\_CCM\_8     = {0xC0,0xA9)\\
CipherSuite TLS\_PSK\_DHE\_WITH\_AES\_128\_CCM\_8 = {0xC0,0xAA}\\
CipherSuite TLS\_PSK\_DHE\_WITH\_AES\_256\_CCM\_8 = {0xC0,0xAB}


Neu:\\
CODTLS\_PSK\_WITH\_AES\_128\_CCM = {0xFF,0x01}\\
CODTLS\_PSK\_ECDH\_WITH\_AES\_128\_CCM = {0xFF,0x05}\\
CODTLS\_PSK\_WITH\_AES\_128\_CCM\_8 = {0xFF,0x07}\\
CODTLS\_PSK\_ECDH\_WITH\_AES\_128\_CCM\_8 = {0xFF,0x0C}\\
ccm gemäß rfc3610 \cite{rfc3610}

Nonce:\\
Nonce mit Länge 8 Byte = Epoch + message\_seq\\
Zwar auf beiden Seiten gleich aber irrelevant da unterschiedliche keys.

Generell:\\
CBC-MAC anstatt HMAC mit SHA256\\
psr = (psk, secret + A(x) + seed)

Finished:\\
Hash mit CBC-MAC der während dem Handshake ausgetauschten Daten wie sie über die Leitung gingen (nicht entpackt, headerkompression bleibt)\\
wird mit den Sicherheitsparametern der neuen Epoche verschlüsselt.\\
28 Byte lang   ->   NONCE (8) + Verschlüsselte 12 Byte vom Hash (12) + (CBC\_MAC) (8)