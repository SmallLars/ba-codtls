\chapter{Fazit}
\label{chp:fazit}

Abschließend lässt sich sagen, dass es gelungen ist, \acr{dtls} so anzupassen, dass es sich für den Einsatz auf dem
Redbee Econotag mit dem \glos{mc1322} Mikrocontroller eignet. Das gesamte Programm ist ausreichend klein, um die
Funktionalitäten der Endgeräte selbst ergänzen zu können. Trotz der Anpassungen sind die Elemente von \acr{dtls}
nach wie vor erhalten.

Möglich wurde dies unter anderem durch die Nutzung des Flash-Speichers zur Ablage der Session-Daten. Insgesamt
sind 148 Byte pro Session notwendig, von denen sich 140 Byte für die Ablage im Flash-Speicher eignen. Durch die
Begrenzung auf maximal zehn Sessions werden 1400 Byte im Flash-Speicher genutzt. Dabei hat ein Anwender, durch
die genutzte \glos{ciphersuite} und den Wechsel des \acr{psk}, die volle Kontrolle über die Nutzung dieser zehn
Sessions. Sind mehr als zehn Sessions erforderlich, erlauben die, auf dem Redbee Econotag verfügbaren, Ressourcen
eine Erweiterung auf mehr als 50 Sessions. Soll auf die Nutzung des Flash-Speichers verzichtet werden, wäre es
auch denkbar, die Anzahl der Sessions weiter zu reduzieren und die Daten vollständig im RAM-Speicher abzulegen.

In der Evaluation hat sich jedoch gezeigt, dass weitere Verbesserungen möglich sind. So ist die blockweise
Übertragung der \glos{handshake}-Daten durch \acr{coap} eine gute Lösung, während in Kombination mit dieser, die Modellierung
der \acr{dtls}-Sessions als \acr{coap}-Ressource keine Option ist. Auch werden durch die blockweise Übertragung neue
\acr{dos}-Angriffe ermöglicht, die der in \acr{dtls} definierte Cookie nicht verhindern kann. Ein Idee, um diesem
entgegen zu wirken, konnte in dieser Arbeit erarbeitet werden, wurde jedoch praktisch noch nicht umgesetzt.

Im \acr{ietf}-Entwurf "`Practical Issues with Datagram Transport Layer Security in Constrained Environments"' \cite{draftpractical}
von K. Hartke sind einige weitere Lösungsansätze aufgeführt. Dieser Entwurf ist erst kurz vor Fertigstellung dieser
Arbeit erschienen, und wurde deswegen nicht mehr berücksichtigt. Trotzdem soll hier auf einen Vorschlag dieses Entwurfs
eingegangen werden, der die Verwendung von Bestätigungsnachrichten (Acknowledgements) beschreibt. Dieses Verfahren
kommt dem Verhalten von \acr{coap} sehr nahe. Von Vorteil ist dieser Vorschlag, da sich \acr{dtls} auch in eingeschränkten
Umgebungen ohne \acr{coap} realisieren lässt. Dort wo \acr{coap} Verwendung findet, würde dies den Programmcode jedoch
unnötig vergrößern, da die Mechanismen in \acr{coap} bereits zur Verfügung stehen. Da die Acknowledgement-Nachrichten von
\acr{coap}, während des \acr{dtls}-\glospl{handshake}, in der Evaluation dieser Arbeit auch separat betrachtet werden,
liefert diese Arbeit für zukünftige \acr{dtls}-Implementierungen, ohne \acr{coap} und mit Acknowledgements, Vergleichsmaterial.

Zukünftig sind weitere Optimierungen von Contiki denkbar. Die in dieser Arbeit vorgenommenen Anpassungen der Stack- und Heap-Größe,
sind nur einige der möglichen Optionen. Neben dem System-Stack, beinhaltet Contiki fünf weitere Stacks, deren Größe noch zu überprüfen ist.
Auch ist mit den richtigen Kompileroptionen eine weitere Reduzierung der Programmgröße möglich. So hat sich gezeigt, dass durch
entfernen der Option -mcallee-super-interworking 4,32 KiB eingespart werden kann. Diese ermöglicht generell den Aufruf, der im 16-Bit-Modus
kompilierten Funktionen, durch 32-Bit-Code. Da dies im Allgemeinen garnicht notwendig ist, scheint diese überflüssig.
Auch wenn diese Anpassungen noch gründlich überprüft werden müssen, zeigt dies, das mehr Raum für weiteren Programmcode geschaffen werden kann.

Wünschenswert ist die Verwendung der, in dieser Arbeit entstandenen, \acr{dtls}-Variante im Masterporjekt \glos{gobi}.
Dort kann sich diese im praktischen Einsatz bewähren, während die genannten Verbesserungen im Projektverlauf realisiert werden können.