\section{Motivation}
Während es im Internet schon seit Mitte der 90er Jahre das  "`Transport Layer Security (TLS) Protocol"' \cite{rfc5246} (früher SSL) gibt,
um den Datenverkehr über das "`Transmission Control Protocol (TCP)"' abzusichern, kam es erst 2004 zu einem ersten Versuch den Datenverkehr
über das "`User Datagram Protocol (UDP)"' zu sichern. Dieser Versuch führte 2006 schließlich zu der Standardisierung des "`Datagram Transport
Layer Security (DTLS) Protocol"' \cite{rfc6347}. Dieses ist TLS sehr ähnlich, wurde jedoch um Mechanismen ergänzt, die in UDP im Gegensatz
zu TCP fehlen. Dazu gehört insbesondere die Zuverlässigkeit der Datenübertragung die beim Verbindungsaufbau und somit der Aushandlung der
Sicherheitsmechanismen notwendig ist. TLS und DTLS haben sich im Internet bewährt, sind jedoch zu einer Zeit entstanden, als Sensornetze und das
"`Web of Things"' noch nicht vertreten waren. Die dort verwendeten Geräte haben nur sehr wenig Ressourcen zur Verfügung.
Dies betrifft neben wenig Rechenleistung und Speicher auch den Energievorrat. Etabliert hat sich dort das "`Constrained
Application Protocol (CoAP)"' \cite{draftcoap} über das UDP-Protokoll aufgrund seines schlanken Designs. Passend zu UDP ist
das umfangreiche DTLS-Protokoll, das sich jedoch nicht so einfach auf "`kleinen"' Endgeräten realisieren lässt.
Genau hier soll diese Arbeit ansetzen und DTLS als Lösung bestätigen oder Alternativen aufzeigen und realisieren.

%  Ich wuerde DTLS nicht rundweg als »ersten Versuch den Datenverkehr u ̈ber das ”User Datagram Protocol (UDP)“ zu sichern.« bezeichnen.

%  Hm, worauf stuetzt sich die Aussage mit dem "ersten Versuch"? Ich
%  wuerde fast Geld darauf wetten, dass es frueher bereits Versuche 
%  dazu gab. SRTP ist z.B. auch aelter als DTLS; und das SNMP-Volk hat
%  sich auch viele kranke Dinge ausgedacht, bevor DTLS in ihr Universum 
%  aufgenommen wurde.

%  "Sind jedoch zu einer Zeit enstanden, als..." -- das trifft auf das
%  Web of Things wohl zu, auf Sensornetze sicher nicht. Die sind schon
%  ziemlich lange da (und als Begriff ein bisschen olivgruen).