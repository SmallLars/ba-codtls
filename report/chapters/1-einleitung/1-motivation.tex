\section{Motivation}
Während es im Internet schon seit Mitte der 90er Jahre das  "`\acr{tls} Protocol"' \cite{rfc5246} (früher SSL) gibt, um den Datenverkehr
über das "`\acr{tcp}"' abzusichern fehlte lange Zeit ein Standard um den Datenverkehr über das "`\acr{udp}"' zu sichern, während die
Beliebtheit des Protokolls, unter anderem im Bereich der Onlinespiele, zunahm \cite[Kapitel 1]{rfc6347}. Um diese Lücke zu schließen begann
die \acr{ietf} 2004 damit ein Protokoll nach dem Vorbild von \acr{tls} zu entwickeln, was 2006 schließlich zu der Standardisierung des
"`\acr{dtls} Protocol"' \cite{rfc6347} führte. Dieses ist fast identisch mit \acr{tls}, wurde jedoch um Mechanismen ergänzt, die in \acr{udp}
im Gegensatz zu \acr{tcp} fehlen. Dazu gehört insbesondere die Zuverlässigkeit der Datenübertragung die beim Verbindungsaufbau und somit der
Aushandlung der Sicherheitsmechanismen notwendig ist. \acr{tls} und \acr{dtls} haben sich im Internet bewährt, sind jedoch zu einer Zeit
entstanden, als das "`\acr{wot}"' noch nicht vertreten war. Die dort verwendeten Geräte haben nur sehr wenig Ressourcen zur Verfügung.
Dies betrifft neben wenig Rechenleistung und Speicher auch den Energievorrat. Etabliert hat sich dort das "`\acr{coap}"' \cite{draftcoap13}
über \acr{udp} aufgrund seines schlanken Designs. Passend zu \acr{udp} ist das umfangreiche \acr{dtls}-Protokoll, das sich jedoch nicht so
einfach auf "`kleinen"' Endgeräten realisieren lässt. Genau hier soll diese Arbeit ansetzen und \acr{dtls} entsprechend Anpassen damit es
auch auf Geräten mit wenig Ressourcen funktionieren kann.