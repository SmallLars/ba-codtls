\section{Verwandte Arbeiten}

Da die Anpassung von \acr{dtls}, für den Einsatz in eingeschränkten Umgebungen, ein aktuelles Thema ist, und bereits einige Vorschläge
existieren, die in dieser Arbeit aufgegriffen werden sollen, werden in den folgen beiden Abschnitten zwei Entwürfe der \acr{ietf} betrachtet.

\subsection{Datagram Transport Layer Security in Constrained Environments}
Im \acr{ietf}-Entwurf "`Datagram Transport Layer Security in Constrained Environments"' \cite{draftcodtls} haben K. Hartke und O. Bergmann
bereits einige Probleme aufgezeigt, die der Einsatz von \acr{dtls} in eingeschränkten Umgebungen mit sich bringt und mögliche Lösungen vorgeschlagen.

Als eines der Hauptprobleme nennen die Autoren die geringe Paketgröße in Netzen, die den Funkstandard IEEE 802.15.4 \cite{ieee802154} verwenden,
da hier die Nutzdaten auf eine Länge von 127 Byte pro Paket beschränkt sind. Insbesondere beim Aufbau der sicheren Verbindung (\glos{handshake}) müssen
viele Daten ausgetauscht werden, was bei der Verwendung von \acr{dtls} die Paketgröße überschreiten würde. Durch \acr{ipv6} \cite{rfc2460}, das in
eingeschränkten Umgebungen mit Hilfe von \acr{6lowpan} \cite{rfc4944} realisiert wird, würde sich das Problem durch eine Nutzung der IP-\glos{fragmentierung}
lösen lassen. Das führt aber bei Verlust einzelner Pakete, zu einem neuen Versand aller IP-\glospl{fragment}, und erzeugt somit umfangreichen Datenverkehr,
der einen hohen Energieverbauch mit sich bringt. Ein weiterer Ansatz besteht darin, die Menge der Daten sowohl beim Verbindungsaufbau als auch bei der
Datenübertragung durch Komprimierung der Headerdaten zu verringern. Dafür schlagen die Autoren eine Stateless-Header-Compression vor. Bei Nutzung von
\acr{coap} wäre es auch möglich, den Verbindungsaufbau über \acr{coap} zu realisieren. Dadurch ist die Zuverlässigkeit des Transports gegeben und große Pakete
könnten mit einer blockweisen Übertragung effizient übertragen werden, so dass bei Paketverlusten nur die verlorenen Pakete erneut übertragen werden müssten.

Beachtet werden müssen auch die Zeiten, nach denen ein Paket als verloren angesehen und erneut gesendet wird. Gerade beim Verbindungsaufbau
kann es durch aufwendige Berechnungen, wie sie beispielsweise im Elliptic Curve Diffie-Hellman benötigt werden, zu einer erhöhten
Antwortzeit kommen, was nicht zu einem erneuten Paketversand führen sollte.

Beim Verbindungsaufbau werden viele Daten ausgetauscht, was gerade in eingeschränkten Umgebungen einige Zeit dauern kann. Um die Zeit
möglichst kurz zu halten, ist es wichtig, die Anzahl der Kommunikationsvorgänge gering zu halten. Auch kann der Verbindungsaufbau schon
durchgeführt werden, bevor Anwendungsdaten zum Übertragen vorhanden sind. Liegen dann Anwendungsdaten vor, können diese ohne Verzögerung
übertragen werden.

Um Speicher zu sparen, müssen auch die Anzahl der sicheren Verbindungen begrenzt werden und Verbindungen nach einiger Zeit
automatisch geschlossen werden, um neue Verbindungen zu ermöglichen.

\subsection{A Hitchhiker's Guide to the (Datagram) Transport Layer Security Protocol}
Im \acr{ietf}-Entwurf "`A Hitchhiker's Guide to the (Datagram) Transport Layer Security Protocol"' \cite{draftmintls} haben H. Tschofenig, S.S. Kumar
und S. Keoh zunächst die Unterschiede von \acr{tls} 1.0, 1.1 und 1.2 erläutert und klargestellt, dass die Details bei einem \glos{handshake} von der Wahl des
\glospl{ciphersuite} abhängen. Anhand einiger Beispiele erläutern sie, dass es wichtig ist, sich der Position eines Gerätes in einer Verbindung bewusst zu sein.
So kann ein Sensor mit beschränkten Ressourcen sowohl als Server als auch als Client realisiert werden, wobei es auch auf die Anzahl der möglichen
Verbindungen ankommt. Ein Sensor, der als Client agiert, wird mit großer Wahrscheinlichkeit immer nur einen Server kontaktieren, um dort neue Sensordaten
zu hinterlegen, während ein als Server realisierter Sensor durchaus auch Anfragen von mehreren Clients erhalten kann. Je klarer die Position und die
Umgebung des Sensors sind, desto weniger flexibel muss dieser sein, woraus eine spezialisierte Implementierung resultiert, die den Aufwand und die Codegröße reduziert.

Im weiteren Verlauf gehen sie auf wichtige Design-Entscheidungen ein und erläutern deren Bedeutung und mögliche Auswirkungen.

Kernstück der Arbeit ist die Auswertung des Speicherverbrauchs, sowohl im \acr{rom} als auch im \acr{ram}, und die Menge der übertragenen Daten bei einem \glos{handshake}.
Anhand eines modifizierten Prototyps zeigen sie dort auf, welche grundlegenden Teile von \acr{dtls}, ohne Berücksichtigung der \glos{ciphersuite} spezifischen Funktionen,
wieviel Speicher verbrauchen und werten die Menge der übertragenen Daten in einem kompletten \glos{handshake} für die unterschiedlichen Protokollschichten aus.
Des weiteren haben sie die Codegrößen von beispielsweise Hash-Funktionen und anderen für \acr{tls} notwendigen Berechnungen ausgewertet, wie sie in unterschiedlichen
\glospl{ciphersuite} verwendet werden.

Abschließend stellen sie fest, dass sich \acr{tls}/\acr{dtls} durchaus auf eingeschränkte Umgebungen zuschneiden lässt, wobei mehr Flexibilität aber zu
größerem Programmcode führt. \TODO{hier vielleicht ein paar Kennzahlen zitieren?}

% \subsection{Generic Header Compression}
% todo