\section{Verwandte Arbeiten}

\subsection{Datagram Transport Layer Security in Constrained Environments}
Im Internet-Entwurf "`Datagram Transport Layer Security in Constrained Environments"' \cite{draftcodtls} haben K. Hartke und O. Bergmann
bereits einige Probleme aufgezeigt, die der Einsatz von \acr{dtls} in eingeschränkten Umgebungen mit sich bringt und mögliche Lösungen vorgeschlagen.

Eines der Hauptprobleme ist hier die geringe Paketgröße in Netzen die \acr{6lowpan} verwenden, da hier die Nutzdaten auf eine Länge von 127 Byte beschränkt sind.
Insbesondere beim Aufbau der sicheren Verbindung (\glos{handshake}) müssen viele Daten ausgetauscht werden, was bei der Verwendung von \acr{dtls} die Paketgröße
überschreiten würde. Lösen würde das Problem bspw. eine Nutzung der IP-\glos{fragmentierung}, was aber bei Verlust einzelner Pakete zu einem neuen Versand aller
IP-\glospl{fragment} führen würde und somit umfangreichen DAtenverkehr erzeugt, der einen hohen Energieverbauch mit sich bringt. Ein weiterer Ansatz besteht darin,
die Menge der Daten sowohl beim Verbindungsaufbau als auch bei der Datenübertragung durch Komprimierung der Headerdaten zu verringern wofür es unterschiedliche
Vorschläge gibt. Bei Nutzung von \acr{coap} wäre es auch möglich, den Verbindungsaufbau über \acr{coap} zu realisieren. Dadurch ist die Transportsicherung gegeben
und große Pakete könnten mit einer blockweisen Übertragung effizient übertragen werden, so dass bei Paketverlusten nur die verlorenen Pakete erneut übertragen werden
müssten.

Beachtet werden müssen auch die Zeiten, nach denen ein Paket als verloren angesehen und erneut gesendet wird. Gerade beim Verbindungsaufbau
kann es durch aufwendige Berechnungen, wie sie bspw. im Elliptic Curve Diffie-Hellman Schlüsselaustausch benötigt werden, zu einer erhöhten
Antwortzeit kommen, was nicht zu einem erneuten Paketversand führen sollte.

Beim Verbindungsaufbau werden viele Daten ausgetauscht, was gerade in eingeschränkten Umgebungen einige Zeit dauern kann. Um die Zeit
möglichst kurz zu halten, ist es wichtig die Anzahl der Kommunikationsvorgänge gering zu halten oder den Verbindungsaufbau schon durchzuführen
bevor Anwendungsdaten ausgetauscht werden, damit diese dann sofort übertragen werden können.

Um Speicher zu sparen müssen auch die Anzahl der sicheren Verbindungen begrenzt werden und/oder Verbindungen nach einiger Zeit
automatisch geschlossen werden um neue Verbindungen zu ermöglichen.

\subsection{A Hitchhiker's Guide to the (Datagram) Transport Layer Security Protocol}
Im Internet-Entwurf "`A Hitchhiker's Guide to the (Datagram) Transport Layer Security Protocol"' \cite{draftmintls} haben H. Tschofenig, S.S. Kumar
und S. Keoh zunächst die Unterschiede von \acr{tls} 1.0, 1.1 und 1.2 erläutert und klargestellt, dass die Details beim \glos{handshake} von der Wahl des
\glospl{ciphersuite} abhängen. Anhand einiger Beispiele erläutern sie, dass es wichtig ist, sich der Position eines Gerätes in einer Verbindung bewusst zu sein.
So kann ein Sensor mit beschränkten Ressourcen sowohl als Server als auch als Client realisiert werden wobei es auch auf die Anzahl der möglichen
Verbindungen ankommt. Ein Sensor der als Client agiert wird mit großer Wahrscheinlichkeit immer nur einen Server kontaktieren um dort neue Sensordaten
zu hinterlegen, während ein als Server realisierter Sensor durchaus auch Anfragen von mehreren Clienten erhalten kann. Je klarer die Position und die
Umgebung des Sensors ist, desto weniger flexibel kann dieser Implementiert werden was den Aufwand und die Codegröße reduziert.

Im weiteren Verlauf gehen sie auf wichtige Design-Entscheidungen ein und Erläutern deren Bedeutung und mögliche Auswirkungen.

Kernstück des Entwurfs ist die Auswertung des Speicherverbrauchs, sowohl im \acr{rom} als auch im \acr{ram}, und die Menge der Übertragenen Daten bei einem \glos{handshake}.
Anhand eines modifizierten Prototypens zeigen sie dort auf, welche grundlegenden Teile von \acr{dtls}, ohne Berücksichtigung der \glos{ciphersuite} spezifischen Funktionen,
wieviel Speicher verbrauchen und werten die Menge der übertragenen Daten in einem kompletten \glos{handshake} für die unterschiedlichen Protokollschichten aus.
Des weiteren haben sie die Codegrößen von bspw. Hash-Funktionen und anderen für \acr{tls} notwendigen Berechnungen ausgewertet, wie sie in unterschiedlichen
\glospl{ciphersuite} verwendet werden.

Abschließend stellen sie fest, dass sie \acr{tls}/\acr{dtls} durchaus auf eingeschränkte Umgebungen zuscheiden lässt, wobei mehr Flexibilität aber zu einem
größeren Programmcode führt.

% \subsection{Generic Header Compression}
% todo