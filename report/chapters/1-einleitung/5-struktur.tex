\section{Struktur}

Im Anschluss an die Einleitung folgt in \textbf{Kapitel \ref{chp:dtls}} eine Zusammenfassung
über die Funktionsweise von \acr{dtls}, damit für die Beschreibung der Anpassungen,
in \textbf{Kapitel \ref{chp:anpassungen}}, eine Grundlage vorhanden ist.

Die für diese Arbeit verwendete \glos{ciphersuite} wird in \textbf{Kapitel \ref{chp:ciphercuite}} beschrieben.

In \textbf{Kapitel \ref{chp:praktisch}} werden die Details der praktischen Umsetzung beschrieben, wobei
dieses in 3 Abschnitte unterteilt ist. Während in \textbf{Abschnitt \ref{sec:server}} der Server, der für den
\glos{mc1322} konzipiert ist, erläutert wird, werden in \textbf{Abschnitt \ref{sec:client}} und
\textbf{Abschnitt \ref{sec:entwicklungsumgebung}} einige Dinge beschrieben, die den Client und
die Entwicklungsumgebung auf einem gängigen Computer betreffen.

\textbf{Kapitel \ref{chp:evaluation}} behinhaltet die Evaluation, in der \acr{dtls} mit den, in
dieser Arbeit vorgeschlagenen, Anpassungen verglichen wird, während in \textbf{Kapitel \ref{chp:fazit}}
die persönliche Meinung des Autors enthalten ist.