\section{Struktur}

... \TODO{Struktur beschreiben} ...

% Im Folgenden soll ein Überblick über den Aufbau dieses Dokumentes
% gegeben werden und die behandelten Themen kurz vorgestellt werden.
% 
% Im Anschluss an die Einleitung folgen in den \textbf{Kapiteln \ref{chp:hardware} und \ref{chp:wsn}} die
% Beschreibungen der benutzten Hardware sowie der offenen Funk- und Protokollstandards,
% welche im \glos{gobi}-System eingesetzt wurden.
% 
% In \textbf{Kapitel \ref{chp:overview}} wird das \glos{gobi}-System als ganzes erläutert,
% wobei die Zusammenhänge zwischen einzelnen Kompontenten beschrieben werden.
% 
% Die \textbf{Kapitel \ref{chp:sicherheit}, \ref{chp:geraetebeschreibung}
% und \ref{chp:zentrale}} beschreiben die einzelnen Komponenten des \glos{gobi}-Systems
% im Detail.
% 
% Die Möglichkeiten des Benutzers, mit dem \glos{gobi}-System zu interagieren, werden
% in \textbf{Kapitel \ref{chp:user-interfaces}} beschrieben.
% 
% Das zur Veranschaulichung unseres System erstellte Modell wird in \textbf{Kapitel \ref{chp:modell}} erläutert.
% 
% In \textbf{Kapitel \ref{chp:ergebnisse}} fassen wir schließlich die Ergebnisse des 1. Projektjahres zusammen.
% 
% Abschließend folgen das Glossar und die Akronyme sowie das Literatur- und Abbildungsverzeichnis.
% 
% Die Projektorganisation ist in \textbf{Anhang \ref{chp:projektorga}} beschrieben während die Ausarbeitungen der
% Referate, die zum Großteil am Projektwochenende gehalten wurden, in \textbf{Anhang \ref{chp:ausarbeitungen}} aufgeführt sind.