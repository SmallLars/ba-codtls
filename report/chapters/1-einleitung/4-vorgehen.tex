\section{Vorgehensweise}

Im Vordergrund soll die Implementierung eines Sicherheitsprotokolls stehen, das sich an den Prinzipien von \acr{dtls} orientiert
und Vorschläge aus dem \acr{ietf}-Draft "`Datagram Transport Layer Security in Constrained Environments"' \cite{draftcodtls} aufgreift.
Ein besonderes Interesse liegt darauf, den \glos{handshake} über \acr{coap} \cite{draftcoap13} zu realisieren und somit einige in \acr{dtls} eingefügte
Mechanismen überflüssig zu machen. Die Implementierung wird im Anschluss durch einen Vergleich mit \acr{dtls} evaluiert, wobei folgende Punkte eine
Rolle spielen: Volumen des generierten Datenaufkommens, Umfang des Quellcodes und die vom Protokoll benötigte Speichermenge.

Die Implementierung besteht aus dem Client auf einem gängigen PC, bei dem es keine speziellen Einschränkungen an Energie,
Speicher oder Effizienz gibt, und aus dem Server, der für einen Redbee Econotag \cite{econotag} mit dem \glos{mc1322} \cite{mc1322}
Mikrocontroller optimiert werden soll. Da der genannte Mikrocontroller die Verschlüsselung mit dem \acr{aes} im \acr{ctr}- und \acr{cbc}-Mode
in Hardware unterstützt und die Rechenleistung sowie der Speicher beschränkt ist, soll nur eine \glos{ciphersuite} realisiert werden.
Für diese dient "`TLS\_PSK\_DHE\_WITH\_AES\_128\_CCM\_8"' aus RFC 6655 \cite{rfc6655} als Grundlage, wobei dort, aufgrund von Erfahrungen
aus dem Bachelorprojekt, Anpassungen notwendig sind. Durch diese wird für den Verbindungsaufbau ein Schlüsselaustausch vorgegeben, wobei zusätzlich
ein \acr{psk} verwendet wird, damit sich die Kommunikationspartner gegenseitig authentifzieren können. Ein Verbindungsaufbau ist somit nur möglich,
wenn beide Kommunikationspartner vorher einen gemeinsamen \acr{psk} vereinbart haben. Die Verschlüsselung der Anwendungsdaten erfolgt dann im Modus
"`\acr{aead}"' \cite{rfc5116} wobei sich hier "`\acr{ccm}"' \cite{rfc3610} aufgrund der Hardwarevorraussetzungen am besten eignet. Dieser besteht
aus einer Verschlüsselung der Daten mit \acr{aes} im \acr{ctr}-Modus, während der dazugehörende \acr{mac} durch \acr{aes} im \acr{cbc}-Modus
berechnet wird. Die Anzahl der möglichen sicheren Verbindungen soll beschränkt werden, um den Speicherverbrauch gering zu halten.

Bei der Evaluation soll die Datenmenge der Header-Daten, und der für einen \glos{handshake} benötigten Daten, mit einer reinen \acr{dtls}-Implementierung
verglichen werden, wobei auch eine \acr{dtls}-Variante herangezogen wird, die eine Stateless-Header-Compression benutzt. Bewertet wird auch die Programmgröße,
wobei einige Komponenten denen von \acr{dtls} gegenübergestellt werden. Ebenso soll die Dauer des \glos{handshake} bewertet werden.