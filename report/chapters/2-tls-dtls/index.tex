\chapter{DTLS}
\label{chp:dtls}

Um eine Grundlage für die Anpassungen im folgenden Kapitel zu schaffen, und in der Evaluation einen Vergleich zu ermöglichen,
wird hier zunächst \acr{dtls}, wie es in RFC 6347 \cite{rfc6347} standardisiert ist, erläutert. Da \acr{dtls} basierend auf
\acr{tls}, gemäß RFC 5246 \cite{rfc5246}, definiert ist, wird auch auf den Unterschied zwischen beiden eingegangen.

Das Sicherheitsprotokoll \acr{tls} wird im Allgemeinen mit dem stromorientierten \acr{tcp} verwendet.
Wurde eine Verbindung mit \acr{tcp} hergestellt, können Daten beliebiger Größe in jede Richtung
übertragen werden. \acr{tcp} sorgt dafür, dass der eingegebene Bytestrom vollständig und in der
richtigen Reihenfolge auf der Gegenseite wieder ausgegeben wird. Um die \acr{tls}-bezogenen Daten nun
zu kennzeichnen und voneinander abzugrenzen, existiert das Record-Layer-Protokoll, dessen Header in Abbildung \ref{fig:recordlayer}
dargestellt ist. Dort ist, neben der Art des Inhalts und der Protokollversion, auch die Länge enthalten, so dass
aufeinanderfolgende Pakete im Datenstrom voneinander abgegrenzt werden können. Als Inhalt kommen vier Sub-Protokolle
in Frage. Während das Application-Data-Protokoll für den Transport der Anwendungsdaten genutzt wird, kommt
das \glos{handshake}-Protokoll für die Aushandlung der Sicherheitsparameter zum Einsatz. Über das Change-Cipher-Spec-Protokoll
werden die zuletzt ausgehandelten Sicherheitsparameter aktiviert. Sollte es beim \glos{handshake} oder der Übertragung
von Anwendungsdaten zu Fehlern kommen, werden diese mit Hilfe des Alert-Protokolls übertragen.

Da bei \acr{dtls} im Allgemeinen das paketorientierte \acr{udp} verwendet wird, bei dem die Länge eines
Paketinhalts bekannt ist, wirkt die Längenangabe zunächst überflüssig. Jedoch ist es insbesondere bei einem
\glos{handshake} sinnvoll, mehrere \acr{dtls}-Pakete innerhalb eines \acr{udp}-Pakets zusammenzufassen,
so dass auch hier wieder eine Längenangabe benötigt wird, um die Pakete voneinander abzugrenzen.
Zusätzlich sind bei \acr{dtls}, gegenüber \acr{tls}, nun die Datenfelder für die Epoche und die Sequenznummer hinzugekommen.
Während diese beiden Werte bei \acr{tls}, durch die gewährleistete Reihenfolge der Daten durch \acr{tcp},
implizit bekannt sind, müssen diese bei \acr{dtls} explizit angegeben werden, da \acr{udp} weder die
Reihenfolge, noch den fehlerfreien Transport der Daten garantiert. Die Epoche wird bei einem erfolgreichen \glos{handshake}
erhöht, und ordnet so die dazugehörenden Daten den im \glos{handshake} ausgehandelten Sicherheitsparametern zu,
während die Sequenznummer in jeder Epoche bei $ 0 $ beginnt und bei jedem Paketversand erhöht wird.

\begin{figure}[ht]
  \centering
  \begin{lstlisting}[language=c]
  struct {
    ContentType type;
    ProtocolVersion version;
    uint16 epoch;                           // Nur bei DTLS
    uint48 sequence_number;                 // Nur bei DTLS
    uint16 length;
  } DTLS_Record;
  \end{lstlisting}
  \caption{Header des Record-Layer-Protokolls von DTLS}
  \label{fig:recordlayer}
\end{figure}

\section{Handshake}
\label{sec:ori-handshake}

Damit es überhaupt zu einer sicheren Verbindung kommen kann, müssen zunächst einige Sicherheitsparameter mit Hilfe des \glos{handshake}-Protokolls ausgehandelt werden.
Der Header einer \glos{handshake}-Nachricht setzt sich gemäß Abbildung \ref{fig:handshakelayer} zusammen. Während es bei TLS ausreichend ist, den Typ, die Länge und die Daten selbst
zu senden, wurden bei DTLS weitere Datenfelder ergänzt. message\_seq dient zur Durchnummerierung der \glos{handshake}-Nachrichten, um zu gewährleisten, dass diese vollständig,
und in der richtigen Reihenfolge, bearbeitet werden. Da auf eine Fragmentierung der \acr{udp}-Pakete auf IP-Ebene vermieden werden soll, und somit die Paketgröße begrenzt ist,
müssen \glos{handshake}-Nachrichten eventuell auf mehrere UDP-Pakete verteilt werden. Um dies zu ermöglichen, wurden fragment\_offset und fragment\_length
ergänzt. So können die Daten in mehrere Teile geteilt werden, wobei die Länge und die Position im Paket hinterlegt werden. length enthält nach wie vor die
Gesamtlänge, so dass eine Fragmentierung jederzeit erkannt werden kann.

\begin{figure}[ht]
  \centering
  \begin{lstlisting}[language=c]
  struct {
    HandshakeType msg_type;
    uint24 length;
    uint16 message_seq;                     // Nur bei DTLS
    uint24 fragment_offset;                 // Nur bei DTLS
    uint24 fragment_length;                 // Nur bei DTLS
  } Handshake;
  \end{lstlisting}
  \caption{Header des \glos{handshake}-Protokolls von DTLS}
  \label{fig:handshakelayer}
\end{figure}

Der für einen \glos{handshake} durchzuführende Nachrichtenaustausch ist in vollständiger Form in Abbildung \ref{fig:handshake} aufgeführt.
Die mit * markierten Pakete werden hier kurz erklärt, spielen aber im weiteren Verlauf keine Rolle, da die Authentifizierung durch
den \acr{psk} realisiert werden soll und auf die Zertifikate verzichtet wird, um Ressourcen zu sparen.

\begin{figure}[ht]
  \centering
  \begin{lstlisting}[language=c]
                      Client           Server
                      ------           ------

        ClientHello  (seq=0) -------->

                             <-------- (seq=0)  HelloVerifyRequest    // Nur bei DTLS

        ClientHello  (seq=1) -------->                                // Nur bei DTLS
       (mit cookie)                                                   // Nur bei DTLS

                             <-------- (seq=1)  ServerHello
                             <-------- (seq=2) *Certificate
                             <-------- (seq=3)  ServerKeyExchange
                             <-------- (seq=4) *CertificateRequest
                             <-------- (seq=5)  ServerHelloDone

        Certificate* (seq=2) -------->
  ClientKeyExchange  (seq=3) -------->
  CertificateVerify* (seq=4) -------->
   ChangeCipherSpec          -------->
           Finished  (seq=5) -------->

                             <--------          ChangeCipherSpec
                             <-------- (seq=6)  Finished  

   Application Data          <------->          Application Data
  \end{lstlisting}
  \caption{Nachrichtenaustausch während eines DTLS-\glospl{handshake}}
  \label{fig:handshake}
\end{figure}

Eingeleitet wird der \glos{handshake} mit einer Nachricht vom Typ ClientHello, in der der Client seine Möglichkeiten bekannt gibt. Dazu gehören u. a. die unterstützten
Protokollversionen, \glospl{ciphersuite} und Kompressionsmethoden. Während der Server bei \acr{tls} nun direkt mit einer ServerHello-Nachricht und weiteren
\glos{handshake}-Nachrichten antworten kann, lässt sich dies bei DTLS so nicht realisieren. Da UDP kein verbindungsorientiertes Protokoll ist, können Pakete mit gefälschtem
Absender versendet werden. Auf diese Art könnte Angriff vom Typ \acr{dos} durchgeführt werden, in dem zahllose Pakete mit unterschiedlichen Absendern an den Server
gesendet werden, welche alle ein ClientHello enthalten. Problematisch ist hierbei der Zustand, der für jedes ClientHello im Server erzeugt wird. Neben dem
Speicherverbrauch kann die Berechnung des ServerKeyExchange eine Menge Rechenleistung benötigen, so dass die Ressourcen des Servers schnell aufgebraucht sind.
Um dies zu vermeiden und den Absender zu validieren, wurde in DTLS ein Cookie ergänzt. Dieser wird aus dem ClientHello generiert und als Antwort an den Client
gesendet. So kann der Server den Cookie bei einem erneuten ClientHello wieder berechnen und mit dem mitgelieferten vergleichen. Dadurch wird bei der ersten Anfrage
ein Zustand vermieden und der Client validiert. Trotz dieses Verfahrens ist es jedoch möglich, den Energievorrat des Servers durch unzählige Anfragen zu reduzieren,
da auch die erste Anfrage (ohne Cookie) bearbeitet werden muss. Lediglich der dafür notwendige Aufwand ist geringer. Auch schützt dieses Verfahren nicht gegen
einen \glos{mitma}.

Im ServerHello gibt der Server bekannt, welche der vom Client genannten Möglichkeiten, wie beispielsweise die unterstützten \glospl{ciphersuite}, ausgewählt wurden.
Folgen können dann ein Zertifikat, Daten für einen Schlüsselaustausch sowie eine Anfrage für das Zertifikat des Clients. Abschließend folgt ein ServerHelloDone, um
dem Client zu signalisieren, dass der \glos{handshake} fortgesetzt werden kann. Dieser sendet nun sein eigenes Zertifikat, falls vom Server angefordert. Es folgen Daten
für den Schlüsselaustausch und Daten, die es dem Server ermöglichen, das Zertifikat des Clients zu überprüfen, falls dieses die Möglichkeit bietet, Daten zu signieren.
Damit sind zunächst alle Daten ausgetauscht, die für die Aushandlung der Sicherheitsmechanismen notwendig sind.

Während die bisher genannten \glos{handshake}-Nachrichten mit den Sicherheitsparametern der aktuell gültigen Epoche versendet werden, folgt nun der Versand eines
ChangeCipherSpec. Dieses Paket gehört formell nicht zum \glos{handshake}-Protokoll, sondern bildet ein eigenes Protokoll, da hier die Epoche verändert wird.
Eine ChangeCipherSpec-Nachricht besteht ausschließlich aus einem 1 Byte langen Header mit dem Wert 1 und enthält keine weiteren Daten.
Nach dem Versand des Pakets werden alle folgenden Pakete mit den Sicherheitsparametern der neuen Epoche versendet, während erst der Empfang solch eines
Pakets dazu führt, dass alle folgenden eingehenden Pakete mit Hilfe der neuen Sicherheitsparameter gelesen werden.

Schließlich wird noch eine Finished-Nachricht ausgetauscht, die wieder zum \glos{handshake}-Protokoll gehört. Diese enthält einen Hashwert von allen bisher ausgetauschten
Nachrichten und wird mit den Sicherheitsparametern der neuen Epoche verschlüsselt. So wird der \glos{handshake} verifiziert, und die neuen Sicherheitsparemeter auf Korrektheit
geprüft.

Ist der \glos{handshake} erfolgreich verlaufen, können anschließend Anwendungsdaten über das Application-Data-Protokoll versendet werden,
wobei die, während des letzten \glospl{handshake} ausgehandelten, Sicherheitsparameter benutzt werden. Sollte es zu einem weiteren \glos{handshake}
kommen, wird dieser ebenfalls mit den Sicherheitsparametern der aktuell gültigen Epoche durchgeführt.

\section{Alert}

Wenn es während des \glospl{handshake} oder der Übertragung von Anwendungsdaten zu Fehlern kommt, werden diese mit Hilfe des Alert-Protokolls übertragen.
Der Header (siehe Abbildung \ref{fig:alertlayer}) enthält neben dem AlertLevel, welches \textit{warning} (1) oder \textit{fatal} (2) sein kann,
die Beschreibung des Fehlers. Während Fehler der Stufe \textit{fatal} zu einem unmittelbaren Verbindungsabbruch führen, sind Fehler der Stufe
\textit{warning} zur Information der Gegenseite über mögliche Probleme gedacht. Das Alert-Protokoll unterscheidet sich bei \acr{tls} und \acr{dtls}
nicht voneinander, da eine zuverlässige Übertragung nicht notwendig ist. Sollte aufgrund einer verloren gegangenen Alert-Nachricht eine Anfrage
wiederholt werden, wird erneut eine Alert-Nachricht generiert. Alert-Nachrichten werden, wie auch alle anderen Daten, mit den Sicherheitsparametern
der aktuell gültigen Epoche übertragen.

\begin{figure}[ht]
  \centering
  \begin{lstlisting}[language=c]
  struct {
    AlertLevel level;
    AlertDescription description;
  } Alert;
  \end{lstlisting}
  \caption{Header des Alert-Protokolls von DTLS}
  \label{fig:alertlayer}
\end{figure}