\section{Dauer}

Der \glos{handshake} dauert unter idealen Bedingungen insgesamt 10 Sekunden wobei dort zwei Faktoren eine wesentliche Rolle spielen.
Zum einen werden während des \glospl{handshake} zwei Multiplikationen auf elliptischen Kurven durchgeführt, wobei eine Multiplikation derzeit \o 4,5 Sekunden dauert.
Zum anderen sind während des \glospl{handshake} drei Schreiboperationen, in die RW-Blöcke des Flash-Speichers, notwendig, die insgesamt 0,6 Sekunden benötigen.
Abzüglich dieser Operationen verbleiben 0,4 Sekunden, die für den Datenaustausch und weitere Berechnungen benötigt werden.

Beachtet werden muss auch, dass, bei der ersten Anfrage nach dem \glos{handshake}, eine weitere Schreiboperation in den RW-Block des Flash-Speichers notwendig ist.
Diese ist notwendig, um die Sicherheitsparameter der neuen Epoche final zu aktivieren und die alten Sicherheitsparameter zu löschen. Dadurch verzögert sich
die Antwort um 0,2 Sekunden.

Ausgehend von dem Szenario, dass neue Endgeräte bei Aktivierung von einer zentralen Software erkannt werden, die einen Handshake unmittelbar durchführt,
wird ein Benutzer davon nichts mitbekommen.