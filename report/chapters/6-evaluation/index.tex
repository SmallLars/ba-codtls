\chapter{Evaluation}

\section{Datenverkehr während des Handshakes}

Verglichen werden 3 Verfahren bei benutzung des definierten ciphersuits.
\begin{itemize}
  \item Neu
  \item DTLS original
  \item DTLS mit stateless header compression
\end{itemize}
dazu werden in den folgenden 3 abschnitten die datenmengen ermittelt, während in 4 abschnitt der vergleich erfolgt.
dabei wird ein handshake ohne datenverluste oder andere fehler und angriffe betrachtet.

je nach umfeld anders aber dort jeweils konstant:
\begin{itemize}
  \item 802.15.4 Header - Bietet 127 Byte Payload
  \item 48 Byte - 6LowPAN Header
  \item 08 Byte - UDP Header
\end{itemize}
deswegen werden diese nur im sinne von anzahl der pakete betrachtet

Folgende werden verwendet und sind bei allen 3 Verfahren gleich:\\
\begin{tabular}{l|r}
  Typ & Größe\\
  \hline
  ClientHello ohne Cookie & 57 Byte\\
  HelloVerifyRequest      & 11 Byte\\
  ClientHello mit Cookie  & 65 Byte\\
  ServerHello             & 56 Byte\\
  ServerKeyExchange       & 87 Byte\\
  ServerHelloDone         &  0 Byte\\
  ClientKeyExchange       & 87 Byte\\
  ChangeCipherSpec        &  1 Byte\\
  Finished                & 12 Byte    
\end{tabular}

darauf basierend werden nun die Datenmengen für alle 3 Verfahren ermittelt\\
ausgangssituation ist epoche 0 mit verwendetem ciphersuit

\subsection{Neu}

gemäß abbildung 3.2

\textbf{erster von drei schritten:}

ClientHello ohne cookie mit direkter Antwort: 2 Anfragen, 2 Antworten\\
in der anfrage und in der antwort ist jeweils ein contentheader a zwei byte enthalten

\begin{verbatim}
--> 3 byte + 13 byte + 32 byte   coap + uri + block 1
<-- 3 byte +  7 byte             coap + block 1
--> 3 byte + 13 byte + 27 byte   coap + uri + block 1
<-- 3 byte + 10 byte + 13 byte   coap + content type + block 1
\end{verbatim}

\textbf{zweiter von drei schritten:}

ClientHello mit cookie mit separate Antwort: 8 Anfragen, 8 Antworten\\
in der anfrage ist ein contentheader a zwei byte enthalten. in der antwort sind drei content header mit insgesamt 5 byte enthalten

\begin{verbatim}
--> 3 byte + 13 byte + 32 byte   coap + uri + block 1
<-- 3 byte +  7 byte             coap + block 1
--> 3 byte + 13 byte + 32 byte   coap + uri + block 1
<-- 3 byte +  7 byte             coap + block 1
--> 3 byte + 13 byte +  3 byte   coap + uri + block 1
<-- 3 byte +  4 byte             coap empty msg (separate)

<-- 3 byte + 11 byte + 32 byte   coap + content type + block 1 + block 2
--> 3 byte +  4 byte             coap empty msg (ack)
<-- 3 byte +  9 byte + 32 byte   coap + content type + block 2
--> 3 byte +  4 byte             coap empty msg (ack)
<-- 3 byte +  9 byte + 32 byte   coap + content type + block 2
--> 3 byte +  4 byte             coap empty msg (ack)
<-- 3 byte +  9 byte + 32 byte   coap + content type + block 2
--> 3 byte +  4 byte             coap empty msg (ack)
<-- 3 byte +  9 byte + 20 byte   coap + content type + block 2
--> 3 byte +  4 byte             coap empty msg (ack)
\end{verbatim}

\textbf{dritter von drei schritten:}

ClientKeyExchange + ChangeCipherSpec + Finished mit separate Antwort: 5 Anfragen, 5 Antworten\\
in der anfrage sind drei contentheader mit insgesamt 6 byte enthalten. in der antwort sind zwei content header mit insgesamt 4 byte enthalten

\begin{verbatim}
--> 3 byte + 22 byte + 32 byte   coap + uri + block 1
<-- 3 byte +  7 byte             coap + block 1
--> 3 byte + 22 byte + 32 byte   coap + uri + block 1
<-- 3 byte +  7 byte             coap + block 1
--> 3 byte + 22 byte + 32 byte   coap + uri + block 1
<-- 3 byte +  7 byte             coap + block 1
--> 3 byte + 22 byte + 18 byte   coap + uri + block 1
<-- 3 byte +  4 byte             coap empty msg (separate)

<-- 3 byte + 10 byte + 25 byte   coap + content type + block 1
--> 3 byte +  4 byte             coap empty msg (ack)
\end{verbatim}

\subsection{DTLS original}
127 - 48 - 8 - 13 - 12 = 46 Byte Daten

\begin{verbatim}
--> 13 byte + 12 byte + 46 byte   ClientHello ohne Cookie
--> 13 byte + 12 byte + 11 byte   ClientHello ohne Cookie

<-- 13 byte + 12 byte + 11 byte   HelloVerifyRequest

--> 13 byte + 12 byte + 46 byte   ClientHello mit Cookie
--> 13 byte + 12 byte + 19 byte   ClientHello mit Cookie

<-- 13 byte + 12 byte + 46 byte   ServerHello
<-- 13 byte + 12 byte + 10 byte   ServerHello
<-- 13 byte + 12 byte + 46 byte   ServerKeyExchange
<-- 13 byte + 12 byte + 41 byte   ServerKeyExchange
<-- 13 byte + 12 byte +  0 byte   ServerHelloDone

--> 13 byte + 12 byte + 46 byte   ClientKeyExchange
--> 13 byte + 12 byte + 41 byte   ClientKeyExchange
--> 13 byte + 12 byte +  1 byte   ChangeCipherSpec
--> 13 byte + 12 byte + 20 byte   Finished (schon verschlüsselt) 8 Byte MAC enthalten

<-- 13 byte + 12 byte +  1 byte   ChangeCipherSpec
<-- 13 byte + 12 byte + 20 byte   Finished (schon verschlüsselt) 8 Byte MAC enthalten
 \end{verbatim}

\subsection{DTLS mit stateless header compression}
%   ClientHello ohne Cookie & 57 Byte
%   HelloVerifyRequest      & 11 Byte
%   ClientHello mit Cookie  & 65 Byte
%   ServerHello             & 56 Byte
%   ServerKeyExchange       & 87 Byte
%   ServerHelloDone         &  0 Byte
%   ClientKeyExchange       & 87 Byte
%   ChangeCipherSpec        &  1 Byte
%   Finished                & 12 Byte

127 - 48 - 8 - 3 - 2 = 66 Byte Daten

mit direkter umsetzung von draft \cite[Kapitel 3]{draftcodtls}

\begin{verbatim}
--> 3 byte + 2 byte + 57 byte   ClientHello ohne Cookie

<-- 3 byte + 2 byte + 11 byte   HelloVerifyRequest

--> 3 byte + 2 byte + 65 byte   ClientHello mit Cookie

<-- 3 byte + 2 byte + 56 byte   ServerHello
<-- 3 byte + 3 byte + 65 byte   ServerKeyExchange
<-- 3 byte + 3 byte + 22 byte   ServerKeyExchange
<-- 3 byte + 2 byte +  0 byte   ServerHelloDone

--> 3 byte + 3 byte + 65 byte   ClientKeyExchange
--> 3 byte + 3 byte + 22 byte   ClientKeyExchange
--> 3 byte + 2 byte +  1 byte   ChangeCipherSpec
--> 3 byte + 2 byte + 20 byte   Finished (schon verschlüsselt) 8 Byte MAC enthalten

<-- 3 byte + 2 byte +  1 byte   ChangeCipherSpec
<-- 3 byte + 2 byte + 20 byte   Finished (schon verschlüsselt) 8 Byte MAC enthalten
 \end{verbatim}

\subsection{Vergleich}

\section{Programmgröße}

Implementierung ist 10 KiB groß\\
7,2 KiB setzen sich zusammen: \TODO{aktualisieren}
\begin{itemize}
  \item 2,41 KiB ECC-Funktionen
  \item 0,95 KiB AES-Funktionen (CCM + CMAC)
  \item 0,80 KiB Flash-Speicher-Funktionen
  \item 0,79 KiB Session-Verwaltung
  \item 0,15 KiB Pseudo-Random-Funktion
  \item 1,78 KiB Handshake-Ressource
  \item 0,32 KiB Parse \& Send
\end{itemize}

\section{Dauer}

\begin{itemize}
  \item ECC-Multiplikation: \o 4,5 s
  \item Handshake: \o 11 s
\end{itemize}