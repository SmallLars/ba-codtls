\section{Datenverkehr während des Handshakes}

Um einen fairen Vergleich durchzuführen, wird neben dem originalen \acr{dtls} auch eine \acr{dtls}-Version mit Stateless Header Compression herangezogen.
Dazu werden in den folgenden 3 Abschnitten die Datenmengen ermittelt, während im 4. Abschnitt der Vergleich erfolgt. Die Datenmengen basieren auf einem
\glos{handshake}, bei dem weder Paketverluste noch Angriffe von dritten, oder andere Fehler, auftreten.

Das Umfeld ist in allen Fällen konstant. In einem IEEE 802.15.4 Paket, lassen sich maximal 127 Byte Nutzdaten versenden. Davon werden 48 Byte für den
\acr{6lowpan}-Header, und 8 Byte für den \acr{udp}-Header benötigt, so dass 71 Byte pro Paket für den \acr{dtls}-\glos{handshake} verbleiben.
Um den Vergleich auf das Wesentliche zu reduzieren, gehen in den Vergleich zunächst nur die \acr{dtls}-Daten selbst ein. Die genannten Header
werden dann als "`Anzahl der benötigten Pakete"' in der Vergleich mit aufgenommen.

Die in dieser Arbeit genutzte \glos{ciphersuite} dient als Basis für die Ermittlung der Datenmengen. Zu beachten ist hier, dass die dort definierte \acr{prf},
keinen Einfluss auf die Datenmenge hat. Für den \glos{handshake} sind generell die in Abbildung \ref{tbl:6-1_handshake-data} aufgeführten Nachrichten mit den
dort angegebenen Größen erforderlich.

\begin{figure}[!ht]
\centering
\begin{tabular}{l|l|r}
  \hiderowcolors
  Typ & Abkürzung & Größe\\
  \hline
  ClientHello ohne Cookie & CHoC & 57 Byte\\
  HelloVerifyRequest      & HVR  & 11 Byte\\
  ClientHello mit Cookie  & CHmC & 65 Byte\\
  ServerHello             & SH   & 56 Byte\\
  ServerKeyExchange       & SKE  & 87 Byte\\
  ServerHelloDone         & SHD  &  0 Byte\\
  ClientKeyExchange       & CKE  & 87 Byte\\
  ChangeCipherSpec        & CCS  &  1 Byte\\
  Finished                & FI   & 12 Byte\\
  \showrowcolors
\end{tabular}
\caption{Größe der Handshake-Nachrichten}
\label{tbl:6-1_handshake-data}
\end{figure}

Darauf basierende werden nun die Datenmengen für alle 3 Verfahren ermittelt, wobei in Epoche 0, ohne jegliche Verschlüsselung, begonnen wird.
In den Abbildungen sind die übertragenen Pakete zur besseren Übersicht jeweils in 3 Gruppen eingeteilt. Jede Gruppe beinhaltet eine vollständige
Anfrage des Clients, sowie die vollständige Antwort des Servers.

\subsection{DTLS mit Anpassungen}
\label{sec:new-dtls}

Der \glos{handshake} über \acr{coap}, gemäß Abbildung \ref{fig:coaphandshake}, erfordert die Übertragung von 30 Datenpaketen, die in Abbildung
\ref{tbl:6-1_data-dtls-neu} dargestellt sind. Gemäß Abschnitt \ref{sec:handshake} wird eine \acr{coap}-Blockgröße von 32 Byte benutzt. Diese
Blockgröße ermöglicht im Allgemeinen auch dann noch einen \glos{handshake}, wenn der \acr{dtls}-Header die maximale Größe von 15 Byte annimmt, und der 8 Byte
lange \acr{mac} bei einer verschlüsselten Datenübertragung hinzukommt. Kritisch wird es nur bei Nachricht 5 gemäß Abbildung \ref{fig:coaphandshake}.
Da dort die \acr{uri} für die Sub-Ressource angegeben wird, die als zusätzliche \acr{coap}-Option angehängt wird, reicht der Platz im Datenpaket
nicht mehr aus. Um dies zu vermeiden, müsste auf die Modellierung von Sessions als Sub-Ressource verzichtet werden, und die Session-ID weiterhin
in der ClientHello-Nachricht übertragen werden, falls es zu einer Fortsetzung einer Session kommt. Für die Nachricht 5 stellt dies kein Problem dar,
da die Session-ID in diesem Schritt nicht benötigt wird.

Zusätzlich zu den übertragenen Handshake-Nachrichten, sind in der letzten Spalte die enthaltenen \acr{coap}-Optionen angegeben. B1 und B2 stehen
für die jeweiligen Block-Optionen, während CT die Content-Type-Option beschreibt.

\begin{figure}[!ht]
\centering
\begin{tabular}{r|c|r|r|r|r|l}
  \hiderowcolors
  Nr. & <-> & \multicolumn{1}{p{0.56cm}|}{\rotatebox{90}{Record-} \rotatebox{90}{Header}} & \multicolumn{1}{p{0.56cm}|}{\rotatebox{90}{\acr{coap}-} \rotatebox{90}{Header}} & \multicolumn{1}{p{0.56cm}|}{\rotatebox{90}{Content-} \rotatebox{90}{Header}} & \multicolumn{1}{p{0.56cm}|}{\rotatebox{90}{\glos{handshake}-} \rotatebox{90}{Daten}} & \acr{coap}-Optionen und Inhalt\\
  \hline
  \hline
   1 & --> & 3 & 13 & 2 & 30 & URI, B1, CHoC [1/2]\\
   2 & <-- & 3 &  7 &   &    & B1\\
   3 & --> & 3 & 13 &   & 27 & URI, B1, CHoC [2/2]\\
   4 & <-- & 3 & 10 & 2 & 11 & CT, B1, HVR\\
  \hline
  \hline
   5 & --> & 3 & 13 & 2 & 30 & URI, B1, CHmC [1/3]\\
   6 & <-- & 3 &  7 &   &    & B1\\
   7 & --> & 3 & 13 &   & 32 & URI, B1, CHmC [2/3]\\
   8 & <-- & 3 &  7 &   &    & B1\\
   9 & --> & 3 & 13 &   &  3 & URI, B1, CHmC [3/3]\\
  10 & <-- & 3 &  4 &   &    & EMPTY (Separate Antwort)\\
  11 & <-- & 3 & 11 & 5 & 27 & CT, B1, B2, (SH, SKE, SHD) [1/5]\\
  12 & --> & 3 &  4 &   &    & EMPTY\\
  13 & <-- & 3 &  9 &   & 32 & CT, B2, (SH, SKE, SHD) [2/5]\\
  14 & --> & 3 &  4 &   &    & EMPTY\\
  15 & <-- & 3 &  9 &   & 32 & CT, B2, (SH, SKE, SHD) [3/5]\\
  16 & --> & 3 &  4 &   &    & EMPTY\\
  17 & <-- & 3 &  9 &   & 32 & CT, B2, (SH, SKE, SHD) [4/5]\\
  18 & --> & 3 &  4 &   &    & EMPTY\\
  19 & <-- & 3 &  9 &   & 20 & CT, B2, (SH, SKE, SHD) [5/5]\\
  20 & --> & 3 &  4 &   &    & EMPTY\\
  \hline
  \hline
  21 & --> & 3 & 22 & 6 & 26 & URI, B1, (CKE, CCS, FI) [1/4]\\
  22 & <-- & 3 &  7 &   &    & B1\\
  23 & --> & 3 & 22 &   & 32 & URI, B1, (CKE, CCS, FI) [2/4]\\
  24 & <-- & 3 &  7 &   &    & B1\\
  25 & --> & 3 & 22 &   & 32 & URI, B1, (CKE, CCS, FI) [3/4]\\
  26 & <-- & 3 &  7 &   &    & B1\\
  27 & --> & 3 & 22 &   & 18 & URI, B1, (CKE, CCS, FI) [4/4]\\
  28 & <-- & 3 &  4 &   &    & EMPTY (Separate Antwort)\\
  29 & <-- & 3 & 10 & 4 & 21 & CT, B1, CCS, FI\\
  30 & --> & 3 &  4 &   &    & EMPTY\\
  \hline
  \hline
    & <-> & 90 & 294 & 21 & 405 & Gesamt 810 \\
  \showrowcolors
\end{tabular}
\caption[test 1]{Datenaustausch während eines Handshake mit angepasstem DTLS}
\label{tbl:6-1_data-dtls-neu}
\end{figure}

Der Record-Header nimmt hier generell drei Byte in Anspruch. Zwei Byte beinhalten den komprimierten Record-Header, während die Sequenznummer in einem extra Byte
angehängt werden muss. Der \acr{coap}-Header ist minimal 4 Byte groß. Während bei einer Anfrage der \acr{uri} und, je nach Datenmenge, die Block-1-Option
hinzukommen, sind dies bei einer Antwort der Content-Type und die Block-2-Option. Entsprechend der Anzahl der enthaltenen \glos{handshake}-Nachrichten kommt der
Content-Header mit jeweils 2 Byte hinzu. Dieser setzt sich aus dem ein Byte langen komprimiertem Content-Header, und der 1 Byte langen Längenangabe zusammen.
Die einzige Ausnahme bildet hier der Content-Header für die Finished-Nachricht, der mit einem Byte auskommt, da die Finished-Nachricht eine Länge von 0 hat.

Insgesamt müssen 810 Byte in 30 Paketen übertragen werden. Zu beachten ist, dass 14 Pakete mit 116 Byte \acr{coap} spezifisch sind und keinerlei
\glos{handshake}-Daten enthalten. Diese müssen zwar berücksichtig werden, sind aber dennoch eine separate Betrachtung wert. Ohne diese verbleiben
16 Pakete mit 694 Byte. Auffällig ist auch, dass die Nachricht 5 in vier Blöcke unterteilt werden muss, und so der \acr{uri} entsprechend oft wiederholt wird.
Würde auf die Modellierung der Sessions als Sub-Ressource verzichtet werden, könnten in diesem Fall weitere 36 Byte eingespart werden.

\subsection{DTLS}

Der \glos{handshake}, gemäß Abbildung \ref{fig:handshake}, erfordert die Übertragung von 18 Datenpaketen, die in Abbildung \ref{tbl:6-1_data-dtls} dargestellt sind.
Während im letzten Abschnitt die Menge der Daten in einem Paket durch die Blockgröße von \acr{coap} vorgegeben war, muss diese hier ermittelt werden.
Ausgehend von den genannten 71 Byte pro Paket, sind 13 Byte für einen maximalen Record-Header, und 12 Byte für einen maximalen Handshake-Header zu berücksichtigen.
Ebenfalls muss ein 8 Byte langer \acr{mac} berücksichtig werden. Insgesamt stehen somit in einem Paket 38 Byte für \glos{handshake}-Daten zur Verfügung.

\begin{figure}[!ht]
\centering
\begin{tabular}{r|c|r|r|r|l}
  \hiderowcolors
  Nr. & <-> & \multicolumn{1}{p{0.56cm}|}{\rotatebox{90}{Record-} \rotatebox{90}{Header}} & \multicolumn{1}{p{0.56cm}|}{\rotatebox{90}{Content-} \rotatebox{90}{Header}} & \multicolumn{1}{p{0.56cm}|}{\rotatebox{90}{\glos{handshake}-} \rotatebox{90}{Daten}} & Inhalt\\
  \hline
  \hline
   1 & --> & 13 & 12 & 38 & CHello ohne Cookie [1/2]\\
   2 & --> & 13 & 12 & 19 & CHello ohne Cookie [2/2]\\
   3 & <-- & 13 & 12 & 11 & HelloVerifyRequest\\
  \hline
  \hline
   4 & --> & 13 & 12 & 38 & CHello mit Cookie [1/2]\\
   5 & --> & 13 & 12 & 27 & CHello mit Cookie [2/2]\\
   6 & <-- & 13 & 12 & 38 & SHello [1/2]\\
   7 & <-- & 13 & 12 & 18 & SHello [2/2]\\
   8 & <-- & 13 & 12 & 38 & SKeyExchange [1/3]\\
   9 & <-- & 13 & 12 & 38 & SKeyExchange [2/3]\\
  10 & <-- & 13 & 12 & 11 & SKeyExchange [3/3]\\
  11 & <-- & 13 & 12 &  0 & SHelloDone\\
  \hline
  \hline
  12 & --> & 13 & 12 & 38 & CKeyExchange [1/3]\\
  13 & --> & 13 & 12 & 38 & CKeyExchange [2/3]\\
  14 & --> & 13 & 12 & 11 & CKeyExchange [3/3]\\
  15 & --> & 13 &    &  1 & ChangeCipherSpec\\
  16 & --> & 13 & 12 & 20 & Finished inklusive 8 Byte MAC\\
  17 & <-- & 13 &    &  1 & ChangeCipherSpec\\
  18 & <-- & 13 & 12 & 20 & Finished inklusive 8 Byte MAC\\
  \hline
  \hline
    & <-> & 234 & 192 & 405 & Gesamt 831\\
  \showrowcolors
\end{tabular}
\caption{Datenaustausch während eines Handshake mit DTLS}
\label{tbl:6-1_data-dtls}
\end{figure}

Für den Record-Header werden generell 13 Byte benötigt, während der \glos{handshake}-Header 12 Byte beansprucht. Die Ausnahme bildet hier die ChangeCipherSpec-Nachricht.
Diese ist keine \glos{handshake}-Nachricht, womit hier der \glos{handshake}-Header wegfällt. Entsprechend der maximalen Datenmenge von 38 Byte sind die Handshake-Nachrichten
auf mehrere Pakete verteilt. Insgesamt werden 18 Pakete mit 831 Byte übertragen.

\subsection{DTLS mit Stateless Header Compression}

In diesem Abschnitt wird der \glos{handshake}, gemäß Abbildung \ref{fig:handshake}, mit einer Stateless Header Compression analysiert. Dieser erfordert,
wie auch der \glos{handshake} aus dem vorigen Abschnitt, die Übertragung von 18 Datenpaketen, die in Abbildung \ref{tbl:6-1_data-dtls-comp} dargestellt sind.
Die Stateless Header Compression wird dabei direkt aus dem \acr{ietf}-Entwurf \cite[Kapitel 3]{draftcodtls} übernommen.

Die maximale Datenmenge wird ebenfalls ausgehend von 71 Byte ermittelt. Neben dem 8 Byte langem \acr{mac} müssen dort der maximale Record-Header mit 15 Byte, und
der maximale \glos{handshake}-Header mit 14 Byte, berücksichtigt werden, so dass 34 Byte pro Paket verbleiben.

\begin{figure}[!ht]
\centering
\begin{tabular}{r|c|r|r|r|l}
  \hiderowcolors
  Nr. & <-> & \multicolumn{1}{p{0.56cm}|}{\rotatebox{90}{Record-} \rotatebox{90}{Header}} & \multicolumn{1}{p{0.56cm}|}{\rotatebox{90}{Content-} \rotatebox{90}{Header}} & \multicolumn{1}{p{0.56cm}|}{\rotatebox{90}{\glos{handshake}-} \rotatebox{90}{Daten}} & Inhalt\\
  \hline
  \hline
   1 & --> & 3 & 4 & 34 & CHello ohne Cookie [1/2]\\
   2 & --> & 3 & 4 & 23 & CHello ohne Cookie [2/2]\\
   3 & <-- & 3 & 2 & 11 & HelloVerifyRequest\\
  \hline
  \hline
   4 & --> & 3 & 4 & 34 & CHello mit Cookie [1/2]\\
   5 & --> & 3 & 4 & 31 & CHello mit Cookie [2/2]\\
   6 & <-- & 3 & 4 & 34 & SHello [1/2]\\
   7 & <-- & 3 & 4 & 22 & SHello [2/2]\\
   8 & <-- & 3 & 4 & 34 & SKeyExchange [1/3]\\
   9 & <-- & 3 & 4 & 34 & SKeyExchange [2/3]\\
  10 & <-- & 3 & 4 & 19 & SKeyExchange [3/3]\\
  11 & <-- & 3 & 2 &  0 & SHelloDone\\
  \hline
  \hline
  12 & --> & 3 & 4 & 34 & CKeyExchange [1/3]\\
  13 & --> & 3 & 4 & 34 & CKeyExchange [2/3]\\
  14 & --> & 3 & 4 & 19 & CKeyExchange [3/3]\\
  15 & --> & 3 &   &  1 & ChangeCipherSpec\\
  16 & --> & 3 & 2 & 20 & Finished inklusive 8 Byte MAC\\
  17 & <-- & 3 &   &  1 & ChangeCipherSpec\\
  18 & <-- & 3 & 2 & 20 & Finished inklusive 8 Byte MAC\\
  \hline
  \hline
    & <-> & 54 & 56 & 405 & Gesamt 515\\
  \showrowcolors
\end{tabular}
\caption{Datenaustausch während eines Handshake mit DTLS und stateless header compression}
\label{tbl:6-1_data-dtls-comp}
\end{figure}

Wie auch in Abschnitt \ref{sec:new-dtls} werden für den Record-Header 3 Byte benötigt. Neben dem 2 Byte langen komprimiertem Record-Header wird dort die 1 Byte lange
Sequenznummer übertragen. Der \glos{handshake}-Header kommt generell mit 2 Byte, ohne weitere Anhänge, aus, falls die \glos{handshake}-Nachrichten klein genug sind,
um in einem Paket übertragen zu werden. Müssen diese in mehrere Teile unterteilt werden, kommt eine 1 Byte lange Gesamtlänge, sowie ein 1 Byte langer Offset hinzu.
Diese Angaben sind notwendig, um die einzelnen Teile wieder zusammenzusetzen. Wie auch im letzten Abschnitt, wird für die ChangeCipherSpec-Nachricht kein
\glos{handshake}-Header benötigt. Insgesamt werden 18 Pakete mit 515 Byte übertragen.

\subsection{Vergleich}

In Abbildung \ref{tbl:6-1_vergleich} sind im oberen Bereich nun alle 3 Varianten gegenübergestellt, wobei im unteren Bereich die denkbaren
Alternativen und Betrachtungsweisen aufgeführt sind. In der letzten Spalte sind zusätzlich die die Header von \acr{6lowpan} und \acr{udp},
mit insgesamt 56 Byte pro Paket, berücksichtigt.

\begin{figure}[!ht]
\centering
\begin{tabular}{l|r|r|r}
  \hiderowcolors
  Variante & Pakete & Datenmenge & Gesamt\\
  \hline
  DTLS mit Anpassungen                                         & 30 & 810 & 2490\\
  DTLS                                                         & 18 & 831 & 1839\\
  DTLS mit Stateless Header Compression                        & 18 & 515 & 1523\\
  \hline
  \hline
  DTLS mit Anpassungen (ohne \acr{coap}-ACK)                   & 16 & 694 & 1590\\
  DTLS mit Anpassungen (ohne Sub-Ressource)                    & 30 & 774 & 2454\\
  DTLS mit Anpassungen (ohne Sub-Ressource und \acr{coap}-ACK) & 16 & 658 & 1554\\
  \showrowcolors
\end{tabular}
\caption{Vergleich der drei Varianten}
\label{tbl:6-1_vergleich}
\end{figure}

Die angepasste \acr{dtls}-Variante schneidet bezüglich der Anzahl der benötigten Pakete zunächst am schlechtesten ab. Jedoch muss berücksichtig werden,
dass 14 Pakete davon \acr{coap}-Pakete sind, die keine \glos{handshake}-Daten enthalten. Diese ermöglichen eine zuverlässige blockweise Datenübertragung,
so dass dafür im Gegenzug einiges an Programmcode wegfällt, der diese Dinge ausgleichem müsste. Mit den verbleibenden 16 Paketen liegt die angepasste
\acr{coap}-Variante vorne. Zwei Pakete weniger sind hier nötig, da die \glos{handshake}-Nachrichten kompakt in \acr{coap}-Paketen aneinander gehängt werden,
wodurch das verfügbare Volumen der Datenpakete maximal ausgenutzt wird.

Werden die \acr{dtls} spezifischen Datenmengen verglichen, schneidet die angepasste \acr{dtls}-Variante besser ab als das originale \acr{dtls}.
\acr{dtls} mit Stateless Header Compression ist jedoch, bezogen auf die Datenmenge, am effizientesten, da hier keine \acr{coap}-Header übertragen werden.