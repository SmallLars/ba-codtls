\section{Datenverkehr während des Handshakes}

Um einen fairen Vergleich durchzuführen, wird neben dem originalen \acr{dtls} auch eine \acr{dtls}-Version mit Stateless Header Compression herangezogen.
Dazu werden in den folgenden 3 Abschnitten die Datenmengen ermittelt, während im 4. Abschnitt der Vergleich erfolgt. Die Datenmengen basieren auf einem
\glos{handshake}, bei dem weder Paketverluste noch Angriffe von dritten, oder andere Fehler, auftreten.

Das Umfeld ist in allen Fällen konstant. In einem IEEE 802.15.4 Paket, lassen sich maximal 127 Byte Nutzdaten versenden. Davon werden 48 Byte für den
\acr{6lowpan}-Header, und 8 Byte für den \acr{udp}-Header benötigt, so dass 71 Byte pro Paket für den \acr{dtls}-\glos{handshake} verbleiben.
Um den Vergleich auf das Wesentliche zu reduzieren, gehen in den Vergleich zunächst nur die \acr{dtls}-Daten selbst ein. Die genannten Header
werden dann als "`Anzahl der benötigten Pakete"' in der Vergleich mit aufgenommen.

Die in dieser Arbeit genutzte \glos{ciphersuite} dient als Basis für die Ermittlung der Datenmengen. Zu beachten ist hier, dass die dort definierte \acr{prf},
keinen Einfluss auf die Datenmenge hat. Für den \glos{handshake} sind generell die in Abbildung \ref{tbl:6-1_handshake-data} aufgeführten Nachrichten mit den
dort angegebenen Größen erfolderlich.

\begin{figure}[!ht]
\centering
\begin{tabular}{l|r}
  \hiderowcolors
  Typ & Größe\\
  \hline
  ClientHello ohne Cookie & 57 Byte\\
  HelloVerifyRequest      & 11 Byte\\
  ClientHello mit Cookie  & 65 Byte\\
  ServerHello             & 56 Byte\\
  ServerKeyExchange       & 87 Byte\\
  ServerHelloDone         &  0 Byte\\
  ClientKeyExchange       & 87 Byte\\
  ChangeCipherSpec        &  1 Byte\\
  Finished                & 12 Byte\\
  \showrowcolors
\end{tabular}
\caption{Größe der Handshake-Nachrichten}
\label{tbl:6-1_handshake-data}
\end{figure}

Darauf basierende werden nun die Datenmengen für alle 3 Verfahren ermittelt, wobei in Epoche 0, ohne jegliche Verschlüsselung, begonnen wird.

\subsection{DTLS mit Anpassungen}

gemäß abbildung 3.2

in kapitel 3.2 hat sich blockgröße von 32 byte ergeben die hier herangezogen wird

\begin{figure}[!ht]
\centering
\begin{tabular}{r|c|r|r|r|r|l}
  \hiderowcolors
  Nr. & <-> & \multicolumn{1}{p{0.56cm}|}{\rotatebox{90}{Record-} \rotatebox{90}{Header}} & \multicolumn{1}{p{0.56cm}|}{\rotatebox{90}{\acr{coap}-} \rotatebox{90}{Header}} & \multicolumn{1}{p{0.56cm}|}{\rotatebox{90}{Content-} \rotatebox{90}{Header}} & \multicolumn{1}{p{0.56cm}|}{\rotatebox{90}{\glos{handshake}-} \rotatebox{90}{Daten}} & \acr{coap}-Optionen und Inhalt\\
  \hline
  \hline
   1 & --> & 3 & 13 & 2 & 30 & URI, B1, CHello ohne Cookie [1/2]\\
   2 & <-- & 3 &  7 &   &    & B1\\
   3 & --> & 3 & 13 &   & 27 & URI, B1, CHello ohne Cookie [2/2]\\
   4 & <-- & 3 & 10 & 2 & 11 & CT, B1, HelloVerifyRequest\\
  \hline
  \hline
   5 & --> & 3 & 13 & 2 & 30 & URI, B1, CHello mit Cookie [1/3]\\
   6 & <-- & 3 &  7 &   &    & B1\\
   7 & --> & 3 & 13 &   & 32 & URI, B1, CHello mit Cookie [2/3]\\
   8 & <-- & 3 &  7 &   &    & B1\\
   9 & --> & 3 & 13 &   &  3 & URI, B1, CHello mit Cookie [3/3]\\
  10 & <-- & 3 &  4 &   &    & EMPTY (Separate Antwort)\\
  11 & <-- & 3 & 11 & 5 & 27 & CT, B1, B2, SHello, SKeyExchange, SHelloDone [1/5]\\
  12 & --> & 3 &  4 &   &    & EMPTY\\
  13 & <-- & 3 &  9 &   & 32 & CT, B2, SHello, SKeyExchange, SHelloDone [2/5]\\
  14 & --> & 3 &  4 &   &    & EMPTY\\
  15 & <-- & 3 &  9 &   & 32 & CT, B2, SHello, SKeyExchange, SHelloDone [3/5]\\
  16 & --> & 3 &  4 &   &    & EMPTY\\
  17 & <-- & 3 &  9 &   & 32 & CT, B2, SHello, SKeyExchange, SHelloDone [4/5]\\
  18 & --> & 3 &  4 &   &    & EMPTY\\
  19 & <-- & 3 &  9 &   & 20 & CT, B2, SHello, SKeyExchange, SHelloDone [5/5]\\
  20 & --> & 3 &  4 &   &    & EMPTY\\
  \hline
  \hline
  21 & --> & 3 & 22 & 6 & 26 & URI, B1, CKeyExchange, ChangeCipherSpec, Finished [1/4]\\
  22 & <-- & 3 &  7 &   &    & B1\\
  23 & --> & 3 & 22 &   & 32 & URI, B1, CKeyExchange, ChangeCipherSpec, Finished [2/4]\\
  24 & <-- & 3 &  7 &   &    & B1\\
  25 & --> & 3 & 22 &   & 32 & URI, B1, CKeyExchange, ChangeCipherSpec, Finished [3/4]\\
  26 & <-- & 3 &  7 &   &    & B1\\
  27 & --> & 3 & 22 &   & 18 & URI, B1, CKeyExchange, ChangeCipherSpec, Finished [4/4]\\
  28 & <-- & 3 &  4 &   &    & EMPTY (Separate Antwort)\\
  29 & <-- & 3 & 10 & 4 & 21 & CT, B1, ChangeCipherSpec, Finished\\
  30 & --> & 3 &  4 &   &    & EMPTY\\
  \hline
  \hline
    & <-> & 90 & 294 & 21 & 405 & Gesamt 810 \\
  \showrowcolors
\end{tabular}
\caption{Datenaustausch während eines Handshake mit angepasstem DTLS}
\label{tbl:6-1_data-dtls-neu}
\end{figure}

coap packete die nur zur bestätigung dienen und keine daten enthalten rausrechnen zum zusätzlichen vergleich

zu beachten ist, dass 14 Datenpakete mit 116 Byte coap spezifisch sind und keinerlei dtls handshake daten enthalten\\
die müssen zwar berücksichtig werden, sind aber dennoch eine separate betrachtung wert.\\

\subsection{DTLS}

maximale payload größe: 127 - 48 - 8 - 13 - 12 - 8 = 38

\begin{figure}[!ht]
\centering
\begin{tabular}{r|c|r|r|r|l}
  \hiderowcolors
  Nr. & <-> & \multicolumn{1}{p{0.56cm}|}{\rotatebox{90}{Record-} \rotatebox{90}{Header}} & \multicolumn{1}{p{0.56cm}|}{\rotatebox{90}{Content-} \rotatebox{90}{Header}} & \multicolumn{1}{p{0.56cm}|}{\rotatebox{90}{\glos{handshake}-} \rotatebox{90}{Daten}} & Inhalt\\
  \hline
  \hline
   1 & --> & 13 & 12 & 38 & CHello ohne Cookie [1/2]\\
   2 & --> & 13 & 12 & 19 & CHello ohne Cookie [2/2]\\
   3 & <-- & 13 & 12 & 11 & HelloVerifyRequest\\
  \hline
  \hline
   4 & --> & 13 & 12 & 38 & CHello mit Cookie [1/2]\\
   5 & --> & 13 & 12 & 27 & CHello mit Cookie [2/2]\\
   6 & <-- & 13 & 12 & 38 & SHello [1/2]\\
   7 & <-- & 13 & 12 & 18 & SHello [2/2]\\
   8 & <-- & 13 & 12 & 38 & SKeyExchange [1/3]\\
   9 & <-- & 13 & 12 & 38 & SKeyExchange [2/3]\\
  10 & <-- & 13 & 12 & 11 & SKeyExchange [3/3]\\
  11 & <-- & 13 & 12 &  0 & SHelloDone\\
  \hline
  \hline
  12 & --> & 13 & 12 & 38 & CKeyExchange [1/3]\\
  13 & --> & 13 & 12 & 38 & CKeyExchange [2/3]\\
  14 & --> & 13 & 12 & 11 & CKeyExchange [3/3]\\
  15 & --> & 13 & 12 &  1 & ChangeCipherSpec\\
  16 & --> & 13 & 12 & 20 & Finished inklusive 8 Byte MAC\\
  17 & <-- & 13 & 12 &  1 & ChangeCipherSpec\\
  18 & <-- & 13 & 12 & 20 & Finished inklusive 8 Byte MAC\\
  \hline
  \hline
    & <-> & 234 & 216 & 405 & Gesamt 861\\
  \showrowcolors
\end{tabular}
\caption{Datenaustausch während eines Handshake mit DTLS}
\label{tbl:6-1_data-dtls}
\end{figure}

\subsection{DTLS mit Stateless Header Compression}

mit direkter umsetzung von draft \cite[Kapitel 3]{draftcodtls}

da die angepasste version einen maximalen dtls header ermöglicht, wird hier ebenfalls davon ausgegangen:\\
maximale payload größe: 127 - 48 - 8 - 15 - 14 - 8 = 34

\begin{figure}[!ht]
\centering
\begin{tabular}{r|c|r|r|r|l}
  \hiderowcolors
  Nr. & <-> & \multicolumn{1}{p{0.56cm}|}{\rotatebox{90}{Record-} \rotatebox{90}{Header}} & \multicolumn{1}{p{0.56cm}|}{\rotatebox{90}{Content-} \rotatebox{90}{Header}} & \multicolumn{1}{p{0.56cm}|}{\rotatebox{90}{\glos{handshake}-} \rotatebox{90}{Daten}} & Inhalt\\
  \hline
  \hline
   1 & --> & 3 & 4 & 34 & CHello ohne Cookie [1/2]\\
   2 & --> & 3 & 4 & 23 & CHello ohne Cookie [2/2]\\
   3 & <-- & 3 & 2 & 11 & HelloVerifyRequest\\
  \hline
  \hline
   4 & --> & 3 & 4 & 34 & CHello mit Cookie [1/2]\\
   5 & --> & 3 & 4 & 31 & CHello mit Cookie [2/2]\\
   6 & <-- & 3 & 4 & 34 & SHello [1/2]\\
   7 & <-- & 3 & 4 & 22 & SHello [2/2]\\
   8 & <-- & 3 & 4 & 34 & SKeyExchange [1/3]\\
   9 & <-- & 3 & 4 & 34 & SKeyExchange [2/3]\\
  10 & <-- & 3 & 4 & 19 & SKeyExchange [3/3]\\
  11 & <-- & 3 & 2 &  0 & SHelloDone\\
  \hline
  \hline
  12 & --> & 3 & 4 & 34 & CKeyExchange [1/3]\\
  13 & --> & 3 & 4 & 34 & CKeyExchange [2/3]\\
  14 & --> & 3 & 4 & 19 & CKeyExchange [3/3]\\
  15 & --> & 3 & 2 &  1 & ChangeCipherSpec\\
  16 & --> & 3 & 2 & 20 & Finished inklusive 8 Byte MAC\\
  17 & <-- & 3 & 2 &  1 & ChangeCipherSpec\\
  18 & <-- & 3 & 2 & 20 & Finished inklusive 8 Byte MAC\\
  \hline
  \hline
    & <-> & 54 & 60 & 405 & Gesamt 519\\
  \showrowcolors
\end{tabular}
\caption{Datenaustausch während eines Handshake mit DTLS und stateless header compression}
\label{tbl:6-1_data-dtls-comp}
\end{figure}

\subsection{Vergleich}

tada