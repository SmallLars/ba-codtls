\section{Programmgröße}

In Abbildung \ref{tbl:contiki-speicher-dtls} sind zunächst die Ausgangs-Werte aus Abschnitt \ref{sec:server}
im Vergleich mit den Werten inklusive der \acr{dtls}-Implementierung aufgeführt. Daraus ergibt sich, dass
für die Funktionen des Endgeräts 4407 Byte (4,30 KiB) verbleiben.

\begin{figure}[!ht]
\centering
\begin{tabular}{l|r|r|r}
  \hiderowcolors
  \textbf{Beschreibung} & \textbf{Basis} & \textbf{mit DTLS} & \textbf{Differenz}\\
  \hline
  Programm        & ~~~~~58760 Byte & ~~~~~70008 Byte & ~~~~~11248 Byte \\
  Irq Stack       &   256 Byte      &   256 Byte      &     0 Byte\\
  Fiq Stack       &   256 Byte      &   256 Byte      &     0 Byte\\
  Svc Stack       &   256 Byte      &   256 Byte      &     0 Byte\\
  Abt Stack       &    16 Byte      &    16 Byte      &     0 Byte\\
  Und Stack       &    16 Byte      &    16 Byte      &     0 Byte\\
  Sys Stack       &  2048 Byte      &  2048 Byte      &     0 Byte\\
  Datensegment    & 20744 Byte      & 21025 Byte      &   281 Byte\\
  Heap            &    16 Byte      &    16 Byte      &     0 Byte\\
  \hline
  \textbf{Gesamt} & 82368 Byte      & 93897 Byte      & 11529 Byte\\
                  & 80,44 KiB       & 91,70 KiB       & 11,26 KiB\\
  \showrowcolors
\end{tabular}
\caption{Speicheraufteilung von SmartAppContiki ohne und mit DTLS}
\label{tbl:contiki-speicher-dtls}
\end{figure}

Der Zuwachs im Datensegment resultiert mit 128 Byte aus dem, für die Behandlung der Block-1-Option notwendigen, Speicher, in dem die Datenblöcke gesammelt und
zusammen gesetzt werden. Außerdem werden 80 Byte benötigt um je zwei Sequenznummern für die maximal 10 Sessions abzulegen. Von den 11248 Byte zusätzlichem
Programmcode, setzen sich 9114 Byte aus den folgen Komponenten zusammen:
\begin{itemize}
  \item  2848 Byte \glos{handshake}-Ressource
  \item  2144 Byte \acr{ecc}-Funktionen
  \item  1056 Byte Parse \& Send
  \item ~~896 Byte Session-Verwaltung
  \item ~~824 Byte Flash-Speicher-Funktionen
  \item ~~332 Byte \acr{aes}-\acr{cmac}
  \item ~~328 Byte \acr{aes}-\acr{ccm}
  \item ~~310 Byte \acr{aes}
  \item ~~192 Byte \acr{coap}-Block-1-Handler
  \item ~~184 Byte \acr{prf}
\end{itemize}
Die übrigend 2134 Byte setzten sich zusammen aus kleineren Anpassungen in \acr{coap}, und einigen Ressourcen zum Abruf von gerätespezifischen Daten,
sowie jeweils einer Ressource für das Update der Realzeit und der Software. Zu beachten ist auch, das weitere 1400 Byte im Flash-Speicher genutzt werden,
um die Daten von 10 Sessions abzulegen. Auch sind dort 96 Byte genutzt, die den Basispunkt und die Ordnung der elliptischen Kurve enthalten.

Es hat sich gezeigt, dass von den 2048 Byte des System-Stacks nur maximal 1504 Byte genutzt werden. Sollte weiterer Speicher benötigt werden, könnte
der System-Stack auf 1536 Byte begrenzt werden, wodurch weitere 0,5 KiB an Speicher verfübar wären.

Im \acr{ietf}-Entwurf "`A Hitchhiker's Guide to the (D)TLS Protocol"' \cite{draftmintls} werden Programmgrößen für einzelne
Funktionen aufgeführt, die sich jedoch nicht zum direkten Vergleich eignen, da diese für eine 64-bit-Architektur kompiliert wurden.
Um einige Programmgrößen zu ermitteln wird deshalb TinyDTLS \cite{tinydtls} herangezogen und mit Contiki kompiliert.
Die aktuelle Version ist \textasciitilde 7 KiB zu groß für den \glos{mc1322} und konnte somit nicht getestet werden. Für die Ermittlung
der Programmgrößen einzelner Komponenten reicht dies jedoch aus. Einige nennenswerte Komponenten sind in Abbildung \ref{tbl:tinydtls} aufgeführt.

\begin{figure}[!ht]
\centering
\begin{tabular}{l|r|r}
  \hiderowcolors
  \textbf{Komponente} & \textbf{Programmgröße} & \textbf{Datensegment}\\
  \hline
  \acr{aes}  & 1692 Byte & 4096 Byte\\
  SHA2       & 1072 Byte &  288 Byte\\
  \acr{hmac} &  348 Byte &    0 Byte\\
  \acr{prf}  &  504 Byte &    0 Byte\\
  \showrowcolors
\end{tabular}
\caption{Größe einiger nennenswerter Komponenten aus TinyDTLS}
\label{tbl:tinydtls}
\end{figure}

\acr{aes} benötigt neben 1692 Byte Programmcode 4096 Byte im Datensegment für einige Konstanten. Es zeigt sich hier also, dass die \acr{aes}-Unterstützung
des \glos{mc1322} wesentlich für das Gelingen dieser Realisierung ist. Eine Software-Lösung wäre zu groß für den zur Verfügung stehenden Platz.

Anders verhält es sich mit der genutzten \acr{prf}. Ersetzt man die genutzte \acr{prf} und die dafür notwendige \acr{aes}-\acr{cmac}-Funktion durch die
TinyDTLS-Komponenten \acr{prf}, \acr{hmac} und SHA2 wird das Programm um 1696 Byte (1,66 KiB) größer. Dadurch würden immer noch mehr als 2,5 KiB
Speicher zur Verfügung stehen, um Funktionen des Endgeräts zu realisieren. Von Vorteil wäre dies, da somit die in \acr{dtls} definierte \acr{prf}
benutzt wird, und die Definition der \acr{prf} in der benutzen \glos{ciphersuite} überflüssig ist.