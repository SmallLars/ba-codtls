\section{Sicherheit}

Der in Abschnitt \ref{sec:ori-handshake} beschriebene Cookie soll den Absender validieren und so \acr{dos}-Angriffe verhindern.
Dabei liegt der Fokus darauf, die Erstellung einer Session durch einen Angreifer zu vermeiden, da dieser so die Ressourcen des Endgeräts aufbrauchen könnte.
Diese Funktion ist auch bei der Realisierung des \glospl{handshake} über \acr{coap} gegeben. Durch die Verwendung der Block-1-Option treten jedoch neue
Angriffsmöglichkeiten auf, die in dieser Arbeit nicht verhindert werden konnten. Die \acr{coap}-Ressource für den \acr{dtls}-\glos{handshake} akzeptiert, aufgrund
der begrenzten Ressourcen, nur eine Anfrage zur Zeit. Eine Anfrage kann jedoch aus mehreren \acr{coap}-Paketen mit der Block-1-Option bestehen. Während
die Datenblöcke gesammelt werden, ist die Ressource für alle anderen Anfragen gesperrt. Zwar wird die Ressource nach 60 Sekunden freigegeben, falls die Anfrage
nicht beendet wurde, was einem Angreifer jedoch die Möglichkeit offen lässt, alle 60 Sekunden eine neue Anfrage zu senden und die Ressource erneut zu sperren.

Um Angriffe dieser Art zu verhindern, ist es notwendig, für den ersten Validierungsprozess auf die Nutzung der Block-1-Option zu verzichten, oder dafür ausschließlich
die Daten im ersten Datenblock zu nutzen. Letzteres wäre möglich, durch die Anpassung der ClientHello-Nachricht, wie in Abbildung \ref{fig:clienthelloalternative} dargestellt.
Dort wird gleich hinter der Version ein Hash, der aus ClientHello-Nachricht ohne Hash und Cookie resultiert, hinterlegt, worauf direkt der Cookie folgt.

\begin{figure}[ht]
  \centering
  \begin{lstlisting}[language=c]
  struct {
    ProtocolVersion client_version;
    opaque client_hello_hash<0..2^8-1>;                  // New field
    opaque cookie<0..2^8-1>;                             // New field
    Random random;
    SessionID session_id;
    CipherSuite cipher_suites<2..2^16-1>;
    CompressionMethod compression_methods<1..2^8-1>;
  } ClientHello;
  \end{lstlisting}
  \caption{Alternative zur ClientHello-Nachricht}
  \label{fig:clienthelloalternative}
\end{figure}

Ausgehend von einem 16 Byte langen Hash-Wert und einem 8 Byte langen Cookie, passen diese Werte exakt in den ersten 32 Byte großen Block. Fehlt dort der Cookie,
kann die HelloVerifyRequest-Nachricht auf Basis des Hash-Wertes sofort berechnet und versendet werden. Der Client kann die Anfrage somit umgehend, inklusive dem
Cookie, erneut beginnen. Wie im nächsten Abschnitt zu sehen, würde dies sogar den Versand von 2 Datenpaketen einsparen. Außerdem ist es auf Serverseite
nicht mehr notwendig, den Hash-Wert der vollständigen ClientHello-Nachricht zu berechnen, was weitere Ressourcen spart. Falls sich eine Überbrüfung des Hash-Wertes
als notwendig erweist, kann dieser trotzdem berechnet und verglichen werden.