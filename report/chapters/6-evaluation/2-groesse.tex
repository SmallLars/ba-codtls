\section{Programmgröße}

Aktuelle Werte:

\begin{figure}[!ht]
\centering
\begin{tabular}{l|r}
  \hiderowcolors
  \textbf{Beschreibung} & \textbf{DTLS}\\
  \hline
  Programm        & ~~~~~70008 Byte\\
  Irq Stack       &   256 Byte\\
  Fiq Stack       &   256 Byte\\
  Svc Stack       &   256 Byte\\
  Abt Stack       &    16 Byte\\
  Und Stack       &    16 Byte\\
  Sys Stack       &  2048 Byte\\
  Datensegment    & 21025 Byte\\
  Heap            &    16 Byte\\
  \hline
  \textbf{Gesamt} & 93897 Byte\\
                  & 91,70 KiB\\
  \showrowcolors
\end{tabular}
\caption{Speicheraufteilung von SmartAppContiki mit DTLS}
\label{tbl:contiki-speicher-dtls}
\end{figure}

Implementierung ist 93897 - 82368 = 11529 = 11,26 KiB groß\\
8,41 KiB setzen sich zusammen:
\begin{itemize}
  \item 2,09 KiB ECC-Funktionen
  \item 0,95 KiB AES-Funktionen (CCM + CMAC)
  \item 0,80 KiB Flash-Speicher-Funktionen
  \item 0,88 KiB Session-Verwaltung
  \item 0,18 KiB Pseudo-Random-Funktion
  \item 2,48 KiB Handshake-Ressource
  \item 1,03 KiB Parse \& Send
\end{itemize}

Gesamtgröße: 93897 = 91,70 KiB\\
Verbleiben   98304 - 93897 = 4407 = 4,30 KiB\\
-> praxistauglich

die maximale stackausnutzung beträgt 1504 byte von 2048\\
somit wäre es möglich den stack von 2048 Byte auf 1536 Byte zu reduzieren, um 0,5 KiB weiteren Speicher nutzen zu können