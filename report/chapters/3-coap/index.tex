\chapter{CoAP}
\label{chp:coap}

\acr{coap} ist wie \acr{http} ein auf der REST-Architektur basierendes Protokoll. Im Unterschied zu \acr{http} wurde
\acr{coap} jedoch für den Einsatz in eingeschränkten Umgebunden, in denen \acr{6lowpan} und \acr{udp} üblich sind, entwickelt.
Hier müssen bei Bedarf Paketverluste kompensiert, sowie Paket-Dopplungen oder Änderungen in der Reihenfolge erkannt und behoben werden.

\TODO{Die wichtigsten Methoden erläutern: GET POST DELET}

Auf die 2 für die Umsetzung wichtigen Funktionen soll es in den folgenden Unterkapiteln gehen.

\section{Separate}
\label{chp:coap-separate}

\section{Blockwise}
\label{chp:coap-blockwise}