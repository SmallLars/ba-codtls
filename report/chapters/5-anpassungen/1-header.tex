\section{Header}
\label{sec:headercompression}

Da die maximale Datenmenge eines Pakets im genutzen Datenübertragungsstandard IEEE 802.15.4 \cite{ieee802154} auf 127 Byte begrenzt ist würde der in DTLS
definierte Header mit 13 Byte schon mehr als 10\% des Datenvolumens ausmachen. Um das zu vermeiden wird die Stateless Header Compression aus dem Entwurf
von K. Hartke und O. Bergmann \cite[Kapitel 3]{draftcodtls} angewendet. Diese zeichnet sich durch eine verlustfreie Komprimierung aus, für die keine weiteren
Informationen bereitgestellt werden müssen. Damit lässt sich der Header im besten Fall auf 2 Byte, wie in Abbildung \ref{fig:com_handshake_header} dargestellt,
komprimieren.

\begin{figure}[ht]
  \centering
  \begin{lstlisting}[language=c]
   0                   1
   0 1 2 3 4 5 6 7 8 9 0 1 2 3 4 5
  +-+-+-+-+-+-+-+-+-+-+-+-+-+-+-+-+
  |0| T | V |  E  |1 1 0|  S  | L |
  +-+-+-+-+-+-+-+-+-+-+-+-+-+-+-+-+
  \end{lstlisting}
  \caption{Komprimierter Handshake-Header}
  \label{fig:com_handshake_header}
\end{figure}

Der RecordType (T) kann mit zwei Bit folgende 4 Zustände annehmen: \textit{8-Bit-Feld} (0), \textit{Alert} (1), \textit{Handshake} (2) und
\textit{Anwendungsdaten} (3). Trotz Realisierung des Handshakes über \acr{coap} ist diese Unterteilung notwendig, damit auch der DTLS-Layer über die
Art des Inhalts informiert ist und speziell die direkt für ihn bestimmten Daten bearbeiten kann. Hierzu gehören die Daten des Alert-Protokolls,
welche ohne \acr{coap} übertragen werden. Bei den Anwendungsdaten muss außerdem überprüft werden, dass diese nicht innerhalb der Epoche 0, ohne
Sicherheitsparamater, versendet oder empfangen werden. Auf direkte Angabe von \textit{ChangeCipherSpec} wurde verzichtet, da dies bei einem
Handshake über \acr{coap} nicht mehr notwendig ist (siehe Kapitel \ref{sec:handshake}). Sollten  weitere Unterptokolle notwendig sein, können
diese innerhalb eines 1 Byte langen Typenfeldes an den Header gehangen werden, was durch den Wert 0 signalisiert wird. In diesem zusätzlichen
Byte wird dann der im \acr{tls}/\acr{dtls} definierte Wert hinterlegt. So ist es auch möglich die 3 direkt definierten Werte unkomprimiert zu
versenden. Die komprimierten Werte wurden so angeordnet, dass durch Addition von 20 die in \acr{tls}/\acr{dtls} definierten Werte ermittelt werden können.

Die Version (V) kann mit zwei Bit folgende 4 Zustände annehmen von denen 3 Benutzt werden: \textit{\acr{dtls} 1.0} (0), \textit{16-Bit-Feld} (1) und
\textit{\acr{dtls} 1.2} (2). \acr{dtls} 1.0 und \acr{dtls} 1.2 können hier direkt definiert werden, da \acr{dtls} 1.0 weit verbreitet
 und \acr{dtls} 1.2 die aktuellste Version ist. Auf \acr{dtls} 1.1 wurde verzichtet, da Implementierungen die über \acr{dtls} 1.0 hinaus gehen im Allgemeinen auch
\acr{dtls} 1.2 unterstützen. Auch hier ist es möglich weitere Versionen an den Header anzuhängen, in dem V auf 1 gesetzt wird wobei hier mit 2 Byte das in \acr{tls}
definierte Versionsformat zum Einsatz kommt.

Die Epoche (E) kann mit den Werten 0 bis 4 kann direkt angegeben werden. Da jede Kommunikation mit der Epoche 0 beginnt, und
nach dem ersten Handshake in Epoche 1 fortgeführt wird, sind dies die am Häufigsten verwendeten Werte. Jeder weitere Handshake erhöht die Epoche um 1, so dass auch
weitere Epochen möglich sind ohne den Header zu vergrößern. Sollten höhere Werte benötigt werden, lässt sich das mit den folgenden Zuständen realisieren:
\textit{8-Bit-Feld} (5), \textit{16-Bit-Feld} (6) und \textit{Implizit} (7). So können 8 oder 16 Bit lange Epochen an den Header gehängt werden, was den
durch \acr{dtls} vorgebenen Bereich vollständig abdeckt. Alternativ kann durch den Wert 7 signalisiert werden, dass es sich bei der Epoche um die gleiche
handelt, wie bei dem vorausgehenden \acr{dtls}-Paket innerhalb des gleichen \acr{udp}-Pakets.

Für die Sequenz-Nummer (S) sind mit drei Bit 8 Zustände möglich. Während mit den Werten 1 bi 6 die Länge in Byte der angehängten Sequenz-Nummer angegeben wird,
kann durch den Wert 0 die Angabe unterbunden werden. Im Allgemeinen wird die Sequenz-Nummer in Verbindung mit der Epoche zur Berechnung des \acr{mac}s herangezogen.
Jedoch gibt es \glospl{ciphersuit} die andere Mechanismen verwenden, so dass keine Sequenz-Nummer notwendig ist. Falls mehrere \acr{dtls}-Pakete innerhalb eines
\acr{udp}-Pakets enthalten sind, kann die Sequenz-Nummer durch den Wert 7 auch relativ zum Vorgänger-Paket (+1) angegeben werden.

Schließlich folgt noch ein zwei Bit Wert für die Länge. Falls im \acr{udp}-Paket nur ein \acr{dtls}-Paket enthalten ist, kann hier der Wert 0 gesetzt werden, wodurch
keine Länge angegeben wird. Diese ist durch die Länge des \acr{udp}-Pakets implizit bekannt. Mit den Werten 1 und 2 kann die Länge in Byte der angehängten Länge
angegeben werden, während durch den Wert 3 das letzte \acr{dtls}-Paket im \acr{udp}-Paket gekennzeichnet wird, dessen Länge wieder implizit bekannt ist.