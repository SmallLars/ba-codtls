\chapter{TLS / DTLS}

Das Sicherheitsprotokoll \acr{tls} wird im Allgemeinen mit dem stromorientierten \acr{tcp} verwendet.
Wurde durch \acr{tcp} eine Verbindung hergestellt, können Daten von beliebieger Größer in jede Richtung
übertragen werden. \acr{tcp} wird dafür sorgen, dass der eingegebene Bytestrom vollständig und in der
richtigen Reihenfolge auf der Gegenseite wieder ausgegeben wird. Um die \acr{tls} bezogenen Daten nun
zu kennzeichnen und voneinander abzugrenzen existiert der "`Record Layer"' wie in Abbildung \ref{fig:recordlayer}
dargestellt. \TODO{Quellenangabe zu rfc ?} Dort ist neben der Art des Inhalts und der Protokollversion auch
die Länge enthalten, so dass aufeinanderfolgende Pakete im Datenstrom voneinander abgegrenzt werden können.

Da bei \acr{dtls} im Allgemeinen das paketorientierte \acr{udp} verwendet wird, bei dem die Länge eines
Paketinhalts bekannt ist, wirkt die Längenangabe zunächst überflüssig. Jedoch ist es insbesondere bei einem
\glos{handshake} sinnvoll, mehrere \acr{dtls}-Pakete innerhalb eines \acr{udp}-Pakets zusammenzufassen,
so dass auch hier wieder eine Längenangabe benötigt wird, um die Pakete voneinander abzugrenzen.
Zusätzlich sind bei \acr{dtls} nun die Datenfelder für die Epoche und die Sequenznummer hinzugekommen.
Wärend diese beiden Werte bei \acr{tls} durch die gewährleistete Reihenfolge der Daten durch \acr{tcp}
implizit bekannt sind, müssen diese bei \acr{dtls} explizit angegben werden, da \acr{udp} weder die
Reihenfolge noch den Transport der Daten garantiert. Die Epoche wird bei einem erfolgreichen \glos{handshake}
erhöht und ordnet so die dazugehörenden Daten den im \glos{handshake} ausgehandelten Sicherheitsparametern zu,
während die Sequenznummer in jeder Epoche bei 0 beginnt und bei jedem Paketversand erhöht wird.

\begin{figure}[ht]
  \centering
  \begin{lstlisting}[language=c]
    struct {
      ContentType type;
      ProtocolVersion version;
      uint16 epoch;                           // Nur bei DTLS
      uint48 sequence_number;                 // Nur bei DTLS
      uint16 length;
      opaque fragment[DTLS_Record.length];
    } DTLS_Record;
  \end{lstlisting}
  \caption{Record Layer von TLS / DTLS}
  \label{fig:recordlayer}
\end{figure}

\section{Handshake}

lala


