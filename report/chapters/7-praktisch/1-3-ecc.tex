\subsection{Contiki-App: "`ecc"'}

Für die Berechnung von elliptischen Kurven wurde im Bachelor-Projekt GOBI von Jens Trillmann ein C-Programm implementiert.
Dieses basiert auf einer Implementierung für einen 8-bit Mikrocontroller \cite{eccori}, wurde jedoch für 32-bit-Prozessoren optimiert.
Getestet und benutzt wird diese Implementierung bisher nur auf Desktop-Rechnern, wobei hier die Ausführung der Berechnungen in nicht
wahrnehmbarer Zeit erledigt wird.

Um die Berechnungen auch auf dem Redbee Econotag durchzuführen, wird die Implementierung in eine eigene Contiki-App übernommen.
Da der \glos{mc1322}-Mikrocontroller ebenfalls 32-bit-Berechnungen durchführt, ist dies zunächst direkt möglich. Im Gegensatz
zu Prozessoren in Desktop-Systemen arbeitet der Mikrocontroller jedoch mit einer wesentlich geringen Taktfrequenz, so dass sich
die benötigte Rechenzeit für eine Multiplikation auf elliptischen Kurven auf 13 Sekunden beläuft. In Zusammenarbeit mit Jens
Trillmann sind deshalb zunächst die drei Grundfunktionen "`Addition"', "`Subtraktion"' und "`Right-Shift"' für große Zahlen in
Assembler realisiert worden, um die Berechnung schneller zu machen. Weitere Optimierungen sollen in der Bachelorarbeit von Jens
Trillmann folgen.

Auf Basis des "`ARM GCC Inline Assembler Cookbook"' \cite{armasm} sind für die Addition 3 Varianten 


assembler code erwähnen der in zusammenarbeit mit jens enstanden ist\\
3 methoden und varianten kurz erläuterm\\
neugierig machen auf mehr -> .asm -> carry besser nutzen

auslagerung des base points und der Ordnung