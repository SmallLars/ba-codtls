\chapter{Praktische Umsetzung}

In den folgenden Abschnitten werden wichtige Merkmale der praktischen Umsetzung erläutert und einige Details erklärt,
ohne eine umfangreiche Dokumentation des Quellcodes zu erstellen. Die Dokumentation des Quellcodes erfolgt für öffentliche
Funktionen in den Header-Dateien im Stil von Doxygen \cite{doxygen}.

Das im vorigen Kapitel definierte \glos{ciphersuit} hängt grundlegend vom \acr{psk} des Endgeräts (Server) ab. Jeder, der diesen kennt, ist in
der Lage, während eines Handshakes einen \glos{mitma} durchzuführen oder Werte der \acr{prf} zu Berechnen, da diese auf dem \acr{psk} basiert.
Jedes Endgerät wird bei Herstellung mit einem eigenen \acr{psk} ausgerüstet. Dies wird durch ein Programm namens "`Blaster"' realisiert, das
im Bachelor-Projekt GOBI entstanden ist und für die Verwendung in dieser Arbeit angepasst wurde. Während in GOBI eine \acr{pin} generiert wurde,
die nach Erstellung einer sicheren Verbindung zur Authentifizierung des Besitzers des Endgeräts benutzt wurde, wird hier nun ein \acr{psk}
generiert. Blaster kommt zum Einsatz, nachdem der Quellcode des Endgeräts kompiliert wurde und erweitert die Binärdatei um Daten, die nach dem,
maximal \textasciitilde 96 KiB großen, Programmcode folgen. Diese, maximal 28 KiB, werden nicht mit in den RAM-Speicher kopiert und können zur
Ablage von Daten genutzt werden, die auch bei einem Batterie-Wechsel erhalten bleiben sollen. Neben dem \acr{psk} wird auch ein \acr{uuid}
generiert um das Endgerät eindeutig zu identifizieren. Da diese Daten für den Aufbau der \acr{dtls}-Verbindung genutzt werden, müssen diese
einem Endgerät beigelegt werden, was durch einen Aufkleber auf der Verpackung realisiert werden könnte. Um einem Benutzer das Einbinden neuer
Endgeräte möglichst einfach zu machen, wurde Blaster so erweitert, dass bei Ausführung auch ein QR-Code generiert wird. So kann der QR-Code
frühzeitig in einem \acr{dtls}-Clienten hinterlegt werden, so dass die Daten bei einem Verbindungsaufbau direkt verfügbar sind. Dieses System hat
den Nachteil, dass der \acr{psk} unter Umständen mindestens einem Vorbesitzer des Endgeräts bekannt ist. Dieser soll aber nach Veräußerung eines
Endgeräts keinen Zugriff mehr darauf bekommen. Um dem Vorzubeugen ist der dem Endgerät beiliegende \acr{psk} nur für einen Verbindungsaufbau
gültigt. Ist dieser erfolgreich abgeschlossen, wird automatisch ein neuer \acr{psk} generiert und bei einem weiteren Verbindungsaufbau benutzt.
Möchte der Besitzer eine weitere Verbindung zum Endgerät aufbauen, kann er den neuen \acr{psk} über die vorhandene sichere Verbindung abrufen
und nutzen, wobei dann wieder ein neuer \acr{psk} generiert wird. Um ein Endgerät zu veräußern, kann ein Reset-Knopf gedrückt werden, welcher das
Endgerät auf den Werkszustand zurücksetzt und so den ursprünglichen \acr{psk} wieder aktiviert.

\section{Server}

Der Server wird auf einem Redbee Econotag \cite{econotag} realisiert. Der darauf enthaltene Mikrocontroller \glos{mc1322} \cite{mc1322} enthält,
neben dem IEEE 802.15.4 Funkstandard und einer \acr{aes} Hardware-Engine, 128 KiB Flash-Speicher und 96 KiB RAM-Speicher. Bei Inbetriebnahme
wird das im Flash-Speicher vorliegende Programm vollständig in den RAM-Speicher kopiert und dort ausgeführt, wodurch sich eine maximale
Programmgröße von 96 KiB ergibt. Die zusätzlichen 32 KiB Flash-Speicher können somit für die Ablage von Daten genutzt werden, die auch nach einer
Stromunterbrechung, oder einem Neustart des Geräts, erhalten bleiben sollen. Zu berücksichtigen ist jedoch auch noch, dass der letzte 4 KiB große
Block schon für den Redbee Econotag selbst reserviert ist.

Betrieben wird der Server mit SmartAppContiki \cite{erbium}, das auf Contiki \cite{contiki} basiert, und eine Implementierung von \acr{coap}, in
der Entwurfsversion 13 \cite{draftcoap13}, enthält. In der Standardkonfiguration benötigt SmartAppContiki, mit einer definierten \acr{coap}-Ressource,
die ein "`Hallo Welt!"' zurückgibt, \textasciitilde 84 KiB. Diese Daten teilen sich gemäß Abbildung \ref{tbl:contiki-speicher} auf. Um den benötigten
Speicher zu optimieren wurde die Größe des "`Sys Stack"' und des "`Heap"' in der Konfigurationsdatei "`contiki/cpu/mc1322x/mc1322x.lds"' angepasst.

\begin{figure}[!ht]
\centering
\begin{tabular}{l|r|r}
  \hiderowcolors
  \textbf{Beschreibung} & \textbf{Standard} & \textbf{Angepasst}\\
  \hline
  Programm        & ~~~~~59248 Byte    & ~~~~~59248 Byte\\
  Irq Stack       &   256 Byte         &   256 Byte\\
  Fiq Stack       &   256 Byte         &   256 Byte\\
  Svc Stack       &   256 Byte         &   256 Byte\\
  Abt Stack       &    16 Byte         &    16 Byte\\
  Und Stack       &    16 Byte         &    16 Byte\\
  Sys Stack       &  1024 Byte         &  2048 Byte\\
  Datensegment    & 20872 Byte         & 20872 Byte\\
  Heap            &  4096 Byte         &    16 Byte\\
  \hline
  \textbf{Gesamt} & 86040 Byte         & 82984 Byte\\
                  & $ \approx $ 84 KiB & $ \approx $ 81 KiB\\
  \showrowcolors
\end{tabular}
\caption{Speicheraufteilung von SmartAppContiki}
\label{tbl:contiki-speicher}
\end{figure}

Das wurde möglich durch Verwendung der in Contiki eingebauten Beobachtungswerkzeuge. Durch Definieren von periodischen Ausgaben der benutzen Heap
sowie Sys Stack Größe, in "`contiki/platform/redbee-econotag/contiki-mc1322x-main.c"', können die Auslastungen beobachtet werden. Um diesen Prozess
effizienter zu gestalten, wird nur die Initialisierung durchgeführt und die periodischen Ausgaben deaktiviert. In "`server/server.c"' lässt sich nun,
durch Aktivieren des Debug-Modus, ein Code einbinden, der auf Knopfdruck sowohl die Speicheraufteilung als auch die bisher genutzten Bytes des Sys Stack
und Heap ausgibt. Dadurch lässt sich erkennen, dass der Heap garnicht benutzt wird, und somit unnötig Speicher belegt. Da insbesondere während des Handshakes,
unter anderem aufgrund der Berechnung von elliptischen Kurven, viele Daten zwischengespeichert werden müssen, wird ersichtlich, dass ein Sys Stack von 1024
Byte nicht ausreicht, eine Größe von 2048 Byte jedoch optimal ist. Durch diese Anpassungen wurde der, für SmartAppContiki benötigte, Speicher von \textasciitilde
84 KiB auf \textasciitilde 81 KiB reduziert (siehe Abbildung \ref{tbl:contiki-speicher}). Somit stehen für die Umsetzung von \acr{dtls} \textasciitilde 15 KiB zur
Verfügung, wobei auch berücksichtig werden muss, dass noch die Funktionen des Geräts selbst implementiert werden müssen.

Bei der Benutzung der, in SmartAppContiki enthaltenen, \acr{coap} 13 Implementierung, hat sich herausgestellt, dass die Unterstützung für die \acr{coap}-Option
Block-1 fehlt. Diese Option kann von einem Clienten benutzt werden, um größere Datenmengen, in einer \acr{coap}-Anfrage, in Blöcke zu unterteilen, damit es nicht
zu einer Fragmentierung auf IP-Ebene kommt. Laut Aussage von Matthias Kovatsch wird "`auf die atomare Variante aus Platzgründen verzichtet, da man Block1 ganz
einfach im Resource-Handler lösen kann"`. Da diese Option für den \acr{dtls}-Handshake generell benötigt wird, und ohne Code-Duplizierung auch anderen Ressourcen
zur Verfügung stehen soll, wird sie in ähnlicher Form wie die Separate-Option implementiert. Das Separate-Modul bietet Methoden an, um den Client, während der
Bearbeitung einer Anfrage, zu informieren, dass die Bearbeitung einige Zeit dauert, und die Beantwortung der Anfrage später fortzusetzen. In diesem Sinne bietet
das Block-1-Modul eine Methode an, mit der die Parameter der Block-1-Option überprüft und bearbeitet werden, wobei bei Bedarf die entsprechenden Fehler generiert
werden, oder die erhaltenen Daten auf Wunsch zusammengesetzt werden. Genau wie das Separate-Modul kann das Block1-Modul optional in einer Ressource benutzt werden.
Dabei ist nach wie vor ein Empfang von Daten ohne Block-1-Option möglich. Durch den Rückgabewert der Methode, lässt sich in der Ressource entscheiden, ob schon
die einzelnen Datenblöcke bearbeitet werden, oder erst die vollständige Nachricht, nach Erhalt aller Blöcke. Damit das Separate-Modul auch in Kombination mit dem
Block-1-Modul genutzt werden kann, wurde das Separate-Modul entsprechend angepasst um die Block-1-Option zu berücksichtigen.

Während bisher allgemeine Anpassungen von SmartAppContiki bzw. dem darin enthaltenen \acr{coap} 13 beschrieben wurden, folgt in den nächsten vier Abschnitten
die Erläuterung von vier implementierten Contiki-Apps, welche für die Realisierung von \acr{dtls} benutzt werden. Die Implementierung von \acr{dtls} wird dann
im 5. Abschnitt erläutert, wonach abschließend noch eine Update-Funktion erläutert wird, die für \acr{dtls} nicht notwendig ist, aber dessen Umfeld berücksichtigt.

\subsection{Contiki-App: "`flash-store"'}

Für eine Nutzung des erweiterten Flash-Speichers wird die App "`flash-store"' verwendet. Als Basis für die Implementierung, dient Code aus dem Bachelorprojekt
\glos{gobi}. Dieser war jedoch noch nicht als Contiki-App organisiert sondern direkt mit in den Code eingebunden. Auch ist die Aufteilung der 4 KiB großen
Flash-Speicher-Blöcke eine andere. Diese Aufteilung ist in Abbildung \ref{tbl:2-1_2-persistent} zu sehen. Während oben die 8 Speicherblöcke mit ihren Adressen
aufgeführt sind, werden darunter die Aufteilungen für unterschiedliche Zwecke angegeben, wobei dort sowohl die \glos{gobi}-Aufteilung als auch die neue
Aufteilung aufgeführt sind.

\begin{figure}[!ht]
\centering
\renewcommand{\arraystretch}{1.5}
\begin{tabular}{|p{1.4cm}|p{1.4cm}|p{1.4cm}|p{1.4cm}|p{1.4cm}|p{1.4cm}|p{1.4cm}|p{1.4cm}|}
  \hiderowcolors
  \hline
  $ 0x18000\newline -\newline 0x18 $FFF & $0x19000\newline -\newline 0x19 $FFF & $ 0x1 $A$ 000\newline -\newline 0x1 $AFFF & $ 0x1 $B$ 000\newline -\newline 0x1 $BFFF & $ 0x1 $C$ 000\newline -\newline 0x1 $CFFF & $ 0x1 $D$ 000\newline -\newline 0x1 $DFFF & $ 0x1 $E$ 000\newline -\newline 0x1 $EFFF & $ 0x1 $F$ 000\newline -\newline 0x1 $FFFF\\
  \hline
  \multicolumn{8}{l}{~}\\
  \multicolumn{8}{l}{Aufteilung innerhalb des Bachelorprojekts \glos{gobi}:}\\
  \hline
  RO 1 & \multicolumn{2}{l|}{RW 1} & \multicolumn{2}{l|}{RW 2} & \multicolumn{2}{l|}{RO 2} & SR\\
  \hline
  \multicolumn{1}{c|}{~} & \multicolumn{2}{c|}{$ 0x0000 - 0x0 $FFF} & \multicolumn{2}{c|}{$ 0x1000 - 0x1 $FFF} & \multicolumn{3}{c}{$ \leftarrow $ virtuelle Speicheradressen} \\
  \cline{2-5}
  \multicolumn{8}{l}{~}\\
  \multicolumn{8}{l}{Neue Aufteilung für \acr{dtls}:}\\
  \hline
  \multicolumn{2}{|l|}{RW 1} & \multicolumn{2}{l|}{RW 2} & RAD & \multicolumn{2}{l|}{RO} & SR\\
  \hline
  \multicolumn{2}{|c|}{$ 0x0000 - 0x0 $FFF} & \multicolumn{2}{c|}{$ 0x1000 - 0x1 $FFF} & \multicolumn{1}{c}{~} & \multicolumn{3}{c}{$ \leftarrow $ virtuelle Speicheradressen} \\
  \cline{1-4}
  \multicolumn{8}{l}{~}\\
  \multicolumn{8}{l}{Legende: RW = Read-Write, RAD = Read-Append-Delet, RO = Read-Only, SR = System-Reserved}\\
  \showrowcolors
\end{tabular}
\renewcommand{\arraystretch}{1.0}
\caption{Aufteilung des erweiterten Flash-Speichers}
\label{tbl:2-1_2-persistent}
\end{figure}

Geändert wurde zunächst die Position der beiden RW-Blöcke. Diese ermöglichen das Schreiben von Daten, ohne die Eigenschaften des Flash-Speichers berücksichtigen
zu müssen. Dieser kann nur beschrieben werden, falls die betroffene Position vorher einmal gelöscht wurde, was sich aber nur in Blöcken a 4 KiB realisieren lässt.
Um Datenverluste zu vermeiden, werden jeweils zwei 4 KiB große Blöcke benutzt, um einen 4 KiB großen Speicher zu realisieren, der sich durch virtuelle Adressen
ansprechen lässt, welche ebenfalls in Abbildung \ref{tbl:2-1_2-persistent} aufgeführt sind. Die Daten sind immer nur in einem Block gespeichert, während der andere
Block gelöscht ist. Kommt es zu einem Schreibvorgang, wird der Datenblock in den leeren Block kopiert, wobei die gewünschten Änderungen realisiert werden.
Die Position der beiden RW-Blöcke befindet sich nun am Anfang, da sich so die Adressen, der jeweils zusammengehörenden Blöcke, genau um ein Bit unterscheiden.
Das vereinfacht die Berechnung der Quell- und Ziel-Adresse erheblich, so dass durch Optimierung des Quellcodes 70 Byte an Programmgröße eingespart werden.

Problemtisch ist jedoch die Dauer und der Energieverbrauch bei einem Schreibzugriff dieser Art. Um eine effizientere Ablage von Daten zu ermöglichen folgt
nach den beiden RW-Blöcken anstatt des RO-Blocks nun ein RAD-Block. Dieser ist vergleichbar mit einem Stack ohne Push- und Pop-Funktion. Für die Initialisierung
wird der komplette Block gelöscht. Daten können nun so lange eingefügt werden, bis der Block voll ist. Wieviel Daten gerade enthalten sind, wird dabei in einer
globalen Variablen im RAM-Speicher gespeichert. Der Lesezugriff kann dabei beliebig erfolgen.

Gleich geblieben ist die Position des RO-Blocks. Dort können im Vorfeld, durch das bereits erwähnte Programm "`Blaster"', Daten abgelegt werden, welche zur Laufzeit
ausgelesen werden können. Dies spart Programmgröße, da diese "`Konstanten"' nicht im Datensegment des Programms enthalten sind.

Abschließend folgt noch ein Block der für den Redbee Econotag selbst reserviert ist, und somit nicht genutzt werden kann.
\subsection{Contiki-App: "`time"'}

Im Gegensatz zu herkömmlichen Desktop-Rechnern oder Servern verfügt der Redbee Econotag über keine innere Uhr bezüglich der Realzeit.
Angeboten wird vom \glos{mc1322} das Register "`MACA\_CLK"' welches mit einem Takt von 250 KHz erhöht wird. Dieses läuft, bedingt durch
seine Breite von 32 Bit, jedoch alle 4,77 Stunden über, so dass ohne weitere Eingriffe keine direkte Berechnung der Zeit möglich ist.
Ähnlich verhält es sich mit dem Register "`CRM\_RTC\_COUNT"' welches im Takt von "`CRM\_RTC\_TIMEOUT"' Hz erhöht wird.
Dieser Wert wird von Contiki eingestellt und liegt bei \textasciitilde 20 KHz. Dadurch erfolgt hier ein Überlauf nach ungefähr 60 Stunden.
Neben dem Überlauf haben beide Quellen das Problem, dass die Register bei Einschalten des Econotags bei 0 anfangen, und die Werte somit
keinerlei Bezug zur Realzeit haben.

Contiki löst einen Teil der genannten Probleme und stellt die Funktion "`clock\_seconds"' zur Verfügung. Diese kümmert sich um den Überlauf
und berechnet laufend die, seit dem Einschalten des Econotags, vergangenen Sekunden. Diese werden in einer 32 Bit Variablen gespeichert,
wodurch ein Überlauf erst nach \textasciitilde 136 Jahren vorkommen kann.

Um den Bezug zur Realzeit herzustellen, wird die aktuelle Unixzeit vom Blaster generiert und im RO-Teil des Flash-Speichers hinterlegt.
Diese Zeit spiegelt somit den Herstellungszeitpunkt wieder. Wird die aktuelle Zeit benötigt, kann diese durch Addition der von Contiki
ermittelten Sekunden und der Unixzeit berechnet werden. Das funktioniert natürlich nur so lange, wie der Redbee Econotag nach dem Flashen
ununterbrochen mit Strom versorgt wird. Um eine Korrektur zu ermöglichen, bietet die App eine Methode an, um die aktuelle Uhrzeit zu setzen.
Diese wird mit der alten Zeit vergleichen um einen Korrekturwert zu ermitteln der in einer globalen Variablen hinterlegt wird. Hier macht es
keinen Sinn diesen im Flash-Speicher abzulegen, da er nach einem Batteriewechsel direkt wieder veraltert wäre.
\subsection{Contiki-App: "`aes"'}

Um die \acr{aes}-Funktionen des \glos{mc1322} \cite{mc1322} zu nutzen, dient diese Contiki-App. Die bereits beschriebene \glos{ciphersuite}
verwendet in der \acr{prf} \acr{aes}-\acr{cmac} \cite{rfc4493}. Außerdem wird für die Verschlüsselung \acr{aes}-\acr{ccm} \cite{rfc3610} verwendet. Beide Verfahren
werden vom \glos{mc1322} nicht direkt unterstützt. Bereitgestellt wird nur der reine \acr{aes}-Verschlüsselungsprozess im \acr{ctr}- und \acr{cbc}-Mode.

Damit die \acr{aes}-Hardware genutzt werden kann, muss diese zunächst initialisiert werden. Die dafür notwendige Methode wurde zum Großteil aus dem Bachelorprojekt \glos{gobi}
übernommen und leicht modifiziert. Neben der Aktivierung der \acr{aes}-Hardware wir dort ein Selbsttest mit einem internen Schlüssel durchgeführt. Ist dieser
erfolgreich, werden die beiden Modi \acr{ctr} und \acr{cbc}, für die spätere Nutzung, aktiviert.

Die \acr{aes}-Berechnungen selbst werden durch Übertragen der notwendigen Daten in Register des Mikrocontrollers durchgeführt. Diese sind auf Speicheradressen
abgebildet, die wiederum in Konstanten in der Bibliothek des Mikrocontrollers hinterlegt sind. Da es sich um eine 128-bit-Verschlüsselung handelt, der Mikrocontroller
aber nur in 32 Bit arbeitet, sind für jeden Wert vier Register notwendig. Diese sind jeweils von 0 bis 3 durchnummeriert, was in der folgenden Beschreibung durch <X>
dargestellt wird. Zusätzlich sind zwei Register notwendig, um den Verschlüsselungsprozess zu starten und auswerten zu können.

\begin{description}
  \item[KEY<X>] Schlüssel zur Ver- und Entschlüsselung. Verbleibt so lange im Register, bis das \acr{aes}-Modul zurückgesetzt, oder das Register überschrieben wird.
  \item[DATA<X>] Klar- oder Geheimtext.
  \item[CTR<X>] Zähler für den \acr{ctr}-Mode. Dieser wird nicht automatisch erhöht und muss somit vor jeder Berechnung gesetzt werden.
  \item[CTR<X>\_RESULT] Ergebnis der \acr{ctr}-Berechnung. Dafür wurde der hinterlegte Zähler verschlüsselt und durch die XOR-Funktion mit dem Datenpaket verknüpft. Dieses Register kann nur gelesen werden.
  \item[CBC<X>\_RESULT] Ergebnis der \acr{cbc}-Berechnung. Dafür wurde das Datenpaket durch die XOR-Funktion mit dem Ergebnis der letzten Verschlüsselung verknüpft und dann verschlüsselt. Dieses Register kann nur gelesen werden.
  \item[MAC<X>] Kann mit einem Initialisierungsvektor belegt werden, der für die \acr{cbc}-Berechnung herangezogen wird.
  \item[CONTROL$0$bits] Enthält u. a. ein Bit, durch das die Verwendung des Initilalisierungsvektors gekennzeichnet wird. Außerdem ist ein Bit dafür vorgesehen, den Ver- und Entschlüsselungsprozess zu starten.
  \item[STATUSbits]  Enthät u. a. ein Bit, das kennzeichnet, ob die aktuelle Berechnung abgeschlossen wurde.
\end{description}

Da der Ent- und Verschlüsselungsprozess derselbe ist, reicht eine Methode für die Umsetzung von \acr{aes}-\acr{ccm} aus. Diese arbeitet in-place, was bedeutet,
dass die Daten direkt an ihrer Position im Speicher konvertiert werden, und zusätzlich nur eine konstante, von der Datenmenge unabhängige, Menge Speicher benötigt
wird. Während für die Verschlüsselung ein Aufruf der Methode ausreicht, da die Daten verschlüsselt werden und der \acr{mac} berechnet wird, sind für die Entschlüsselung
zwei Aufrufe notwendig. Zunächst werden die Daten entschlüsselt, wobei automatisch ein neuer \acr{mac}, auf Basis des Geheimtextes, generiert wird. Dieser hat jedoch
keinen Nutzen. Im 2. Schritt wird die Funktion erneut aufgerufen, um ausschließlich den \acr{mac} zu generieren, damit dieser mit dem Erhaltenen verglichen werden kann.
Das führt im 1. Schritt zwar zu unnötigen Berechnungen, verlangsamt aber den Prozess nicht, da Entschlüsselung und \acr{mac}-Berechnung parallel in der Hardware
durchgeführt werden. Der Vorteil liegt hier in der geringen Programmgröße.

Die Methode zur Berechnung der \acr{cmac} ist so gestaltet, dass die Berechnung in mehreren Schritten erfolgen kann, solange die Länge der übergebenen Daten ein
Vielfaches von 16 Byte (128 Bit) beträgt. Erst durch Setzen des letzten Parameters, welcher den Abschluss signalisiert, ist die Datenlänge beliebig, und die
Berechnung wird gemäß \acr{cmac}-Vorgabe abgeschlossen.
\subsection{Contiki-App: "`ecc"'}

Für die Berechnung von elliptischen Kurven wurde im Bachelorprojekt \glos{gobi} von Jens Trillmann ein C-Programm implementiert.
Dieses basiert auf einer Implementierung für einen 8-bit Mikrocontroller \cite{eccori}, wurde jedoch für 32-bit-Prozessoren optimiert.
Getestet und benutzt wird diese Implementierung bisher nur auf Desktop-Rechnern, wobei hier die Ausführung der Berechnungen in nicht
wahrnehmbarer Zeit erledigt wird.

Um die Berechnungen auch auf dem Redbee Econotag durchzuführen, wird die Implementierung in eine eigene Contiki-App übernommen.
Da der \glos{mc1322}-Mikrocontroller ebenfalls 32-bit-Berechnungen durchführt, ist dies zunächst direkt möglich. Im Gegensatz
zu Prozessoren in Desktop-Systemen arbeitet der Mikrocontroller jedoch mit einer wesentlich geringen Taktfrequenz, so dass sich
die benötigte Rechenzeit für eine Multiplikation auf elliptischen Kurven auf 13 Sekunden beläuft. In Zusammenarbeit mit Jens
Trillmann sind deshalb zunächst die drei Grundfunktionen "`Addition"', "`Subtraktion"' und "`Right-Shift"' für große Zahlen in
Assembler realisiert worden, um die Berechnung schneller zu machen. Weitere Optimierungen sollen in der Bachelorarbeit von Jens
Trillmann folgen.

Auf Basis des "`ARM GCC Inline Assembler Cookbook"' \cite{armasm} sind für die drei Grundfunktionen einige Varianten entstanden.
Welche davon jeweils genutzt wird, lässt sich in den einzelnen Quellcode-Dateien einstellen. Generell bietet sich eine Umsetzung
in Assembler an, da sich das sogenannte "`Carry-Bit"' nutzen lässt. In diesem wird bei einer Rechenoperation ein möglicher Überlauf
gespeichert. Für die gängigen Rechenoperationen gibt es zwei unterschiedliche Befehle, wobei nur bei einem das Carry-Bit genutzt wird.
Dieses Potenzial zu nutzen, hat sich jedoch als schwierig herausgestellt, da Contiki das Thumb-Instruktion-Set des \glos{mc1322} nutzt.
Im Gegensatz zum ARM-Instruktion-Set, dass 32-bit-Operationen nutzt, sind es im Thumb-Instruktion-Set nur 16 Bit. Jede Thumb-Instruktion
wird bei Ausführung automatisch in die entsprechende ARM-Instruktion umgewandelt und ausgeführt. Durch die begrenzte Größe stehen jedoch
nicht alle ARM-Instruktionen zur Verfügung und die Anzahl der nutzbaren Register ist auf 8 reduziert. Der Vorteil liegt jedoch in der
geringen Programmgröße, so dass Contiki überhaupt erst auf dem \glos{mc1322} betrieben werden kann.

Alle drei Grundfunktionen sind zunächst ohne Einschränkung, der auch in C implementierten Funktionalität, umgesetzt. Insbesondere sind somit
die Längen der Ein- und Ausgabewerte variabel, was sich nur mit einer Schleife realisieren lässt. Eine Schleife bedeutet jedoch auch, dass
ein Zähler erhöht und verglichen werden muss. Da der Block mit dem Carry-Bit im Thumb-Instruktion-Set durch alle Operationen aktualisiert wird,
geht das Carry-Bit der Hauptoperation vom einen zum nächsten Schleifendurchlauf verloren, muss manuell zwischengespeichert, und bei Bedarf
berücksichtigt werden. Ein Sichern und Wiederherstellen des Blocks mit dem Carry-Bit ist nur im ARM-Instruktion-Set möglich.
Die Optimierung besteht bei dieser Umsetzung somit nur darin, dass es einfach möglich ist, einen Überlauf zu erkennen.
Während dies bei der Addition und Subtraktion 24 und 32 Byte Programmgröße einspart, bringt es bei Right-Shift
keinen Größenvorteil. Jedoch ist die Berechnung aufgrund der eingesparten Vergleiche bei allen Operationen schneller.

Da im Thumb-Instruktion-Set für einen Right-Shift keine Funktion zur Verfügung steht, die das Carry-Bit direkt benutzt, ist hier keine weitere Optimierung möglich.
Für die Addition und Subtraktion sind weitere Varianten verfügbar. Da die Subtraktion ausschließlich für die Berechnung von 256-bit-Zahlen benutzt wird,
was acht 32-bit-Blöcken entspricht, ist es möglich, die acht Subtraktionen direkt hintereinander auszuführen, so dass das Carry-Bit ohne weitere
Eingriffe direkt berücksichtigt wird. Die Programmgröße nimmt dabei, im Vergleich zum C-Code, um \textasciitilde 32 Byte ab und die
Berechnungsgeschwindigkeit nimmt wesentlich zu. Anders verhält es sich bei der Addition, da Werte unterschiedlicher Größe addiert werden.
Notwendig sind hier 128, 256 und 512 Bit. Für jede dieser Größen ist nun ein eigener Additionsblock vorhanden. Bei Aufruf der Funktion wird
die Größe überprüft und der richtige Block ausgewählt. Dies bietet eine maximale Geschwindigkeit, erhöht jedoch die Größe des Programms um 88 Byte.

Um weitere 96 Byte einzusparen, sind 3 benötigte Konstanten im Flash-Speicher hinterlegt. Dazu gehören die X- und Y- Koordinate des Basis-Punkts, der für
den Diffie-Hellman-Schlüsselaustausch verwendet wird, und die Ordnung der verwendeten elliptischen Kurve. Die Ordnung wird nur verwendet, um zu überprüfen,
ob der zufällig generierte private Schlüssel sich für die Benutzung eignet. Bei Bedarf werden die Werte aus dem Flash-Speicher geladen und nur so lange
im Stack abgelegt, wie sie benötigt werden.

Weitere Optimierungen bezüglich der Programmgröße und Berechnungsgeschwindigkeit sollen in der Bachelorarbeit von Jens Trillmann folgen.
\subsection{Contiki-App: "`er-13-dtls"'}

Um diese Contiki-App zu realisieren, sind zunächst einige Anpassungen in er-coap-13 notwendig. Um nach wie vor einen Betrieb ohne \acr{dtls} realsisieren zu können,
werden diese Anpassungen nur dann aktiv, wenn die Compiler-Anweisung WITH\_DTLS gesetzt ist, was im Makefile für ein Contiki-Programm durch "`CFLAGS += -DWITH\_DTLS=1"'
realisiert werden kann. Bei der Verwendung von \acr{dtls} wird der Port gemäß \cite[Seite 93]{portnumbers} von 5683 auf 5684 geändert. Außerdem wird der Datenverkehr
über \acr{dtls}-Funktionen geleitet, die den Record-Layer realisieren und die Daten bei Bedarf ent- oder verschlüsseln. Kernstück ist die Ressource "`/dtls"', die für
die Durchführung des Handshakes in CoAP eingebunden wird.

Für die Realisierung des Recory-Layers und der \acr{dtls}-Ressource sind einige Module notwendig, die im Folgenden zunächst beschrieben werden.

\begin{description}
  \item[er-dtls-13-alert.{$[$h|c$]$}] Stellt zwei Funktionen für den Versand einer Alert-Nachricht zur Verfügung.
					Während die eine, ohne Verwendung con CoAP, eine Nachricht an den Kommunikationspartner sendet,
					konfiguriert die andere eine CoAP-Antwort. Dieser Unterschied ist notwendig, da einige Fehler während
					des Handshakes auftauchen, und per CoAP kommentiert werden, damit es nicht zu einer neuen Anfrage kommt.
					Würden diese Nachrichten direkt mit dem Record-Protokoll versand, müsste sich der Empfänger darum kümmern,
					die Anfrage aus dem CoAP-Layer zu entfernen.
  \item[er-dtls-13-data.{$[$h|c$]$}] Erstellt und verwaltet sessionspezifische Daten. Die Anzahl der Sessions ist hier auf maximal zehn begrenzt.
					Damit die dafür notwendigen 1400 Byte nicht dauerhaft den RAM-Speicher belegen, sind diese im Flash-Speicher
					abgelegt. Die Ausnahme bilden hier die beiden Werte für die Sequenznummern zum Lesen und Schreiben.
					Während letztere bei jedem Datenaustausch geändert werden, werden die grundlegenden Session-Daten nur während
					des Handshakes geschrieben, so dass hier die Nutzung des Flash-Speichers sinnvoll ist. Dafür wird der RW-Block
					verwendet, der durch die Contiki-App flash-store zur Verfügung gestellt wird. Die Sessions sind dort in einem
					Array hinterlegt, wobei die genutzten Stellen entsprechend gekennzeichnet sind. Dadurch werden bei der Suche
					nach Sessions unter Umständen auch leere Stellen durchlaufen, jedoch bleibt der Aufwand erspart die Liste zu
					defragmentieren oder Zeiger einer verketteten Liste zu aktualisieren. Da pro Session zwei Keyblöcke benötigt werden,
					falls es zu einem erneuten Handshake kommt, sind diese in einem separaten Array mit der Länge 20 hinterlegt.
					Der Index der Keyblöcke ergibt sich dabei aus dem Index der Session. Liegt diese im ersten Array an Index $ i $,
					Sind die dazu gehörenden Keyblöcke an Index $ i\cdot2 $ und $ i\cdot2+1 $ hinterlegt. Dabei liegt der derzeit gültige
					Keyblock immer an Index $ i\cdot2 $. Bei einer Weiterentwicklung der Epoche, wird der 2. Keyblock an den Index
					des ersten kopiert. Da die Sequenznummern im RAM-Speicher abgelegt sind, gehen diese bei einem Batteriewechsel,
					oder Neustart des Endgeräts, verloren, womit die Session nicht fortgesetzt werden kann. Da es keine sinnvolle Lösung gibt,
					diese ohne Sicherheitslücken wiederherzustellen, wird bei einem Start des Endgeräts der RW-Block des Flash-Speichers
					zurückgesetzt, wodurch alle Sessiondaten gelöscht werden. Bei einer folgenden Anfrage wird der Client zunächst eine
					Alert-Nachricht bekommen, was aber dann zu einem neuen Handshake führt.
  \item[er-dtls-13-prf.{$[$h|c$]$}] Enthält die im \glos{ciphersuite} definierte \acr{prf}. Die größte Datenmenge wird für die Berechnung des Master-Secrets
					benötigt. Dort gehen 153 Byte in die Berechnung ein. Dieses ist ausreichend klein um die Berechnung mit einem
					Funktionsaufruf durchzuführen, womit der Programmcode klein gehalten wird, da kein Zustand für eine Fortsetzung
					der Berechnung gespeichert und genutzt werden muss.
  \item[er-dtls-13-psk.{$[$h|c$]$}] Verwaltet den \acr{psk} und generiert bei Bedarf einen neuen. Der \acr{psk} wird im Vorfeld durch das Programm Blaster
					generiert und im Flash-Speicher abgelegt. Soll ein neuer \acr{psk} generiert werden, um den Werks-\acr{psk} zu deaktivieren,
					wird ein Byte im RW-Block des Flash-Speichers gesetzt und dort ebenfalls ein neuer \acr{psk} hinterelgt. Wird der \acr{psk}
					abgerufen, kann anhand des gesetzten Bytes erkannt werden, ob der Werks-\acr{psk} gilt oder ein neuer Verfügbar ist.
					Kommt es zu einem Batteriewechsel oder Neustart des Endgeräts, ist der Werks-\acr{psk} wieder gültigt, da der RW-Block
					zurückgesetzt wurde.
  \item[er-dtls-13-random.{$[$h|c$]$}] Bietet Funktionen für die Generierung von Zufallszahlen an. Dieses Modul benutzt das MACA\_RANDOM Register des \glos{mc1322},
					das durch Contiki bereits initialisiert wurde. Durch Auslesen des Registers können beliebig viele Zufallswerte erzeugt werden.
\end{description}

es kann nur ein handshake zur zeit stattfinden\\
psk wird nur benötigt bis pre master secret berechnet\\
am ende des handshakes wird alter psk vernichtet und neuer generiert\\
kann durch uri /psk über sichere leitung abgerufen werden so dass weitere handshakes durchgeführt werden können

für hello verify cookie: cmac(client-ip + clienthello)

verschlüsselung von finish mit seq\_num 0. erste app daten beginnen also bei seq\_num 1.\\
seq\_num wird im ram gehalten da bei flash update bei jedem paket hoher verschleiß.\\

uri syntax \cite{rfc3986}. beachten wegen session-id

Übertragen werden mit Hilfe der Option ein ClientHello oder ClientKeyExchange + ChangeCipherSpec + Finished, wobei die Länge von Letzterem mit
insgesamt 114 (87 + 3 + 22) Byte, durch die einzig definierte \glos{ciphersuite}, fest ist. Anders ist dies bei dem ClientHello, welches durch
eine Vielzahl, vom Client beherrschter \glospl{ciphersuite}, sehr viel größer werden kann.

Hier erfolgt dann jedoch eine \acr{coap}-Fehlermeldung "`4.13 REQUEST ENTITY TOO LARGE"' mit einem Hinweis auf die begrenzte Maximalgröße,
so dass ein Client sein Angebot an \glospl{ciphersuite} reduzieren kann, um der maximalen Größe gerecht zu werden.

auswertung cookie -> query. cookie in packet ist doof

problem: handshake: uri prüfen !
\subsection{Update-Funktion}

flasher beschreiben: überschreibt den programmcode jedoch nicht die vom blaster generierten gerätedaten
\section{Client}

Als Grundlage für den Clienten wird die Implementierung aus dem Bachelorprojekt GOBI übernommen. Diese ist in der Lage, die im Sensornetz eingebundenen
Endgeräte vom Border-Router abzurufen, und in einer Liste darzustellen. Über den Listen-Index ist es dann möglich, mit den Endgeräten zu kommunizieren,
so dass nicht jedesmal die IP-Adresse angegeben werden muss.

Während der Client bisher direkt über \acr{udp} mit den Endgeräten kommuniziert hat, wird nun die libcoap \cite{libcoap} genutzt, um eine Kommunikation
über \acr{coap} zu realisieren. Da es in dieser Arbeit primär um die Realisierung von \acr{dtls} gehen soll, wird der \acr{coap}-Client, aus den in der
Bibliothek enthaltenen Beispielen, übernommen, und für die Verwendung engepasst. Anpassungen sind hier notwendig da der \acr{coap}-Client ein
Kommandozeilen-Tool ist. Die enthaltenen main()-Methode wird dafür umbenannt, und die Ausgabe der Antwort erfolgt in den vom Aufrufer übergebenen Speicher.
Außerdem wird die fehlende Funktionalitat ergänzt, Antworten mit einer Block-2-Option zu empfangen, die auf eine Separate-Antwort oder eine Block-1-Anfrage
folgen. Hier übernimmt der Server die Kontrolle über die Datenübertragung, während der Client Empfangsbestätigungen sendet. Eine Block-1-Anfrage wird
nicht durch den \acr{coap}-Clienten selbst realisiert, sondern muss manuell, durch mehrere Anfragen, umgesetzt werden. Dieser erweiterte \acr{coap}-Client
ist nicht für die Verwendung in anderen Projekten gedacht, und soll hier nur als provisorisches Werkzeug dienen, um \acr{dtls} zu realisieren.

Für die Berechnung von elliptischen Kurven wird direkt der Code, aus dem Bachelor-Projekt GOBI, von Jens Trillmann übernommen. Weitere Anpassungen
sind hier nicht notwendig, da die Berechnung auf einem gängigen Computer mit ausreichender Geschwindigkeit durchgeführt wird.

Die \acr{aes}-Verschlüsselung erfolgt auf der Serverseite durch den \glos{mc1322}. Da dieser hier nicht verfügbar ist, wird dafür die
crypto-Bibliothek von OpenSSL \cite{openssl} genutzt. 

Nach dem Vorbild von libcoap, sind die Module von \acr{dtls} ebenfalls in einem Archiv organisiert, das der Linker beim Kompilieren des Clients einbindet.

Bei Start des Clients, ruft dieser die Liste der verfügbaren Endgeräte vom Border-Router ab, so das folgende Aktionen möglich sind:
"`handshake | name | ecc | uuid | time | model | flash <nr>"'. Bevor Informationen von einem Endgerät abgerufen werden können, muss
ein \glos{handshake} durchgeführt werden. Ohne diesen versucht der Client derzeit, Anwendungsdaten in Epoche 0 zu übertragen, was vom
Server nicht akzeptiert, und mit einer Alert-Nachricht beantwortet wird. Da Alert-Nachrichten noch nicht vom Client berücksichtigt
werden, wird die Anfrage deshalb nach 90 Sekunden durch den \acr{coap}-Client mit einem Fehler abgebrochen. Neben der Berücksichtigung
von Alert-Nachrichten ist es auch noch notwendig, den Versand von Anwendungsdaten in Epoche 0 generell zu verhindern. Diese Dinge
wurden bisher vernachlässigt, da der Fokus auf eingeschränkten Umgebungen, und somit auf Seite des Servers, liegt.
\section{Testumgebung}

flashen zunächst generell an ttyUSB1

border-router per "`make border"' an ttyUSB1

sniffer per "`make listen2"' an ttyUSB3\\
"`make listen"' würde direkt in wireshark pipen. funktioniert nicht

server nach initialem flash -> update per client über coap