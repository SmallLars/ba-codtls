\chapter{Vorgehensweise}

Im Vordergrund soll die Implementierung eines Sicherheitsprotokolls stehen, das sich an den Prinzipien von \acr{dtls} orientiert
und einige der Vorschläge aus dem Internet-Entwurf von K. Hartke und O. Bergmann, und/oder eigene Ideen, realisiert.
Dabei liegt ein besonderes Interesse darauf den Handshake über \acr{coap} \cite{draftcoap} zu realisieren und somit einige in \acr{dtls} eingefügte
Konzepte überflüssig zu machen. Die Implementierung wird im Anschluss durch einen Vergleich mit \acr{dtls} evaluiert, wobei folgende Punkte eine
Rolle spielen sollen: Volumen des generierten Traffics, Größe/Komplexität des Quellcodes und die vom Protokoll benötigte Speichermenge.

Die Implementierung besteht aus dem Clienten auf einem gängigen PC/Laptop, bei dem es keine speziellen Einschränkungen an Energie,
Speicher oder Effizienz gibt, und aus dem Server der für einen Redbee Econotag \cite{econotag} mit dem MC1322X
Mikrocontroller \cite{mc1322x} optimiert werden soll. Da der genannte Mikrocontroller die Verschlüsselung mit dem \acr{aes} im \acr{ctr}- und \acr{cbc}-Mode
in Hardware unterstützt und die Rechenleistung sowie der Speicher beschränkt ist, soll nur ein \glos{ciphersuit} "`TLS\_PSK\_ECDH\_WITH\_AES\_128\_CCM\_8"',
in Anlehnung an den RFC 6655 \cite{rfc6655}, realisiert werden. Dieses gibt einen Schlüsselaustausch mit
Hilfe elliptischer Kurven vor, die mit kleineren Zahlen genau so sicher sind wie ein standard Diffie-Hellman-Schlüsselaustausch und sich somit
effizient berechnen lassen. Zusätzlich wird auch ein \acr{psk} verwendet, damit der Verbindungsaufbau nur den Clienten möglich ist, die über diesen
Schlüssel verfügen. Die Verschlüsselung der Anwendungsdaten erfolgt dann im "`\acr{aead}"' Modus \cite{rfc5116}
wobei sich hier "`\acr{ccm}"' \cite{rfc3610} aufgrund der Hardwarevorraussetzungen am besten eignet. Dieser besteht
aus einer Verschlüsselung der Daten durch \acr{aes} im \acr{ctr}-Modus während der dazugehörende \acr{mac} durch \acr{aes} im \acr{cbc}-Modus berechnet wird.
Die Anzahl der möglichen sicheren Verbindungen soll beschränkt werden, um den Speicherverbrauch gering zu halten.
Die dafür notwendigen verbindungsspezifischen Daten wie bspw. der Key-Block, sollen dabei im Flash-Speicher des Redbee Econotags
abgelegt werden, so dass nur die für das aktuelle Paket benötigten Daten im RAM-Speicher befinden. Auf diese Weise lässt sich
die Anzahl der möglichen sicheren Verbindungen trotz Beschränkung maximieren.

Bei der Evaluation soll die Datenmenge der Header-Daten sowohl beim Verbindungsaufbau als auch bei der Übertragung von Anwendungsdaten
mit einer reinen \acr{dtls}-Implementierung verglichen werden. Dabei werden nicht nur fehlerfreie Verbindungen betrachtet sondern auch Paketverluste
mit einbezogen. Verglichen wird auch der notwendige Speicherbedarf bei Verbindungsaufbau und Übertragung der Anwendungsdaten.
Ebenso soll die Größe/Komplexität des zugrunde liegenden Quellcodes bewertet werden.