\chapter{Anpassungen}

Da die maximale Datenmenge in \acr{6lowpan}-Paketen auf 127 Byte begrenzt ist würde der in DTLS definierte Header mit 13 Byte schon mehr als 10\% des
Datenvolumens ausmachen. Um das zu vermeiden wird die Stateless Header Compression aus dem Entwurf von K. Hartke und O. Bergmann \cite{draftcodtls}
angewendet. Diese zeichnet sich durch eine verlustfreie Komprimierung aus, für die keine weiteren Informationen bereitgestellt werden müssen. Damit
lässt sich der Header im besten Fall auf 2 Byte, wie in Abbildung \ref{fig:com_handshake_header} dargestellt, komprimieren.

\begin{figure}[ht]
  \centering
  \begin{lstlisting}[language=c]
   0 1 2 3 4 5 6 7 8 9 0 1 2 3 4 5
  +-+-+-+-+-+-+-+-+-+-+-+-+-+-+-+-+
  |0| T | V |  E  |1 1 0|  S  | L |
  +-+-+-+-+-+-+-+-+-+-+-+-+-+-+-+-+
  \end{lstlisting}
  \caption{Komprimierter Handshake-Header}
  \label{fig:com_handshake_header}
\end{figure}

Der RecordType (T) kann mit 2 Bit folgende 4 Zustände annehmen: \textit{type\_8\_bit} (0), \textit{alert} (1), \textit{dtls\_data} (2) und \textit{application\_data} (3).
Diese weichen von dem oben genannten Entwurf und \acr{dtls} ab, da der Handshake über \acr{coap} realisiert wird. Somit ist es erforderlich, zwischen
Anwendungsdaten, \acr{dtls}-Daten, welche in \acr{coap}-Paketen transportiert werden, und Daten des Alert-Protokolls zu unterscheiden. \acr{dtls}-Daten sind die
Daten die während eines Handshakes ausgetauscht werden womit dort auch das ChangeChiperSpec-Protokoll enthalten ist. Das Alert-Protokoll ist nach wie vor separat,
da hier eine zuverlässige Datenübertragung nicht erforderlich ist und \acr{coap} somit nicht benutzt wird. Sollten weitere RecordTypes erforderlich sein, ist es möglich
T auf 0 zu setzten und einen 8 Bit langen RecordType an den Header anzuhängen.

Die Codierung der Version (V) wurde ohne Änderungen übernommen und kann mit 2 Bit folgende 4 Zustände annehmen: \textit{dtls\_1\_0} (0), \textit{version\_16\_bit} (1),
\textit{dtls\_1\_2} (2) und \textit{version\_future\_use} (3). Auch hier ist es möglich weitere Versionen an den Header anzuhängen, in dem V auf 1 gesetzt wird wobei
hier mit 2 Byte das in \acr{tls} definierte Versionsformat zum Einsatz kommt.

Auch die Epoche (E) wurde ohne Änderungen übernommen. \TODO{es geht noch weiter ...}


% typedef enum {
%     epoch_0 = 0,
%     epoch_1 = 1,
%     epoch_2 = 2,
%     epoch_3 = 3,
%     epoch_4 = 4,
%     epoch_8_bit = 5,
%     epoch_16_bit = 6,
%     epoch_implicit = 7 // same as previous record in the datagram
% } Epoch;
% 
% typedef enum {
%     snr_0 = 0,
%     snr_8_bit = 1,
%     snr_16_bit = 2,
%     snr_24_bit = 3,
%     snr_32_bit = 4,
%     snr_40_bit = 5,
%     snr_48_bit = 6,
%     snr_implicit = 7 // number of previous record in the datagram + 1
% } SequenceNumber;
% 
% typedef enum {
%     rec_length_0 = 0,
%     rec_length_8_bit = 1,
%     rec_length_16_bit = 2,
%     rec_length_implicit = 3 // datagram size - sizeof(DTLSRecord_t) or last datagram in record
% } RecordLength;