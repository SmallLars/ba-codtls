\section{Handshake}

Wie in Kapitel \ref{chp:dtls} beschrieben, wurde der Header des \acr{dtls}-Handshake-Protokolls um eine Sequenz-Nummer
und 2 Datenfelder für die Fragmentierung ergänzt. Diese sind zunächst notwendig, um die in \acr{dtls} fehlenden Mechanismen
auszugleichen. Da der Handshake nun über \acr{coap} realisiert wird, können diese Datenfelder jedoch wieder wegfallen.
\TODO{begründung und weiter schreiben}

% - separat
% - blockwise

\begin{figure}[ht]
  \centering
  \begin{lstlisting}[language=c]
                      Client           Server
                      ------           ------

                  POST /dtls -------->
             ClientHello

                             <-------- 1.02 Verify
                                           HelloVerifyRequest

                  POST /dtls -------->
             ClientHello
            (mit cookie)

                             <-------- 2.01 Created
                                           ServerHello (enthält Session X)
                                          *Certificate
                                           ServerKeyExchange
                                          *CertificateRequest
                                           ServerHelloDone

              POST /dtls?s=X -------->
             Certificate*
       ClientKeyExchange
       CertificateVerify*
        ChangeCipherSpec
                Finished

                             <-------- 2.04 Changed
                                           ChangeCipherSpec
                                           Finished

            Application Data <-------> Application Data
  \end{lstlisting}
  \caption{Nachrichtenaustausch während eines TLS / DTLS Handshakes über CoAP}
  \label{fig:newhandshake}
\end{figure}