\subsection{Update-Funktion}

Damit Endgeräte im Einsatz ein Softwareupdate erhaltenen können, ohne diese direkt mit einem Computer zu verbinden,
dient die Ressource "`/f"'. Mit Hilfe dieser kann der Programmcode aktualisiert werden, ohne die Daten im
erweiterten Flash-Speicher zu überschreiben. So bleiben unter anderem der Werksmäßig definierte \acr{psk} und die \acr{uuid} erhalten.

Um ein IEEE 802.15.4 Paket mit maximal 127 Byte Nutzdaten voll auszuschöpfen, wird hier auf den Einsatz der Block-1-Option verzichtet, und
die neue Software in Blöcken von 46 Byte übertragen. Würde hier die Block-1-Option genutzt, müsste die Blockgröße 32 Byte betragen, womit
sich die Anzahl der notwendigen Pakete um \textasciitilde 50 \% erhöhen würde. Damit der Block identifiziert werden kann, wird vor jedem Block ein zwei
Byte langer Index übertragen. Ist dieser null, handelt es sich um den ersten Block und der Flash-Speicher wird gelöscht, damit die neue Software
dort hinterlegt werden kann. Hat der Index den Wert $ 0 $xFFFF folgen keine weiteren Blöcke und das Endgerät wird neu gestartet.

Das führt zum Verlust der Session-Daten, wodurch ein Handshake erneut erforderlich ist. Um dies zu verhindern ist es zukünftig denkbar,
die erforderlichen Daten vor einem Neustart zu sichern, um diese im Anschluss wieder herzustellen.

Bedingt durch die zuverlässige Übertragung kann der Client sicherstellen, dass alle Datenblöcke angekommen sind, bevor der Neustart ausgelöst wird.
\TODO{ausfall während der übertragung - error}