\subsection{Contiki-App: "`time"'}

Im Gegensatz zu herkömmlichen Desktop-Rechnern oder Servern verfügt der Redbee Econotag über keine innere Uhr bezüglich der Realzeit.
Angeboten wird vom \glos{mc1322} das Register "`MACA\_CLK"' welches mit einem Takt von 250 KHz erhöht wird. Dieses läuft, bedingt durch
seine Breite von 32 Bit, jedoch alle 4,77 Stunden über, so dass ohne weitere Eingriffe keine direkte Berechnung der Zeit möglich ist.
Ähnlich verhält es sich mit dem Register "`CRM\_RTC\_COUNT"' welches im Takt von "`CRM\_RTC\_TIMEOUT"' Hz erhöht wird.
Dieser Wert wird von Contiki eingestellt und liegt bei \textasciitilde 20 KHz. Dadurch erfolgt hier ein Überlauf nach ungefähr 60 Stunden.
Neben dem Überlauf haben beide Quellen das Problem, dass die Register bei Einschalten des Econotags bei 0 anfangen, und die Werte somit
keinerlei Bezug zur Realzeit haben.

Contiki löst einen Teil der genannten Probleme und stellt die Funktion "`clock\_seconds"' zur Verfügung. Diese kümmert sich um den Überlauf
und berechnet laufend die, seit dem Einschalten des Econotags, vergangenen Sekunden. Diese werden in einer 32-bit-Variablen gespeichert,
wodurch ein Überlauf erst nach \textasciitilde 136 Jahren vorkommen kann.

Um den Bezug zur Realzeit herzustellen, wird die aktuelle Unixzeit vom Blaster generiert und im RO-Teil des Flash-Speichers hinterlegt.
Diese Zeit spiegelt somit den Herstellungszeitpunkt wieder. Wird die aktuelle Zeit benötigt, kann diese durch Addition der von Contiki
ermittelten Sekunden und der Unixzeit berechnet werden. Das funktioniert natürlich nur so lange, wie der Redbee Econotag nach dem Flashen
ununterbrochen mit Strom versorgt wird. Um eine Korrektur zu ermöglichen, bietet die App eine Methode an, um die aktuelle Uhrzeit zu setzen.
Diese wird mit der alten Zeit vergleichen um einen Korrekturwert zu ermitteln der in einer globalen Variablen hinterlegt wird. Hier macht es
keinen Sinn diesen im Flash-Speicher abzulegen, da er nach einem Batteriewechsel direkt wieder veraltert wäre.