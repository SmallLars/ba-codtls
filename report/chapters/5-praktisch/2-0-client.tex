\section{Client}
\label{sec:client}

Als Grundlage für den Clienten wird die Implementierung aus dem Bachelorprojekt \glos{gobi} übernommen. Diese ist in der Lage, die im Sensornetz eingebundenen
Endgeräte vom Border-Router abzurufen, und in einer Liste darzustellen. Über den Listen-Index ist es dann möglich, mit den Endgeräten zu kommunizieren,
so dass nicht jedesmal die IP-Adresse angegeben werden muss.

Während der Client bisher direkt über \acr{udp} mit den Endgeräten kommuniziert hat, wird nun die libcoap \cite{libcoap} genutzt, um eine Kommunikation
über \acr{coap} zu realisieren. Da es in dieser Arbeit primär um die Realisierung von \acr{dtls} gehen soll, wird der \acr{coap}-Client, aus den in der
Bibliothek enthaltenen Beispielen, übernommen, und für die Verwendung engepasst. Anpassungen sind hier notwendig da der \acr{coap}-Client ein
Kommandozeilen-Tool ist. Die enthaltenen main()-Methode wird dafür umbenannt, und die Ausgabe der Antwort erfolgt in den vom Aufrufer übergebenen Speicher.
Außerdem wird die fehlende Funktionalitat ergänzt, Antworten mit einer Block-2-Option zu empfangen, die auf eine Separate-Antwort oder eine Block-1-Anfrage
folgen. Hier übernimmt der Server die Kontrolle über die Datenübertragung, während der Client Empfangsbestätigungen sendet. Eine Block-1-Anfrage wird
nicht durch den \acr{coap}-Clienten selbst realisiert, sondern muss manuell, durch mehrere Anfragen, umgesetzt werden. Dieser erweiterte \acr{coap}-Client
ist nicht für die Verwendung in anderen Projekten gedacht, und soll hier nur als provisorisches Werkzeug dienen, um \acr{dtls} zu realisieren.

Für die Berechnung von elliptischen Kurven wird direkt der Code, aus dem Bachelorprojekt \glos{gobi}, von Jens Trillmann übernommen. Weitere Anpassungen
sind hier nicht notwendig, da die Berechnung auf einem gängigen Computer mit ausreichender Geschwindigkeit durchgeführt wird.

Die \acr{aes}-Verschlüsselung erfolgt auf der Serverseite durch den \glos{mc1322}. Da dieser hier nicht verfügbar ist, wird dafür die
crypto-Bibliothek von OpenSSL \cite{openssl} genutzt. 

Nach dem Vorbild von libcoap, sind die Module von \acr{dtls} ebenfalls in einem Archiv organisiert, das der Linker beim Kompilieren des Clients einbindet.

Bei Start des Clients, ruft dieser die Liste der verfügbaren Endgeräte vom Border-Router ab, so das folgende Aktionen möglich sind:
"`handshake | name | ecc | uuid | time | model | flash <nr>"'. Bevor Informationen von einem Endgerät abgerufen werden können, muss
ein \glos{handshake} durchgeführt werden. Ohne diesen versucht der Client derzeit, Anwendungsdaten in Epoche 0 zu übertragen, was vom
Server nicht akzeptiert, und mit einer Alert-Nachricht beantwortet wird. Da Alert-Nachrichten noch nicht vom Client berücksichtigt
werden, wird die Anfrage deshalb nach 90 Sekunden durch den \acr{coap}-Client mit einem Fehler abgebrochen. Neben der Berücksichtigung
von Alert-Nachrichten ist es auch noch notwendig, den Versand von Anwendungsdaten in Epoche 0 generell zu verhindern. Diese Dinge
wurden bisher vernachlässigt, da der Fokus auf eingeschränkten Umgebungen, und somit auf Seite des Servers, liegt.