\section{Server}
\label{sec:server}

Der Server wird auf einem Redbee Econotag \cite{econotag} realisiert. Der darauf enthaltene Mikrocontroller \glos{mc1322} \cite{mc1322} enthält,
neben dem IEEE 802.15.4 Funkstandard und einer \acr{aes} Hardware-Engine, 128 KiB Flash-Speicher und 96 KiB RAM-Speicher. Bei Inbetriebnahme
wird das im Flash-Speicher vorliegende Programm vollständig in den RAM-Speicher kopiert und dort ausgeführt, wodurch sich eine maximale
Programmgröße von 96 KiB ergibt. Die zusätzlichen 32 KiB Flash-Speicher können somit für die Ablage von Daten genutzt werden, die auch nach einer
Stromunterbrechung, oder einem Neustart des Geräts, erhalten bleiben sollen. Zu berücksichtigen ist jedoch auch noch, dass der letzte 4 KiB große
Block schon für den Redbee Econotag selbst reserviert ist.

Betrieben wird der Server mit SmartAppContiki \cite{erbium}, das auf Contiki \cite{contiki} basiert, und eine Implementierung von \acr{coap}, in
der Entwurfsversion 13 \cite{draftcoap13}, enthält. In der Standardkonfiguration benötigt SmartAppContiki, mit einer definierten \acr{coap}-Ressource,
die ein "`Hallo Welt!"' zurückgibt, 83,73 KiB. Diese Daten teilen sich gemäß Abbildung \ref{tbl:contiki-speicher} auf. Um den benötigten
Speicher zu optimieren wurde die Größe des "`Sys Stack"' und des "`Heap"' in der Konfigurationsdatei "`contiki/cpu/mc1322x/mc1322x.lds"' angepasst.
Außerdem wurde "`REST\_MAX\_CHUNK\_SIZE"' in der Contiki-App "`Erbium"' von 128 auf 48 Byte reduziert, wodurch das Datensegment weniger Speicher benötigt.
Diese Konstante definiert die maximale Größe der Anwendungsdaten, die in einem Contiki-Paket untergebracht werden können, und stellt somit sicher,
dass jedes \acr{coap}-Paket in ein einzelnes IP-Paket passt.

\begin{figure}[!ht]
\centering
\begin{tabular}{l|r|r}
  \hiderowcolors
  \textbf{Beschreibung} & \textbf{Standard} & \textbf{Angepasst}\\
  \hline
  Programm        & ~~~~~58760 Byte    & ~~~~~58760 Byte\\
  Irq Stack       &   256 Byte         &   256 Byte\\
  Fiq Stack       &   256 Byte         &   256 Byte\\
  Svc Stack       &   256 Byte         &   256 Byte\\
  Abt Stack       &    16 Byte         &    16 Byte\\
  Und Stack       &    16 Byte         &    16 Byte\\
  Sys Stack       &  1024 Byte         &  2048 Byte\\
  Datensegment    & 21064 Byte         & 20744 Byte\\
  Heap            &  4096 Byte         &    16 Byte\\
  \hline
  \textbf{Gesamt} & 85744 Byte         & 82368 Byte\\
                  & 83,73 KiB          & 80,44 KiB\\
  \showrowcolors
\end{tabular}
\caption{Speicheraufteilung von SmartAppContiki}
\label{tbl:contiki-speicher}
\end{figure}

Das wurde möglich durch Verwendung der in Contiki eingebauten Beobachtungswerkzeuge. Durch Definieren von periodischen Ausgaben der benutzen Heap
sowie Sys Stack Größe, in "`contiki/platform/redbee-econotag/contiki-mc1322x-main.c"', können die Auslastungen beobachtet werden. Um diesen Prozess
effizienter zu gestalten, wird nur die Initialisierung durchgeführt und die periodischen Ausgaben deaktiviert. In "`server/server.c"' lässt sich nun,
durch Aktivieren des Debug-Modus, ein Code einbinden, der auf Knopfdruck sowohl die Speicheraufteilung als auch die bisher genutzten Bytes des Sys Stack
und Heap ausgibt. Dadurch lässt sich erkennen, dass der Heap garnicht benutzt wird, und somit unnötig Speicher belegt. Da insbesondere während des \glospl{handshake},
unter anderem aufgrund der Berechnung von elliptischen Kurven, viele Daten zwischengespeichert werden müssen, wird ersichtlich, dass ein Sys Stack von 1024
Byte nicht ausreicht, eine Größe von 2048 Byte jedoch optimal ist. Durch diese Anpassungen wurde der, für SmartAppContiki benötigte, Speicher von 83,73 KiB
auf 80,44 KiB reduziert (siehe Abbildung \ref{tbl:contiki-speicher}). Somit stehen für die Umsetzung von \acr{dtls} \textasciitilde 15,5 KiB zur
Verfügung, wobei auch berücksichtig werden muss, dass noch die Funktionen des Geräts selbst implementiert werden müssen.

Bei der Benutzung der, in SmartAppContiki enthaltenen, \acr{coap} 13 Implementierung, hat sich herausgestellt, dass die Unterstützung für die \acr{coap}-Option
Block-1 fehlt. Diese Option kann von einem Clienten benutzt werden, um größere Datenmengen, in einer \acr{coap}-Anfrage, in Blöcke zu unterteilen, damit es nicht
zu einer Fragmentierung auf IP-Ebene kommt. Laut Aussage von Matthias Kovatsch wird "`auf die atomare Variante aus Platzgründen verzichtet, da man Block1 ganz
einfach im Resource-Handler lösen kann"`. Da diese Option für den \acr{dtls}-\glos{handshake} generell benötigt wird, und ohne Code-Duplizierung auch anderen Ressourcen
zur Verfügung stehen soll, wird sie in ähnlicher Form wie die Separate-Option implementiert. Das Separate-Modul bietet Methoden an, um den Client, während der
Bearbeitung einer Anfrage, zu informieren, dass die Bearbeitung einige Zeit dauert, und die Beantwortung der Anfrage später fortzusetzen. In diesem Sinne bietet
das Block-1-Modul eine Methode an, mit der die Parameter der Block-1-Option überprüft und bearbeitet werden, wobei bei Bedarf die entsprechenden Fehler generiert
werden, oder die erhaltenen Daten auf Wunsch zusammengesetzt werden. Genau wie das Separate-Modul kann das Block1-Modul optional in einer Ressource benutzt werden.
Dabei ist nach wie vor ein Empfang von Daten ohne Block-1-Option möglich. Durch den Rückgabewert der Methode, lässt sich in der Ressource entscheiden, ob schon
die einzelnen Datenblöcke bearbeitet werden, oder erst die vollständige Nachricht, nach Erhalt aller Blöcke. Damit das Separate-Modul auch in Kombination mit dem
Block-1-Modul genutzt werden kann, wurde das Separate-Modul entsprechend angepasst um die Block-1-Option zu berücksichtigen.

Während bisher allgemeine Anpassungen von SmartAppContiki bzw. dem darin enthaltenen \acr{coap} 13 beschrieben wurden, folgt in den nächsten vier Abschnitten
die Erläuterung von vier implementierten Contiki-Apps, welche für die Realisierung von \acr{dtls} benutzt werden. Die Implementierung von \acr{dtls} wird dann
im 5. Abschnitt erläutert, wonach abschließend noch eine Update-Funktion erläutert wird, die für \acr{dtls} nicht notwendig ist, aber dessen Umfeld berücksichtigt.

\subsection{Contiki-App: "`flash-store"'}

Für eine Nutzung des erweiterten Flash-Speichers wird die App "`flash-store"' verwendet. Als Basis für die Implementierung dient Code aus dem Bachelorprojekt
\glos{gobi}. Dieser war jedoch noch nicht als Contiki-App organisiert, sondern direkt mit in den Code eingebunden. Auch ist die Aufteilung der 4 KiB großen
Flash-Speicher-Blöcke eine Andere. Diese Aufteilung ist in Abbildung \ref{tbl:2-1_2-persistent} zu sehen. Während oben die 8 Speicherblöcke mit ihren Adressen
aufgeführt sind, werden darunter die Aufteilungen für unterschiedliche Zwecke angegeben, wobei dort sowohl die \glos{gobi}-Aufteilung als auch die neue
Aufteilung aufgeführt sind.

\begin{figure}[!ht]
\centering
\renewcommand{\arraystretch}{1.5}
\begin{tabular}{|p{1.4cm}|p{1.4cm}|p{1.4cm}|p{1.4cm}|p{1.4cm}|p{1.4cm}|p{1.4cm}|p{1.4cm}|}
  \hiderowcolors
  \hline
  $ 0x18000\newline -\newline 0x18 $FFF & $0x19000\newline -\newline 0x19 $FFF & $ 0x1 $A$ 000\newline -\newline 0x1 $AFFF & $ 0x1 $B$ 000\newline -\newline 0x1 $BFFF & $ 0x1 $C$ 000\newline -\newline 0x1 $CFFF & $ 0x1 $D$ 000\newline -\newline 0x1 $DFFF & $ 0x1 $E$ 000\newline -\newline 0x1 $EFFF & $ 0x1 $F$ 000\newline -\newline 0x1 $FFFF\\
  \hline
  \multicolumn{8}{l}{~}\\
  \multicolumn{8}{l}{Aufteilung innerhalb des Bachelorprojekts \glos{gobi}:}\\
  \hline
  RO 1 & \multicolumn{2}{l|}{RW 1} & \multicolumn{2}{l|}{RW 2} & \multicolumn{2}{l|}{RO 2} & SR\\
  \hline
  \multicolumn{1}{c|}{~} & \multicolumn{2}{c|}{$ 0x0000 - 0x0 $FFF} & \multicolumn{2}{c|}{$ 0x1000 - 0x1 $FFF} & \multicolumn{3}{c}{$ \leftarrow $ virtuelle Speicheradressen} \\
  \cline{2-5}
  \multicolumn{8}{l}{~}\\
  \multicolumn{8}{l}{Neue Aufteilung für \acr{dtls}:}\\
  \hline
  \multicolumn{2}{|l|}{RW 1} & \multicolumn{2}{l|}{RW 2} & RAD & \multicolumn{2}{l|}{RO} & SR\\
  \hline
  \multicolumn{2}{|c|}{$ 0x0000 - 0x0 $FFF} & \multicolumn{2}{c|}{$ 0x1000 - 0x1 $FFF} & \multicolumn{1}{c}{~} & \multicolumn{3}{c}{$ \leftarrow $ virtuelle Speicheradressen} \\
  \cline{1-4}
  \multicolumn{8}{l}{~}\\
  \multicolumn{8}{l}{Legende: RW = Read-Write, RAD = Read-Append-Delet, RO = Read-Only, SR = System-Reserved}\\
  \showrowcolors
\end{tabular}
\renewcommand{\arraystretch}{1.0}
\caption{Aufteilung des erweiterten Flash-Speichers}
\label{tbl:2-1_2-persistent}
\end{figure}

Geändert wurde zunächst die Position der beiden RW-Blöcke. Diese ermöglichen das Schreiben von Daten, ohne die Eigenschaften des Flash-Speichers berücksichtigen
zu müssen. Dieser kann nur beschrieben werden, falls die betroffene Position vorher einmal gelöscht wurde, was sich aber nur in Blöcken a 4 KiB realisieren lässt.
Um Datenverluste zu vermeiden, werden jeweils zwei 4 KiB große Blöcke benutzt, um einen 4 KiB großen Speicher zu realisieren, der sich durch virtuelle Adressen
ansprechen lässt, welche ebenfalls in Abbildung \ref{tbl:2-1_2-persistent} aufgeführt sind. Die Daten sind immer nur in einem Block gespeichert, während der andere
Block gelöscht ist. Kommt es zu einem Schreibvorgang, wird der Datenblock in den leeren Block kopiert, wobei die gewünschten Änderungen realisiert werden.
Die Position der beiden RW-Blöcke befindet sich nun am Anfang, da sich so die Adressen, der jeweils zusammengehörenden Blöcke, genau um ein Bit unterscheiden.
Das vereinfacht die Berechnung der Quell- und Ziel-Adresse erheblich, so dass durch Optimierung des Quellcodes 70 Byte an Programmgröße eingespart werden.

Problemtisch ist jedoch die Dauer und der Energieverbrauch bei einem Schreibzugriff dieser Art. Um eine effizientere Ablage von Daten zu ermöglichen folgt
nach den beiden RW-Blöcken anstatt des RO-Blocks nun ein RAD-Block. Dieser ist vergleichbar mit einem Stack ohne Push- und Pop-Funktion. Für die Initialisierung
wird der komplette Block gelöscht. Daten können nun so lange eingefügt werden, bis der Block voll ist. Wieviel Daten gerade enthalten sind, wird dabei in einer
globalen Variablen im RAM-Speicher gespeichert. Der Lesezugriff kann dabei beliebig erfolgen.

Gleich geblieben ist die Position des RO-Blocks. Dort können im Vorfeld, durch das bereits erwähnte Programm "`Blaster"', Daten abgelegt werden, welche zur Laufzeit
ausgelesen werden können. Dies spart Programmgröße, da diese "`Konstanten"' nicht im Datensegment des Programms enthalten sind.

Abschließend folgt noch ein Block der für den Redbee Econotag selbst reserviert ist, und somit nicht genutzt werden kann.
\subsection{Contiki-App: "`time"'}

Im Gegensatz zu herkömmlichen Desktop-Rechnern oder Servern verfügt der Redbee Econotag über keine innere Uhr bezüglich der Realzeit.
Angeboten wird vom \glos{mc1322} das Register "`MACA\_CLK"' welches mit einem Takt von 250 KHz erhöht wird. Dieses läuft, bedingt durch
seine Breite von 32 Bit, jedoch alle 4,77 Stunden über, so dass ohne weitere Eingriffe keine direkte Berechnung der Zeit möglich ist.
Ähnlich verhält es sich mit dem Register "`CRM\_RTC\_COUNT"' welches im Takt von "`CRM\_RTC\_TIMEOUT"' Hz erhöht wird.
Dieser Wert wird von Contiki eingestellt und liegt bei \textasciitilde 20 KHz. Dadurch erfolgt hier ein Überlauf nach ungefähr 60 Stunden.
Neben dem Überlauf haben beide Quellen das Problem, dass die Register bei Einschalten des Econotags bei 0 anfangen und die Werte somit
keinerlei Bezug zur Realzeit haben.

Contiki löst einen Teil der genannten Probleme und stellt die Funktion "`clock\_seconds"' zur Verfügung. Diese kümmert sich um den Überlauf
und berechnet laufend die, seit dem Einschalten des Econotags, vergangenen Sekunden. Diese werden in einer 32-bit-Variablen gespeichert,
wodurch ein Überlauf erst nach \textasciitilde 136 Jahren vorkommen kann.

Um den Bezug zur Realzeit herzustellen, wird die aktuelle Unixzeit vom Blaster generiert und im RO-Teil des Flash-Speichers hinterlegt.
Diese Zeit spiegelt somit den Herstellungszeitpunkt wieder. Wird die aktuelle Zeit benötigt, kann diese durch Addition der von Contiki
ermittelten Sekunden und der Unixzeit berechnet werden. Das funktioniert natürlich nur so lange, wie der Redbee Econotag nach dem Flashen
ununterbrochen mit Strom versorgt wird. Um eine Korrektur zu ermöglichen, bietet die App eine Methode an, um die aktuelle Uhrzeit zu setzen.
Diese wird mit der alten Zeit vergleichen um einen Korrekturwert zu ermitteln, der in einer globalen Variablen hinterlegt wird. Hier macht es
keinen Sinn, diesen im Flash-Speicher abzulegen, da er nach einem Batteriewechsel direkt wieder veraltert wäre.
\subsection{Contiki-App: "`aes"'}

Um die \acr{aes}-Funktionen des \glos{mc1322} \cite{mc1322} zu nutzen, dient diese Contiki-App. Die bereits beschriebene \glos{ciphersuite}
verwendet in der \acr{psr} AES-CMAC \cite{rfc4493}. Außerdem wird für die Verschlüsselung AES-CCM \cite{rfc3610} verwendet. Beide Verfahren
werden vom \glos{mc1322} nicht direkt unterstützt. Bereitgestellt wird nur der reine \acr{AES} Verschlüsselungsprozess im \acr{ctr}- und \acr{cbc}-Mode.

Um die \acr{aes}-Hardware nutzen zu können, muss diese zunächst initialisiert werden. Die dafür notwendige Methode wurde zum Großteil aus dem Bachelorprojekt GOBI
übernommen und leicht modifiziert. Neben der Aktivierung der \acr{aes}-Hardware wir dort ein Selbsttest mit einem internen Schlüssel durchgeführt. Ist dieser
erfolgreich, werden die beiden Modi \acr{ctr} und \acr{cbc}, für die spätere Nutzung, aktiviert.


% 	uint32_t KEY0;
% 	uint32_t KEY1;
% 	uint32_t KEY2;
% 	uint32_t KEY3;
% 	uint32_t DATA0;
% 	uint32_t DATA1;
% 	uint32_t DATA2;
% 	uint32_t DATA3;
% 	uint32_t CTR0;
% 	uint32_t CTR1;
% 	uint32_t CTR2;
% 	uint32_t CTR3;
% 	uint32_t CTR0_RESULT;
% 	uint32_t CTR1_RESULT;
% 	uint32_t CTR2_RESULT;
% 	uint32_t CTR3_RESULT;
% 	uint32_t CBC0_RESULT;
% 	uint32_t CBC1_RESULT;
% 	uint32_t CBC2_RESULT;
% 	uint32_t CBC3_RESULT;
% 	uint32_t MAC0;
% 	uint32_t MAC1;
% 	uint32_t MAC2;
% 	uint32_t MAC3;
\input{chapters/5-praktisch/1-4-ecc.tex}
\subsection{Contiki-App: "`er-13-dtls"'}

es kann nur ein handshake zur zeit stattfinden\\
psk wird nur benötigt bis pre master secret berechnet\\
am ende des handshakes wird alter psk vernichtet und neuer generiert\\
kann durch uri /psk über sichere leitung abgerufen werden so dass weitere handshakes durchgeführt werden können

für hello verify cookie: cmac(client-ip + clienthello)

verschlüsselung von finish mit seq\_num 0. erste app daten beginnen also bei seq\_num 1.\\
seq\_num wird im ram gehalten da bei flash update bei jedem paket hoher verschleiß.\\
sollte seq\_num je 0 sein, wird sie mit der seq\_num der anfrage "`wiederhergestellt"' (erläuterung: ist nicht exakt)

uri syntax \cite{rfc3986}. beachten wegen session-id

Übertragen werden mit Hilfe der Option ein ClientHello oder ClientKeyExchange + ChangeCipherSpec + Finished, wobei die Länge von Letzterem mit
insgesamt 114 (87 + 3 + 22) Byte, durch die einzig definierte \glos{ciphersuite}, fest ist. Anders ist dies bei dem ClientHello, welches durch
eine Vielzahl, vom Client beherrschter \glospl{ciphersuite}, sehr viel größer werden kann.

Hier erfolgt dann jedoch eine \acr{coap}-Fehlermeldung "`4.13 REQUEST ENTITY TOO LARGE"' mit einem Hinweis auf die begrenzte Maximalgröße,
so dass ein Client sein Angebot an \glospl{ciphersuite} reduzieren kann, um der maximalen Größe gerecht zu werden.

auswertung cookie -> query. cookie in packet ist doof
\subsection{Update-Funktion}

Damit Endgeräte im Einsatz ein Softwareupdate erhaltenen können, ohne diese direkt mit einem Computer zu verbinden,
dient die Ressource "`/f"'. Mit Hilfe dieser kann der Programmcode aktualisiert werden, ohne die Daten im
erweiterten Flash-Speicher zu überschreiben. So bleiben unter anderem der Werksmäßig definierte \acr{psk} und die \acr{uuid} erhalten.

Um ein IEEE 802.15.4 Paket mit maximal 127 Byte Nutzdaten voll auszuschöpfen, wird hier auf den Einsatz der Block-1-Option verzichtet, und
die neue Software in Blöcken von 46 Byte übertragen. Würde hier die Block-1-Option genutzt, müsste die Blockgröße 32 Byte betragen, womit
sich die Anzahl der notwendigen Pakete um \textasciitilde 50 \% erhöhen würde. Damit der Block identifiziert werden kann, wird vor jedem Block ein zwei
Byte langer Index übertragen. Ist dieser null, handelt es sich um den ersten Block und der Flash-Speicher wird gelöscht, damit die neue Software
dort hinterlegt werden kann. Hat der Index den Wert $ 0 $xFFFF folgen keine weiteren Blöcke und das Endgerät wird neu gestartet.

Das führt zum Verlust der Session-Daten, wodurch ein Handshake erneut erforderlich ist. Um dies zu verhindern ist es zukünftig denkbar,
die erforderlichen Daten vor einem Neustart zu sichern, um diese im Anschluss wieder herzustellen.

Bedingt durch die zuverlässige Übertragung kann der Client sicherstellen, dass alle Datenblöcke angekommen sind, bevor der Neustart ausgelöst wird.
\TODO{ausfall während der übertragung - error}