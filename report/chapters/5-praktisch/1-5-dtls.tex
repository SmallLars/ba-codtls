\subsection{Contiki-App: "`er-13-dtls"'}

es kann nur ein handshake zur zeit stattfinden\\
psk wird nur benötigt bis pre master secret berechnet\\
am ende des handshakes wird alter psk vernichtet und neuer generiert\\
kann durch uri /psk über sichere leitung abgerufen werden so dass weitere handshakes durchgeführt werden können

für hello verify cookie: cmac(client-ip + clienthello)

verschlüsselung von finish mit seq\_num 0. erste app daten beginnen also bei seq\_num 1.\\
seq\_num wird im ram gehalten da bei flash update bei jedem paket hoher verschleiß.\\
sollte seq\_num je 0 sein, wird sie mit der seq\_num der anfrage "`wiederhergestellt"' (erläuterung: ist nicht exakt)

uri syntax \cite{rfc3986}. beachten wegen session-id

Übertragen werden mit Hilfe der Option ein ClientHello oder ClientKeyExchange + ChangeCipherSpec + Finished, wobei die Länge von Letzterem mit
insgesamt 114 (87 + 3 + 22) Byte, durch die einzig definierte \glos{ciphersuite}, fest ist. Anders ist dies bei dem ClientHello, welches durch
eine Vielzahl, vom Client beherrschter \glospl{ciphersuite}, sehr viel größer werden kann.

Hier erfolgt dann jedoch eine \acr{coap}-Fehlermeldung "`4.13 REQUEST ENTITY TOO LARGE"' mit einem Hinweis auf die begrenzte Maximalgröße,
so dass ein Client sein Angebot an \glospl{ciphersuite} reduzieren kann, um der maximalen Größe gerecht zu werden.

auswertung cookie -> query. cookie in packet ist doof