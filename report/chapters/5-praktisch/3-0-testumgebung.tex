\section{Entwicklungsumgebung}

Für die Erstellung der Programme und die Weiterentwicklung von einigen Werkzeugen wird im Allgemeinen Ubuntu 12.04.2 LTS (32 bit) benutzt.
Damit bei einem Betrieb der Endgeräte an einem Windows-Recher, ebenfalls die vom Redbee Econotag an die USB-Schnittstelle gesendeten
Daten abgerufen werden können, wird ein Skript für die Windows-PowerShell-Umgebung erstellt, wobei zusätzliche Treiber notwendig sind.

Im Bachelorprojekt GOBI wurde die libmc1322x \cite{libmc1322x} angepasst, damit die von Blaster produzierten Dateien auf den
\glos{mc1322} geladen werden können. Um diesen Flash-Vorgang durchzuführen, sind zwei Komponenten notwendig. Zum einen ist dies eine
Software für den \glos{mc1322}, Flasher genannt, die zuerst in den RAM-Speicher übertragen wird, um im Anschluss die Software für
das Endgerät anzunehmen, und im Flash-Speicher abzulegen. Zum anderen ist dies eine Software, Loader genannt, die sowohl die erste
Software als auch die Software für das Endgerät per USB an den \glos{mc1322} sendet. Während der Flasher ausschließlich in C
implementiert ist, sind für den Loader eine C-Variante und eine Perl-Variante verfügbar. Angepasst wurden im Bachelorporjekt GOBI
nur die beiden C-Programme, wobei diese sich ausschließlich für von Blaster produzierte Dateien eignen. Um diese Situation zu verbessern,
werden die beiden C-Programme und das Perl-Programm erneut angepasst. Zur gewährleistung der Abwärtskompatibilität, sind beide Varianten
so angepasst, dass ohne weitere Aktionen die gängigen Dateien übertragen werden können. Erst durch Nutzung des Parameters -l, bei einem
Flash-Vorgang, wird das von Blaster produzierten Dateieformat berücksichtigt. Zusätzlich wird in die libmc1322x eine Methode eingefügt,
die den Neustart des \glos{mc1322} ermöglicht. Die Methode, zum Vergleich von Daten im RAM-Speicher mit Daten im Flash-Speicher, aus dem
Bachelorprojekt GOBI, wird übernommen. Die Änderungen an der libmc1322x sind in einem Fork des originalen Repositorys veröffentlicht,
und wurden Mariano Alvira zur Übernahme angeboten.

Um die Überwachung in Wireshark \cite{wireshark} angemessen zu realisieren, wird ein Dissector ergänzt. Dieser wertet die übertragenen
Daten des angepassten \acr{dtls}-Protokolls aus und stellt diese übersichtlich dar. Da die Übertragung des ersten \glospl{handshake}
unverschlüsselt erfolgt, werden die Daten aus Epoche 0 an den bereits enthaltenen \acr{coap}-Dissector weitergereicht, so dass
diese ebenfalls dargestellt werden. Da der \acr{coap}-Dissector blockweise übertragene Daten nicht zusammensetzt, ist eine Darstellung
der \glos{handshake}-Daten selbst bisher nicht möglich.

Für den Server, den Border-Router und den Sniffer sind die Make-Regeln clear und upload verfügbar. upload überträgt die Programme in den
Flash-Speicher des Redbee Econotag, während clear den Flash-Speicher löscht. Für die beiden Vorgänge ist generell /dev/ttyUSB1 vorgesehen.
Um die, an die USB-Schnittstelle gesendeten, Daten des Servers abzurufen, kann die Regel listen verwendet werden. Die Netzwerkbrücke
zur Inbetriebnahme des Border-Routers wird mit der Regel border erzeugt. Für den Sniffer sind schließlich die Regeln listen und listen2
vorhanden. listen leitet die Daten direkt in Wireshark, während listen2 die Daten in die Datei cap.pcap schreibt.