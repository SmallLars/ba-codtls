
\newacronym{acr:ssl}{SSL}{Secure Sockets Layer}

\newacronym{acr:tls}{TLS}{Transport Layer Security}

\newacronym{acr:dtls}{DTLS}{Datagram Transport Layer Security}

\newacronym{acr:ietf}{IETF}{Internet Engineering Task Force}

\newacronym{acr:tcp}{TCP}{Transmission Control Protocol}

\newacronym{acr:udp}{UDP}{User Datagram Protocol}

\newacronym{acr:coap}{CoAP}{Constrained Application Protocol}

\newacronym{acr:wot}{WoT}{Web of Things}

\newacronym{acr:6lowpan}{6LoWPAN}{IPv6 over Low power Wireless Personal Area Network}

\newacronym{acr:rom}{ROM}{Read-Only Memory}

\newacronym{acr:ram}{RAM}{Random-Access Memory}

\newacronym{acr:ccm}{CCM}{Counter with CBC-MAC}

\newacronym{acr:aead}{AEAD}{Authenticated Encryption with Associated Data}

\newacronym{acr:mac}{MAC}{Message Authentication Code}

\newacronym{acr:aes}{AES}{Advanced Encryption Standard}

\newacronym{acr:dos}{DoS}{Denial of Service}

\newacronym{acr:psk}{PSK}{Pre-Shared Key}

\newacronym{acr:ack}{ACK}{Acknowledgement}

\newacronym{acr:uri}{URI}{Uniform Resource Identifier}

\newacronym{acr:sha}{SHA}{Secure Hash Algorithm}

\newacronym{acr:psr}{PSR}{Pseudo-Random-Funktion}

% \newacronym{label}{kurz}{lang}
% Zugriff via \gls{label} und Co.
% 
% Beispiel:
% 
%   \newacronym{acr:da}{DA}{Diplomarbeit}
% 
% wird im Text
% 
%   Heute schreibe ich meine \gls{arc:da}. Diese \gls{arc:da}.
% 
% zu
% 
%   Heute schreibe ich meine Diplomarbeit (DA). Diese DA.
% 
% Nur bei der ersten Verwendung der Abkürzung wird diese in langer Form
% dargestellt. Weitere Vorkommen nutzen die Kurz-Version.
% 
% Neben \gls{} gibt es auch noch \Gls{} und \GLS{}, welche die Groß- und
% Klein-Schreibung beeinflussen. Beispiel:
% 
%   \newacronym{acr:http}{http}{Hypertext Transfer Protocol}
%   \gls{acr:http} --> http
%   \Gls{acr:http} --> Http
%   \GLS{acr:http} --> HTTP
% 
% Zur Vereinfachung gibt es auch die Befehle \acr{}, \Acr{} und \ACR{}. Diese
% Befehle sind äquivalent:
% 
%   \gls{acr:foo} <=> \acr{foo}
%   \Gls{acr:foo} <=> \Acr{foo}
%   \GLS{acr:foo} <=> \ACR{foo}
%

